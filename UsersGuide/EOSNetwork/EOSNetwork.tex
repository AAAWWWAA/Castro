\section{Equation of State}
\castro\ is written in a modular fashion so that the EOS and network
burning routines can be supplied by the user.   However, for the
examples presented later we use several EOS and network routines
that come with the \castro\ distribution.  

EOS routines that come with \castro\ are (listed by directory name):
\begin{itemize}

\item {\tt gamma\_law\_general} represents a gamma law gas, with equation of state:
\begin{equation}
  P = (\gamma - 1) \rho e.
\end{equation}
The gas is currently assumed to be monatomic and ideal. The entropy comes from the Sackur-Tetrode equation.

\item {\tt polytrope} represents a polytropic fluid, with equation of state:
\begin{equation}
  P = K \rho^\gamma.
\end{equation}
The gas is also assumed to obey the above gamma law equation of state
connecting the pressure and internal energy. Therefore $\rho$ is the
only independent variable; there is no temperature dependence. The
user either selects from a set of predefined options reflecting
physical polytropes (e.g. a non-relativistic, fully degenerate
electron gas) or inputs their own values for $K$ and $\gamma$.

\item {\tt ztwd} is the zero-temperature degenerate electron equation of state 
of Chandrasekhar (1935), which is designed to describe white dward material. The 
pressure satisfies the equation:
\begin{equation}
  P(x) = A \left( x(2x^2-3)(x^2 + 1)^{1/2} + 3\, \text{sinh}^{-1}(x) \right),
\end{equation}
with $A = \pi m_e^4 c^5 / (3 h^3)$. Here $x$ is a dimensionless measure of density,
with $\rho = B x^3$ and $B = 8\pi \mu_e m_p m_e^3 c^3 / (3h^3)$, 
where $\mu_e$ is the electron fraction and $h$ is the Planck constant. The enthalpy 
was worked out by Hachisu (1986):
\begin{equation}
  h(x) = \frac{8A}{B}\left(x^2 + 1\right)^{1/2}.
\end{equation}
The internal energy satisfies the standard relationship to the enthalpy:
\begin{equation}
  e = h + p / \rho.
\end{equation}
Since the pressure-density relationship does not admit a closed-form 
solution for the density in terms of the pressure, if we call the EOS 
with pressure as a primary input then we do Newton-Raphson iteration 
to find the density that matches this pressure.

\item {\tt multigamma} is an ideal gas equation of state where each
  species can have a different value of $\gamma$.  This mainly affects
  how the internal energy is constructed as each species, represented
  with a mass fraction $X_k$ will have its contribution to the total
  specific internal energy take the form of $e = p/\rho/(\gamma_k -
  1)$.  The main thermodynamic quantities take the form:
\begin{align}
p &= \frac{\rho k T}{m_u} \sum_k \frac{X_k}{A_k} \\
e &= \frac{k T}{m_u} \sum_k \frac{1}{\gamma_k - 1} \frac{X_k}{A_k} \\
h &= \frac{k T}{m_u} \sum_k \frac{\gamma_k}{\gamma_k - 1} \frac{X_k}{A_k}
\end{align}
We recognize that the usual astrophysical $\bar{A}^{-1} = \sum_k
X_k/A_k$, but now we have two other sums that involve different
$\gamma_k$ weightings.

The specific heats are constructed as usual, 
\begin{align}
c_v &= \left . \frac{\partial e}{\partial T} \right |_\rho = 
    \frac{k}{m_u} \sum_k \frac{1}{\gamma_k - 1} \frac{X_k}{A_k} \\
c_p &= \left . \frac{\partial h}{\partial T} \right |_p = 
    \frac{k}{m_u} \sum_k \frac{\gamma_k}{\gamma_k - 1} \frac{X_k}{A_k} 
\end{align}
and it can be seen that the specific gas constant, $R \equiv c_p - c_v$ is
independent of the $\gamma_i$, and is simply $R = k/m_u\bar{A}$ giving the
usual relation that $p = R\rho T$.  Furthermore, we can show
\begin{equation}
\Gamma_1 \equiv \left . \frac{\partial \log p}{\partial \log \rho} \right |_s =  
   \left ( \sum_k \frac{\gamma_k}{\gamma_k - 1} \frac{X_k}{A_k} \right ) \bigg /
   \left ( \sum_k \frac{1}{\gamma_k - 1} \frac{X_k}{A_k} \right ) =
\frac{c_p}{c_v} \equiv \gamma_\mathrm{effective} 
\end{equation}
and $p = \rho e (\gamma_\mathrm{effective} - 1)$.

This equation of state takes several runtime parameters that can set the
$\gamma_i$ for a specific species.  These are set in the {\tt \&extern}
namelist in the {\tt probin} file.  The parameters are:
\begin{itemize}
\item {\tt eos\_gamma\_default}: the default $\gamma$ to apply for
  all species
\item {\tt species\_X\_name} and {\tt species\_X\_gamma}: set the $\gamma_i$
  for the species whose name is given as {\tt species\_X\_name} to the
  value provided by {\tt species\_X\_gamma}.  Here, {\tt X} can be one
  of the letters: {\tt a}, {\tt b}, or {\tt c}, allowing us to specify
  custom $\gamma_i$ for up to three different species.
\end{itemize}

\item {\tt helmeos} contains a general, publicly available
stellar equation of state based on the Helmholtz free energy,
with contributions from ions, radiation, and electron degeneracy, as
described in (Timmes and Arnett 1999, Times and Swesty 2000, and Fryxell et al. 2000).

%% \item {\bf LattimerSwestyEOS} directory contains a modified version of the
%% LS EOS available at http://www.astro.sunysb.edu/dswesty.  Full
%% documentation is available through that web site.  We use this EOS
%% in the 1D core collapse supernova example in a later section.
\end{itemize}

Each EOS should have two main routines by which it interfaces to the
rest of \castro.  At the beginning of the simulation, {\tt eos\_init}
will perform any initialization steps and save EOS variables (mainly
\texttt{smallt}, the temperature floor, and \texttt{smalld}, the
density floor). Then, whenever you want to call the EOS, use
\[
{\bf \rm eos}(\texttt{eos\_input, eos\_state, do\_eos\_diag, pt\_index}).
\]
The first argument specifies the inputs to the EOS, which will be held
constant. The options that are currently available are stored in
{\tt EOS/eos\_data.F90}, and are always a combination of two
thermodynamic quantities. For real (non-analytic) equations of state
in which $\rho$, $T$ and species are the independent variables, such
as the Helmholtz EOS, {\tt eos\_input\_rt} directly calls the EOS
and obtains the other thermodynamic variables (internal energy,
pressure, $\gamma$, sound speed, etc.). For other inputs,
e.g. {\tt eos\_input\_re}, a Newton-Raphson iteration is performed
to find the density or temperature that corresponds to the given
input.

The {\tt eos\_state} variable is a Fortran derived type (similar to
a C++ struct). It stores a complete set of thermodynamic
variables. When calling the EOS, you should first fill the variables
that are the inputs, for example with
\begin{align*}
  &\texttt{eos\_state } \% \texttt{ rho = state(i,j,k,URHO)} \\
  &\texttt{eos\_state } \% \texttt{ e   = state(i,j,k,UEINT) / state(i,j,k,URHO)} \\
  &\texttt{eos\_state } \% \texttt{ xn  = state(i,j,k,UFS:UFS+nspec-1) / state(i,j,k,URHO)} \\
  &\texttt{eos\_state } \% \texttt{ aux = state(i,j,k,UFX:UFX+naux-1) / state(i,j,k,URHO)}.
\end{align*}
Whenever the \texttt{eos\_state} type is initialized, the
thermodynamic state variables are filled with unphysical numbers. If
you do not input the correct arguments to match your input quantities,
the EOS will call an error. However, this means that it is good
practice to fill the quantities that will be iterated over with an
initial guess. The values they are initialized with will likely not
converge. Usually a prior value of the temperature or density suffices
if it's available, but if not then use \texttt{small\_temp} or
\texttt{small\_dens}. \textbf{Please note that the auxiliary variables
  now go in a separate field in the derived type, {\tt eos\_state \%
  aux}. The original \castro\ behavior was to pack them in with the
  species abundances, xn.}

The last two arguments are optional quantities: set \texttt{do\_eos\_diag} to \texttt{.true.} if you want diagnostic output as the iterations occur, and set \texttt{pt\_index} to be equal to the indices of the current zone if you want the zone information to be displayed on any error messages.

Please note that there is an important difference with how the
Helmholtz EOS is treated now, compared to \castro's original behavior,
when the temperature or density reaches their respective floors
({\tt small\_temp} and {\tt small\_dens}) in a Newton-Raphson
iteration (such as when you call it with
{\tt eos\_input\_re}). Previously, we would have the outgoing state
be thermodynamically inconsistent because the density or temperature
would be floored but the internal energy would be the same as the
input. Now, by default, the outgoing state is always thermodynamically
consistent, which means that the argument that was previously held
constant may substantially differ from its input. In our testing, we
have found that retaining this thermodynamic consistency is necessary
for accurately treating shocks. If you wish to disable this behavior,
set \texttt{eos\_input\_is\_constant = F} in your \texttt{extern}
namelist in your probin file.

\section{Burning Network}
Burning network routines that come with \castro\ are (listed by directory name):
\begin{itemize}
\item {\tt Networks/general\_null} directory is a bare interface for a
  nuclear reaction network, but no reactions are enabled, and no
  auxiliary variables are accepted. It contains several sets of
  isotopes; for example,
  {\tt Networks/general\_null/triple\_alpha\_plus\_o.net} would describe the
  triple-$\alpha$ reaction converting helium into carbon, as well as
  oxygen and iron.
\item {\tt Networks/ignition\_simple} directory contains a single-step
  $^{12}\mathrm{C}(^{12}\mathrm{C},\gamma)^{24}\mathrm{Mg}$ reaction.
  The carbon mass fraction equation appears as
\begin{equation}
\frac{D X(^{12}\mathrm{C})}{Dt} = - \frac{1}{12} \rho X(^{12}\mathrm{C})^2
    f_\mathrm{Coul} \left [N_A \left <\sigma v \right > \right]\enskip,
\end{equation}
where $N_A \left <\sigma v\right>$ is evaluated using the reaction
rate from (Caughlan and Fowler 1988).  The Coulomb screening factor,
$f_\mathrm{Coul}$, is evaluated using the general routine from the
Kepler stellar evolution code (Weaver 1978), which implements the work
of (Graboske 1973) for weak screening and the work of (Alastuey 1978
and Itoh 1979) for strong screening.
\end{itemize}

There are two primary files within each network directory. The first,
{\tt castro\_burner.f90}, contains the burner routine, which takes
$\rho^\inp, e^\inp, X_k^\inp$, and $\Delta t$ as inputs.  It is
possible for the internal energy, $e$, which is computed from $\Ub$,
to be negative due to roundoff error.  \castro\ has an option to protect
against using a negative value of $e$ by recomputing $e =
e(\rho,T_{\rm small},X_k)$ using the equation of state, where $T_{\rm
  small}$ is a user-defined temperature floor.  In the event that $e$
is still negative, we abort the program.  \castro\ also has an option to
skip the reactions if the density is below a user-defined density
floor.

Next, the burner computes $T = T(\rho^\inp,e^\inp,X_k^\inp)$ using the
equation of state.  The burner returns $X_k^\outp$ and $e^\outp$ by
solving over a time interval of $\Delta t/2$,
\begin{eqnarray}
\frac{\partial X_k}{\partial t} &=& \omegadot_k.\\
\end{eqnarray}
In particular, to evolve the species, we solve the system:
\begin{eqnarray}
\frac{dX_k}{dt} &=& \omegadot_k(\rho,X_k,T)\enskip, \label{eq:VODE1C} \\
\frac{dT}{dt} &=&\frac{1}{c_p} \left ( -\sum_k \xi_k  \omegadot_k  \right )\enskip. \label{eq:tempreactC}
\end{eqnarray}
\MarginPar{Need to include temperature evolution equation somewhere}
using the stiff ordinary differential equation integration methods
provided by the VODE package.  The absolute error tolerances are set
to $10^{-12}$ for the species, and a relative tolerance of $10^{-5}$
is used for the temperature.  The integration yields the new values of
the mass fractions, $X_k^\outp$.  Equation (\ref{eq:tempreactC}) is
derived from equation (???) by assuming that the pressure is constant
during the burn state.  In evolving these equations, we need to
evaluate $c_p$ and $\xi_k$.  In theory, this means evaluating the
equation of state for each right-hand side evaluation that VODE
requires.  In practice, we freeze $c_p$ and $\xi_k$ at the start of
the integration time step and compute them using $\rho^\inp,
X_k^\inp,$ and $T^\inp$ as inputs to the equation of state.  Note that
the density remains unchanged during the burning.  At the end of the
routine, we compute $T^\outp = T(\rho^\outp,e^{\outp},X_k^\outp)$.

The second file, {\tt network.f90}, supply the number of species and
auxiliary variables, names of each species and auxiliary variable, as
well as other initializing data, such as aion, zion and the binding
energy.

It is straightforward to implement additional EOS and network
routines; all that is required is to create an appropriate interface
to the \castro\ calls, which is easily done given the prototypes
supplied with the \castro\ distribution.
