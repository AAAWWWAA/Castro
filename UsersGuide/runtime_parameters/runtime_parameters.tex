
\label{ch:parameters}


%%%%%%%%%%%%%%%%
% symbol table
%%%%%%%%%%%%%%%%

\begin{landscape}


{\small

\renewcommand{\arraystretch}{1.5}
%
\begin{center}
\begin{longtable}{|l|p{5.25in}|l|}
\caption[ diagnostics
 parameters.]{ diagnostics
 parameters.} \label{table:  diagnostics
 parameters. runtime} \\
%
\hline \multicolumn{1}{|c|}{\textbf{parameter}} & 
       \multicolumn{1}{ c|}{\textbf{description}} & 
       \multicolumn{1}{ c|}{\textbf{default value}} \\ \hline 
\endfirsthead

\multicolumn{3}{c}%
{{\tablename\ \thetable{}---continued}} \\
\hline \multicolumn{1}{|c|}{\textbf{parameter}} & 
       \multicolumn{1}{ c|}{\textbf{description}} & 
       \multicolumn{1}{ c|}{\textbf{default value}} \\ \hline 
\endhead

\multicolumn{3}{|r|}{{\em continued on next page}} \\ \hline
\endfoot

\hline 
\endlastfoot


\rowcolor{tableShade}
\verb=  job_name  = &    &  "" \\
\verb=  print_energy_diagnostics  = &   display breakdown of energy sources  &  (0, \\
\rowcolor{tableShade}
\verb=  print_fortran_warnings  = &   display warnings in Fortran90 routines  &  (0, \\
\verb=  show_center_of_mass  = &   display center of mass diagnostics  &  0 \\
\rowcolor{tableShade}
\verb=  sum_interval  = &   how often to compute integral sums (for runtime diagnostics)  &  -1 \\


\end{longtable}
\end{center}

} % ends \small


{\small

\renewcommand{\arraystretch}{1.5}
%
\begin{center}
\begin{longtable}{|l|p{5.25in}|l|}
\caption[ embiggening
 parameters.]{ embiggening
 parameters.} \label{table:  embiggening
 parameters. runtime} \\
%
\hline \multicolumn{1}{|c|}{\textbf{parameter}} & 
       \multicolumn{1}{ c|}{\textbf{description}} & 
       \multicolumn{1}{ c|}{\textbf{default value}} \\ \hline 
\endfirsthead

\multicolumn{3}{c}%
{{\tablename\ \thetable{}---continued}} \\
\hline \multicolumn{1}{|c|}{\textbf{parameter}} & 
       \multicolumn{1}{ c|}{\textbf{description}} & 
       \multicolumn{1}{ c|}{\textbf{default value}} \\ \hline 
\endhead

\multicolumn{3}{|r|}{{\em continued on next page}} \\ \hline
\endfoot

\hline 
\endlastfoot


\rowcolor{tableShade}
\verb=  grown_factor  = &   the factor by which to extend the domain upon restart for embiggening  &  1 \\
\verb=  star_at_center  = &   used with the embiggening routines to determine how to extend the domain  &  -1 \\


\end{longtable}
\end{center}

} % ends \small


{\small

\renewcommand{\arraystretch}{1.5}
%
\begin{center}
\begin{longtable}{|l|p{5.25in}|l|}
\caption[ gravity
 parameters.]{ gravity
 parameters.} \label{table:  gravity
 parameters. runtime} \\
%
\hline \multicolumn{1}{|c|}{\textbf{parameter}} & 
       \multicolumn{1}{ c|}{\textbf{description}} & 
       \multicolumn{1}{ c|}{\textbf{default value}} \\ \hline 
\endfirsthead

\multicolumn{3}{c}%
{{\tablename\ \thetable{}---continued}} \\
\hline \multicolumn{1}{|c|}{\textbf{parameter}} & 
       \multicolumn{1}{ c|}{\textbf{description}} & 
       \multicolumn{1}{ c|}{\textbf{default value}} \\ \hline 
\endhead

\multicolumn{3}{|r|}{{\em continued on next page}} \\ \hline
\endfoot

\hline 
\endlastfoot


\rowcolor{tableShade}
\verb=  do_grav  = &   permits gravity calculation to be turned on and off  &  -1 \\
\verb=  grav_source_type  = &    &  2 \\
\rowcolor{tableShade}
\verb=  moving_center  = &   to we recompute the center used for the multipole gravity solve each step?  &  0 \\


\end{longtable}
\end{center}

} % ends \small


{\small

\renewcommand{\arraystretch}{1.5}
%
\begin{center}
\begin{longtable}{|l|p{5.25in}|l|}
\caption[ hydrodynamics
 parameters.]{ hydrodynamics
 parameters.} \label{table:  hydrodynamics
 parameters. runtime} \\
%
\hline \multicolumn{1}{|c|}{\textbf{parameter}} & 
       \multicolumn{1}{ c|}{\textbf{description}} & 
       \multicolumn{1}{ c|}{\textbf{default value}} \\ \hline 
\endfirsthead

\multicolumn{3}{c}%
{{\tablename\ \thetable{}---continued}} \\
\hline \multicolumn{1}{|c|}{\textbf{parameter}} & 
       \multicolumn{1}{ c|}{\textbf{description}} & 
       \multicolumn{1}{ c|}{\textbf{default value}} \\ \hline 
\endhead

\multicolumn{3}{|r|}{{\em continued on next page}} \\ \hline
\endfoot

\hline 
\endlastfoot


\rowcolor{tableShade}
\verb=  add_ext_src  = &   if true, define an additional source term  &  0 \\
\verb=  allow_negative_energy  = &    &  1 \\
\rowcolor{tableShade}
\verb=  cg_maxiter  = &    &  12 \\
\verb=  cg_tol  = &    &  1.0e-5 \\
\rowcolor{tableShade}
\verb=  difmag  = &   the coefficient of the artifical viscosity  &  0.1 \\
\verb=  do_hydro  = &   permits hydro to be turned on and off for running pure rad problems  &  -1 \\
\rowcolor{tableShade}
\verb=  do_sponge  = &   permits sponge to be turned on and off  &  0 \\
\verb=  dual_energy_eta1  = &    &  1.0e0 \\
\rowcolor{tableShade}
\verb=  dual_energy_eta2  = &    &  1.0e-4 \\
\verb=  dual_energy_eta3  = &    &  0.0e0 \\
\rowcolor{tableShade}
\verb=  dual_energy_update_E_from_e  = &    &  1 \\
\verb=  fix_mass_flux  = &    &  0 \\
\rowcolor{tableShade}
\verb=  hard_cfl_limit  = &    &  1 \\
\verb=  hybrid_riemann  = &    &  0 \\
\rowcolor{tableShade}
\verb=  normalize_species  = &    &  0 \\
\verb=  ppm_flatten_before_integrals  = &    &  0 \\
\rowcolor{tableShade}
\verb=  ppm_predict_gammae  = &    &  0 \\
\verb=  ppm_reference  = &    &  1 \\
\rowcolor{tableShade}
\verb=  ppm_reference_edge_limit  = &    &  1 \\
\verb=  ppm_reference_eigenvectors  = &    &  0 \\
\rowcolor{tableShade}
\verb=  ppm_tau_in_tracing  = &    &  0 \\
\verb=  ppm_temp_fix  = &    &  0 \\
\rowcolor{tableShade}
\verb=  ppm_trace_sources  = &    &  0 \\
\verb=  ppm_type  = &    &  1 \\
\rowcolor{tableShade}
\verb=  riemann_solver  = &    &  0 \\
\verb=  small_dens  = &   the small density cutoff.  Densities below this value will be reset  &  -1.e200 \\
\rowcolor{tableShade}
\verb=  small_ener  = &   the small specific internal energy cutoff.  Pressures below this value will be reset  &  -1.e200 \\
\verb=  small_pres  = &   the small pressure cutoff.  Pressures below this value will be reset  &  -1.e200 \\
\rowcolor{tableShade}
\verb=  small_temp  = &   the small temperature cutoff.  Temperatures below this value will be reset  &  -1.e200 \\
\verb=  source_term_predictor  = &    &  0 \\
\rowcolor{tableShade}
\verb=  sum_turb_src  = &    &  0.0 \\
\verb=  transverse_reset_density  = &    &  0 \\
\rowcolor{tableShade}
\verb=  transverse_reset_rhoe  = &    &  0 \\
\verb=  transverse_use_eos  = &    &  0 \\
\rowcolor{tableShade}
\verb=  use_colglaz  = &    &  0 \\
\verb=  use_flattening  = &    &  1 \\
\rowcolor{tableShade}
\verb=  use_pslope  = &    &  1 \\


\end{longtable}
\end{center}

} % ends \small


{\small

\renewcommand{\arraystretch}{1.5}
%
\begin{center}
\begin{longtable}{|l|p{5.25in}|l|}
\caption[ parallelization
 parameters.]{ parallelization
 parameters.} \label{table:  parallelization
 parameters. runtime} \\
%
\hline \multicolumn{1}{|c|}{\textbf{parameter}} & 
       \multicolumn{1}{ c|}{\textbf{description}} & 
       \multicolumn{1}{ c|}{\textbf{default value}} \\ \hline 
\endfirsthead

\multicolumn{3}{c}%
{{\tablename\ \thetable{}---continued}} \\
\hline \multicolumn{1}{|c|}{\textbf{parameter}} & 
       \multicolumn{1}{ c|}{\textbf{description}} & 
       \multicolumn{1}{ c|}{\textbf{default value}} \\ \hline 
\endhead

\multicolumn{3}{|r|}{{\em continued on next page}} \\ \hline
\endfoot

\hline 
\endlastfoot


\rowcolor{tableShade}
\verb=  bndry_func_thread_safe  = &    &  1 \\
\verb=  do_acc  = &   determines whether we use accelerators for specific loops  &  -1 \\


\end{longtable}
\end{center}

} % ends \small


{\small

\renewcommand{\arraystretch}{1.5}
%
\begin{center}
\begin{longtable}{|l|p{5.25in}|l|}
\caption[ reactions
 parameters.]{ reactions
 parameters.} \label{table:  reactions
 parameters. runtime} \\
%
\hline \multicolumn{1}{|c|}{\textbf{parameter}} & 
       \multicolumn{1}{ c|}{\textbf{description}} & 
       \multicolumn{1}{ c|}{\textbf{default value}} \\ \hline 
\endfirsthead

\multicolumn{3}{c}%
{{\tablename\ \thetable{}---continued}} \\
\hline \multicolumn{1}{|c|}{\textbf{parameter}} & 
       \multicolumn{1}{ c|}{\textbf{description}} & 
       \multicolumn{1}{ c|}{\textbf{default value}} \\ \hline 
\endhead

\multicolumn{3}{|r|}{{\em continued on next page}} \\ \hline
\endfoot

\hline 
\endlastfoot


\rowcolor{tableShade}
\verb=  burning_timestep_factor  = &   limit the timestep based on how much the burning can change the internal energy of a zone.  The goal is to have the timestep equal to {\tt burning\_timestep\_factor}  $(e / \Delta e)$  &  1.e200 \\
\verb=  do_react  = &   permits reactions to be turned on and off -- mostly for efficiency's sake  &  -1 \\


\end{longtable}
\end{center}

} % ends \small


{\small

\renewcommand{\arraystretch}{1.5}
%
\begin{center}
\begin{longtable}{|l|p{5.25in}|l|}
\caption[ refinement
 parameters.]{ refinement
 parameters.} \label{table:  refinement
 parameters. runtime} \\
%
\hline \multicolumn{1}{|c|}{\textbf{parameter}} & 
       \multicolumn{1}{ c|}{\textbf{description}} & 
       \multicolumn{1}{ c|}{\textbf{default value}} \\ \hline 
\endfirsthead

\multicolumn{3}{c}%
{{\tablename\ \thetable{}---continued}} \\
\hline \multicolumn{1}{|c|}{\textbf{parameter}} & 
       \multicolumn{1}{ c|}{\textbf{description}} & 
       \multicolumn{1}{ c|}{\textbf{default value}} \\ \hline 
\endhead

\multicolumn{3}{|r|}{{\em continued on next page}} \\ \hline
\endfoot

\hline 
\endlastfoot


\rowcolor{tableShade}
\verb=  do_special_tagging  = &    &  0 \\
\verb=  spherical_star  = &    &  0 \\


\end{longtable}
\end{center}

} % ends \small


{\small

\renewcommand{\arraystretch}{1.5}
%
\begin{center}
\begin{longtable}{|l|p{5.25in}|l|}
\caption[ rotation
 parameters.]{ rotation
 parameters.} \label{table:  rotation
 parameters. runtime} \\
%
\hline \multicolumn{1}{|c|}{\textbf{parameter}} & 
       \multicolumn{1}{ c|}{\textbf{description}} & 
       \multicolumn{1}{ c|}{\textbf{default value}} \\ \hline 
\endfirsthead

\multicolumn{3}{c}%
{{\tablename\ \thetable{}---continued}} \\
\hline \multicolumn{1}{|c|}{\textbf{parameter}} & 
       \multicolumn{1}{ c|}{\textbf{description}} & 
       \multicolumn{1}{ c|}{\textbf{default value}} \\ \hline 
\endhead

\multicolumn{3}{|r|}{{\em continued on next page}} \\ \hline
\endfoot

\hline 
\endlastfoot


\rowcolor{tableShade}
\verb=  do_rotation  = &   permits rotation calculation to be turned on and off  &  -1 \\
\verb=  rot_axis  = &   the coordinate axis ($x=1$, $y=2$, $z=3$) for the rotation vector  &  3 \\
\rowcolor{tableShade}
\verb=  rot_source_type  = &    &  1 \\
\verb=  rotational_dPdt  = &   the rotation periods time evolution---this allows the rotation rate to change durning the simulation time  &  0.0 \\
\rowcolor{tableShade}
\verb=  rotational_period  = &   the rotation period for the corotating frame  &  -1.e200 \\


\end{longtable}
\end{center}

} % ends \small


\end{landscape}

%


