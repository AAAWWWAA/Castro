There are currently four options for how gravity is calculated -- 
these are controlled by setting {\bf gravity.gravity\_type}.  
The options are {\bf ConstantGrav, PoissonGrav, Monopole Grav} or {\bf PrescribedGrav}.
Note that these are only relevant if {\bf USE\_GRAV = TRUE} in the GNUmakefile
and {\bf castro.do\_grav} = 1 in the inputs file.  If both of these are
set then the user is required to specify the gravity type in the inputs file
or the program will abort. 

Note that  {\bf MonopoleGrav} and {\bf PoissonGrav} are only correct for
spherical stars, i.e. in 1D we must have {\bf coord\_sys} = 2, 
in 2D we must have {\bf coord\_sys} = 1, and in 3D we only support {\bf coord\_sys} = 0.

\section{Types of Approximations}

\begin{itemize}
\item {\bf ConstantGrav}: Gravity can be defined as constant in direction and magnitude, 
defined in the inputs file by 

\noindent {\bf gravity.const\_grav} = -9.8

for example, to set the gravity to have magnitude 9.8 in the 
negative y-direction if in 2D, negative z-direction if in 3-D.

\item {\bf PoissonGrav}: 
The most general case is time-evolving self-gravity with no assumptions about sphericity:
\begin{equation}
\mathbf{g}(\mathbf{x},t) = -\nabla \phi 
\end{equation}
is a self-induced gravitational field defined by solving
\begin{equation}
\mathbf{\Delta} \phi = 4 \pi G \rho .\label{eq:Self Gravity}
\end{equation}

We only allow {\bf PoissonGrav} in 2D or 3D because in 1D, computing the monopole 
approximation in spherical coordinates is faster and more accurate than solving
the Poisson equation.

In 2D or 3D we define boundary conditions for $\phi$ using the monopole approximation
at the coarsest level.  (Without this the boundary conditions would be homogeneous
Dirichlet which results in loss of sphericity of the gravitational field.)  We first
compute a 1D radial profile of the average density at the coarsest level, then integrate
that to define a 1D radial profile of the gravitational acceleration as below.  We
then integrate {\bf g} to define $\phi,$ starting with $\phi = 0$ at the center.

\item {\bf MonopoleGrav}: 
\begin{itemize}

\item In 1D spherical coordinates we compute 
\[g(r) = G * ({\rm Mass_{enclosed}}) / r^2 \; \; , \]
where ${\rm Mass_{enclosed}}$ is calculated from the density at the time of the call. 
For levels above the coarsest level we define the extent of that level's radial arrays 
as ranging from the center of the star ($r=0$) to the cell at that level farthest away from the
origin.  If there are gaps between fine grids in that range then we interpolate the density
from a coarser level in order to construct a continuous density profile.  We note that
the location of values in the density profile and in the gravitational field exactly match the
location of data at that level so there is no need to interpolate between points when
mapping the 1D radial profile of $g$ back onto the original grid.

\item In 2D or 3D we compute a 1D radial average of density and use this to compute 
gravity as a one-dimensional integral, then interpolate the gravity vector back onto 
the Cartesian grid cells. At the coarsest level we define the extent of the 1D arrays as
ranging from the center of the star to the farthest possible point in the grid (plus a
few extra cells so that we can fill ghost cell values of gravity).  At finer levels
we first define a single box that contains all boxes on that fine level, then we
interpolate density from coarser levels as needed to fill the value of density at every
fine cell in that box.   The extent of the radial array is from the center of the star
to the {\em nearest} cell on one of the faces of the single box.  This ensures that
all cells at that maximum radius of the array are contained in this box.

We then average the density onto a 1D radial array.
We note that there is a mapping from the Cartesian cells to the radial array and back;
unlike the 1D case this requires interpolation. We use quadratic interpolation with
limiting so that the interpolation does not create new maxima or minima.

The default resolution of the radial arrays at a level is the grid cell spacing at
that level, i.e. $\Delta r = \Delta x.$   One optimization we have recently added is that one 
can define {\bf castro.drdxfac} as a number greater than 1 (2 or 4 are recommended) and the
spacing of the radial array will then satisfy $\Delta x / \Delta r = $ {\bf drdxfac}.
Individual Cartesian grid cells are subdivided by {\bf drdxfac} in each coordinate direction
for the purposing of averaging the density, and the integration that creates $g$ is
done at the finer resolution of the new $\Delta r.$   

Note that the center of the star is defined in the subroutine PROBINIT and the radius
is computed as the distance from that center.

\end{itemize}

\item{\bf PrescribedGrav}: 

\noindent{\bf Note: The {\bf PrescribedGrav} option and text here were 
contributed by Jan Frederik Engels of University of Gottingen.}

With this option, gravity can be defined as a function that
is specified by the user.  The option is allowed in 2D and 3D.  To define the gravity
vector, copy {\bf prescribe\_grav\_2d.f90} from {\bf Src\_2d} to your run directory
(analogously copy {\bf prescribe\_grav\_3d.f90} from {\bf Src\_3d} if you're working in 3D).
The makefile system will always choose this local copy of the file over the one in another directory.
Then define the components of gravity inside a loop over the grid inside the file.
If your problem uses a radial gravity in the form $g(r)$, you can simply adapt 
{\bf ca\_prescribe\_grav\_gravityprofile}, otherwise you will have to adapt 
{\bf ca\_prescribe\_grav}, both are located in {\bf prescribed\_grav\_2d.90}.
\end{itemize}

\section{GR correction}

\noindent{\bf Note: The GR code and text here were contributed by Ken Chen of Univ. of Minnesota.}

In the cases of compact objects or very massive stars, the general relativity 
(GR) effect starts to play a role. First, we consider the hydrostatic equilibrium 
due to effects of GR then derive GR-correction term for Newtonian gravity. 
The correction term is applied to the monopole approximation only when 
{\bf USE\_GR  = TRUE} is set in the GNUmakefile. 

The formulae of GR-correction here are based on \cite{grbk1}. For detailed physics,
 please refer to \cite{grbk2}. For describing very strong gravitational field, we need to use Einstein 
field equations
\begin{equation}\label{field}
R_{ik}-\frac{1}{2}g_{ik}R=\frac{\kappa}{c^{2}}T_{ik} \quad , \quad
\kappa=\frac{8\pi G}{c^{2}}\quad ,
\end{equation} 
where $R_{ik}$ is the Ricci tensor, $g_{ik}$ is the metric tensor, $R$ is the Riemann curvature, $c$ is 
the speed of light and $G$ is gravitational constant. $T_{ik}$ is the energy momentum tensor, which for 
ideal gas has only the non-vanishing components $T_{00}$ = $\varrho c^2$ , $T_{11}$ = $T_{22}$ = $T_{33}$ = $P$ 
( contains rest mass and energy density, $P$ is pressure). We are interested in spherically symmetric mass 
distribution. Then the line element $ds$ for given spherical coordinate $(r, \vartheta, \varphi)$ has the 
general form
\begin{equation}\label{metric}
  ds^{2} = e^{\nu}c^{2}dt^{2}-e^{\lambda}dr^{2}-r^{2}(d\vartheta^{2}+\sin^{2}
  \vartheta d\varphi) \quad ,
\end{equation} 
with $\nu = \nu(r)$, $\lambda = \lambda(r)$. Now we can put the expression of $T_{ik}$ and $ds$ into (\ref{field}), then field equations can be reduced to 3 ordinary differential equations:
\begin{equation}\label{diff1}
   \frac{\kappa P}{c^{2}} =
   e^{-\lambda}(\frac{\nu^{\prime}}{r}+\frac{1}{r^{2}})-\frac{1}{r^{2}}
   \quad ,
\end{equation}
\begin{equation}\label{diff2}
  \frac{\kappa P}{c^{2}} =
  \frac{1}{2}e^{-\lambda}(\nu^{\prime\prime}+\frac{1}{2}{\nu^{\prime}}^{2}+\frac{\nu^
    {\prime}-\lambda^{\prime}}{r}
   -\frac{\nu^{\prime}\lambda^{\prime}}{2}) \quad ,
\end{equation}
\begin{equation}\label{diff3}
  \kappa \varrho =
  e^{-\lambda}(\frac{\lambda^{\prime}}{r}-\frac{1}{r^{2}})+\frac{1}{r^{2}} \quad ,
\end{equation} 
where primes means the derivatives with respect to $r$. After multiplying with $4\pi r^2$, (\ref{diff3}) can 
be integrated and yields
\begin{equation}\label{gmass1}
  \kappa m = 4\pi r (1-e^{-\lambda}) \quad ,
\end{equation}
the $m$ is called ``gravitational mass'' inside r defined as
\begin{equation}\label{gmass2}
  m = \int_{0}^{r}4\pi r^{2}  \varrho dr\quad .
\end{equation}      
For the $r = R$, $m$ becomes the mass $M$ of the star. $M$ contains not only the rest mass but
the whole energy (divided by $c^2$), that includes the internal and gravitational energy. So the
 $\varrho = \varrho_0 +U/c^2$ contains the whole energy density $U$ and rest-mass density $\varrho_0$. 
Differentiation of (\ref{diff1}) with respect to $r$ gives $P = P^{\prime}(\lambda,\lambda^{\prime},
\nu,\nu^{\prime},r)$, where $\lambda,\lambda^{\prime},\nu,\nu^{\prime}$  can be eliminated by (\ref{diff1}), 
(\ref{diff2}), (\ref{diff3}). Finally we reach \textit{Tolman-Oppenheinmer-Volkoff(TOV)} equation for 
hydrostatic equilibrium in general relativity:
\begin{equation}\label{tov}
  \frac{dP}{dr} = -\frac{Gm}{r^{2}}\varrho (1+\frac{P}{\varrho
    c^{2}})(1+\frac{4\pi r^3 P}{m c^{2}}) (1-\frac{2Gm}{r c^{2}})^{-1} \quad .
\end{equation}                  
For Newtonian case $c^2 \rightarrow  \infty $, it reverts to usual form
\begin{equation}\label{newton}
  \frac{dP}{dr} = -\frac{Gm}{r^{2}}\varrho \quad .
\end{equation}
Now we take effective monopole gravity as
\begin{equation}\label{tov2}                                                      
\tilde{g} = -\frac{Gm}{r^{2}} (1+\frac{P}{\varrho
  c^{2}})(1+\frac{4\pi r^3 P}{m c^{2}}) (1-\frac{2Gm}{r c^{2}})^{-1}  \quad .
\end{equation}                                 
For general situations, we neglect the $U/c^2$ and potential energy in m because they are usually
much smaller than $\varrho_0$. Only when $T$ reaches $10^{13} K$ ($KT \approx m_{p} c^2$, $m_p$ is proton mass)
 before it really makes a difference. So (\ref{tov2}) can be expressed as
\begin{equation}\label{tov3}                                                      
  \tilde{g} = -\frac{GM_{enc}}{r^{2}} (1+\frac{P}{\varrho
    c^{2}})(1+\frac{4\pi r^3 P}{M_{enc} c^{2}}) (1-\frac{2GM_{enc}}{r c^{2}})^{-1} \quad ,
\end{equation}                                              
where $M_{enc}$ is identical to $Mass_{enclosed}$ in the previous section.
