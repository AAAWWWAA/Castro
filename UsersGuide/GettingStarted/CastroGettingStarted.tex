
\section{Downloading the Code}

\castro\ is built on top of the \boxlib\ framework.  In order to run
\castro\, you must download two separate git modules.

\vspace{.1in}

\noindent First, make sure that {\tt git} is installed on your machine---we recommend version 1.7.x or higher.

\vspace{.1in}

\begin{enumerate}

\item Download the \boxlib\ repository by typing: 
\begin{verbatim}
git clone https://ccse.lbl.gov/pub/Downloads/BoxLib.git
\end{verbatim}

This will create a directory called {\tt BoxLib/} on your machine.
Put this somewhere out of the way and set the environment variable,
{\tt BOXLIB\_HOME}, on your machine to the path name where you have
put \boxlib.  You will want to periodically update \boxlib\ by typing
\begin{verbatim}
git pull
\end{verbatim}
in the \boxlib\ directory.  

\item Clone/fork the \castro\ repository from the \boxlib\ Codes {\sf
  github} page (\url{https://github.com/BoxLib-Codes/Castro}).  To
  clone via the commandline, simply type:
\begin{verbatim}
git clone https://github.com/BoxLib-Codes/Castro.git
\end{verbatim}
alternately, if you have a {\sf github} account with your
machine's SSH-keys registered, you can do:
\begin{verbatim}
git clone ssh://git@github.com:/BoxLib-Codes/Castro.git
\end{verbatim}

\end{enumerate}

%\clearpage

\section{Building the Code}

In \castro\ each different problem setup is stored in its own
sub-directory under {\tt Castro/Exec/}.  You build the
\castro\ executable in the problem sub-directory.  Here we'll
build the {\tt Sedov} problem:

\begin{enumerate}

\item From the directory in which you checked out the Castro git repo,
  type
\begin{verbatim}
cd Castro/Exec/Sedov
\end{verbatim}
This will put you into a directory in which you can run the Sedov
problem in 1-d, 2-d or 3-d.

\item In {\tt Sedov/}, edit the {\tt GNUmakefile}, and set

{\tt DIM = 2} (for example)

{\tt COMP = gcc} (or your favorite C++ compiler)

{\tt FCOMP = gfortran} (or your favorite Fortran 90 compiler)

{\tt DEBUG = FALSE}

If you want to try other compilers than the GNU suite and they don't
work, please let us know.

To build a serial (single-processor) code, set {\tt USE\_MPI = FALSE}.
This will compile the code without the MPI library.  If you want to do
a parallel run, then you would set {\tt USE\_MPI = TRUE}.  In this
case, the build system will need to know about your MPI installation.
This can be done by editing the makefiles in the \boxlib\ tree, but a
simple method is to set the shell environment variable {\tt
  BOXLIB\_USE\_MPI\_WRAPPERS=1}.  If this is set, then the build
system will fall back to using the local MPI compiler wrappers
(e.g.\ {\tt mpic++} and {\tt mpif90}) to do the build.

\item Now type {\tt make}.

  The resulting executable will look something like {\tt
    Castro2d.Linux.gcc.gfortran.ex}, which means this is a 2-d version
  of the code, made on a Linux machine, with {\tt COMP = gcc} and {\tt
    FCOMP = gfortran}.

\end{enumerate}

\section{Running the Code}

\begin{enumerate}

\item \castro\ takes an input file that overrides the runtime parameter defaults.
  The code is run as:
\begin{verbatim}
Castro2d.Linux.gcc.gfortran.ex inputs.2d.cyl_in_cartcoords
\end{verbatim}

This will run the 2-d cylindrical Sedov problem in Cartesian ($x$-$y$
coordinates).  You can see other possible options, which should be
clear by the names of the inputs files.

\item You will notice that running the code generates directories that
  look like {\tt plt00000/}, {\tt plt00020/}, etc, and {\tt chk00000/},
  {\tt chk00020/}, etc. These are ``plotfiles'' and ``checkpoint''
  files. The plotfiles are used for visualization, the checkpoint
  files are used for restarting the code.

\end{enumerate}

\section{Visualization of the Results}

There are several options for visualizing the data.  The popular
\visit\ package supports the \boxlib\ file format natively, as does
the \yt\ python package.  The standard tool used within the
\boxlib-community is \amrvis, which we demonstrate here.

\begin{enumerate}

\item Get \amrvis:
\begin{verbatim}
git clone https://ccse.lbl.gov/pub/Downloads/Amrvis.git
\end{verbatim}

Then cd into {\tt Amrvis/}, edit the {\tt GNUmakefile} there
to set {\tt DIM = 2}, and again set {\tt COMP} and {\tt FCOMP} to compilers that
you have. Leave {\tt DEBUG = FALSE}.

Type {\tt make} to build, resulting in an executable that
looks like {\tt amrvis2d...ex}.

If you want to build amrvis with {\tt DIM = 3}, you must first
download and build {\tt volpack}:
\begin{verbatim}
git clone https://ccse.lbl.gov/pub/Downloads/volpack.git
\end{verbatim}

Then cd into {\tt volpack/} and type {\tt make}.

Note: \amrvis\ requires the OSF/Motif libraries and headers. If you don't have these 
you will need to install the development version of motif through your package manager. 
{\tt lesstif} gives some functionality and will allow you to build the amrvis executable, 
but \amrvis\ may not run properly.

On most Linux distributions, the motif library is provided by the
{\tt openmotif} package, and its header files (like {\tt Xm.h}) are provided
by {\tt openmotif-devel}. If those packages are not installed, then use the
package management tool to install them, which varies from
distribution to distribution, but is straightforward. 

You may then want to create an alias to {\tt amrvis2d}, for example
\begin{verbatim}
alias amrvis2d /tmp/Amrvis/amrvis2d...ex
\end{verbatim}

\item Return to the {\tt Castro/Exec/Sedov} directory.  You should
  have a number of output files, including some in the form {\tt *pltXXXXX},
  where {\tt XXXXX} is a number corresponding to the timestep the file
  was output.  {\tt
    amrvis2d {\em filename}} to see a single plotfile, or {\tt amrvis2d -a
  *plt*}, which will animate the sequence of plotfiles.

  Try playing
  around with this---you can change which variable you are
  looking at, select a region and click ``Dataset'' (under View)
  in order to look at the actual numbers, etc. You can also export the
  pictures in several different formats under "File/Export".

Please know that we do have a number of conversion routines to other
formats (such as matlab), but it is hard to describe them all. If you
would like to display the data in another format, please let us know
(again, {\tt asalmgren@lbl.gov}) and we will point you to whatever we have
that can help.

\end{enumerate}

You have now completed a brief introduction to \castro. 


\section{Other Distributed Problem Setups}

There are a number of standard problem setups that come with \castro.
These can be used as a starting point toward writing your own setup.
The standard problems are:
\begin{itemize}

\item {\tt double\_bubble}:

  Initialize 1 or 2 bubbles in a stratified atmosphere (isothermal or
  isentropic) and allow for the bubbles to have the same or a
  different $\gamma$ from one another / the background atmosphere.  This
  uses the {\tt multigamma} EOS.

  An analogous problem is implemented in \maestro.
    
\item {\tt DustCollapse}:

  A pressureless cloud collapse that is a standard test problem for
  gravity.  An analytic solution that describes the radius of the
  sphere as a function of time is found in Colgate and
  White~\cite{colgwhite}.  This problem is also found in the FLASH
  User's Guide.
  
  
\item {\tt HCBubble}
  
\item {\tt hydrostatic\_adjust}:

  Model a 1-d stellar atmosphere (plane-parallel or
  spherical/self-gravitating) and dump energy in via an analytic heat
  source and watch the atmosphere's hydrostatic state adjust in
  response.  This is the counterpart to the \maestro\ {\tt
    test\_basestate} unit test.

\item {\tt KH}:

  A Kelvin-Helmholtz shear instability problem.
  
\item {\tt oddeven}:

  A grid-aligned shock hitting a very small density perturbation.
  This demonstrates the odd-even decoupling problem discussed in
  \cite{quirk1997}.  This setup serves to test the {\tt
    castro.hybrid\_riemann} option to hydrodynamics.
  
  
\item {\tt reacting\_bubble}:

  A reacting bubble in a stratified white dwarf atmosphere.  This
  problem was featured in the \maestro\ reaction
  paper~\cite{maestro:III}.

  
\item {\tt RT}:

  A single-model Rayleigh-Taylor instability problem.
  
  
\item {\tt RT\_particles}

\item {\tt Sedov}:

  The standard Sedov-Taylor blast wave problem.  This setup was used
  in the first \castro\ paper~\cite{castro_I}.
    
\item {\tt Sod}:
  
  A one-dimensional shock tube setup, including the classic Sod
  problem.  This setup was used in the original \castro\ paper.
  
\item {\tt Sod\_stellar}:

  A version of the Sod shock tube for the general stellar equation of
  state.  This setup and the included inputs files was used
  in~\cite{zingalekatz}.

  
\item {\tt toy\_convect}:

  A simple nova-like convection problem with an external heating
  source.  This problem shows how to use the model parser to
  initialize a 1-d atmosphere on the Castro grid, incorporate a custom
  tagging routine, sponge the fluid above the atmosphere, and write a
  custom diagnostics routine.

  A \maestro\ version of this problem setup also exists.
  
\end{itemize}
