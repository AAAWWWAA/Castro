\section{Code structure}

The code structure in the {\tt Castro/} directory is as follows:
\begin{itemize}
\item {\tt constants/}: contains a file of useful constants in CGS units

\item {\tt ConvertCheckpoint/}: a tool to convert a checkpoint file to
  a larger domain

\item {\tt EOS/}: contains directories for different EOS routines

\item {\tt Exec/}: various problem implementations, including:
  \begin{itemize}
  \item {\tt Sedov/}: run directory for the Sedov problem
  \item {\tt Sod/}: run directory for the Sod problem
  \item {\tt KH/}: run directory for the Kelvin-Helmholz problem
  \end{itemize}

\item {\tt Networks/}: contains directories for different reaction networks

\item {\tt Source/}: source code

\item {\tt UsersGuide/}: you're reading this now!

\item {\tt Util/}: a catch-all for additional things you may need 
\end{itemize}


\section{An Overview of \castro}

\castro\ is built upon the \boxlib\ C++ framework.  This provides
high-level classes for managing an adaptive mesh refinement simulation,
including the core data structures we will deal with.

The \castro\ simulation begins in {\tt main.cpp} where an instance
of the \boxlib\ {\tt Amr} class is created:
\begin{lstlisting}
  Amr* amrptr = new Amr;
\end{lstlisting}
The initialization, including calling a problem's {\tt initdata()}
routine and refining the base grid occurs next through
\begin{lstlisting}
  amrptr->init(strt_time,stop_time);
\end{lstlisting}
nd then comes the main loop over coarse timesteps until the
desired simulation time is reached:
\begin{lstlisting}
  while ( amrptr->okToContinue()                            &&
         (amrptr->levelSteps(0) < max_step || max_step < 0) &&
         (amrptr->cumTime() < stop_time || stop_time < 0.0) )

  {
     //
     // Do a timestep.
     //
     amrptr->coarseTimeStep(stop_time);
  }
\end{lstlisting}
This uses the \boxlib\ machinery to do the necessary subcycling in time,
including synchronization between levels, to advance the level hierarchy
forward in time.  

\subsection{Geometry class}

\subsection{ParmParse class}

\subsection{Castro Data Structures}

\subsubsection{State Data}

\castro\ relies on the class structure defined by \boxlib\ to manage the
data.

In {\tt Castro.H}, the {\tt enum} {\tt StateType} defines the
different descriptors for the state data that \castro\ recognizes.
The main descriptors are:
\begin{itemize}
\item {\tt State\_Type}: the state variables for the hydrodynamics solver.

\item {\tt Rad\_Type}: the radiation quantities (only enabled if {\tt
  RADIATION} is defined).

\item {\tt Gravity\_Type}: the data required for the gravity solve (only
  enabled if {\tt GRAVITY} is defined).

\item {\tt Reactions\_Type}: {\color{red} what is this for?}
\end{itemize}

The state data is registered with \boxlib\ in {\tt Castro\_setup.cpp}.
We access the multifabs that carry the data of interest by interacting
with this \boxlib\ data-structure.  Each state quantity always has both
an old and new timestate and the BoxLib class knows how to interpolate
in both space and time.  We interact with the data by getting pointers
to multifabs.  For instance:
\begin{lstlisting}
MultiFab& S_new = get_new_data(State_Type);
\end{lstlisting}
gets a pointer to the multifab containing the hydrodynamics state data
at the new time (here {\tt State\_Type} is the {\tt enum} defined in 
{\tt Castro.H}).

We iterate over the multifabs using an iterator {\tt MFIter}.  This
iterator knows about the locality of the data---only the boxes on the
processor will be looped over.  An example loop (for the
initialization, from {\tt Castro\_setup.cpp} would be):
\begin{lstlisting}
for (MFIter mfi(S_new); mfi.isValid(); ++mfi)
  {
     const Box& bx      = mfi.validbox();
     const int* lo      = bx.loVect();
     const int* hi      = bx.hiVect();

     if (! orig_domain.contains(bx)) {
        BL_FORT_PROC_CALL(CA_INITDATA,ca_initdata)
          (level, cur_time, lo, hi, ns,
           BL_TO_FORTRAN(S_new[mfi]), dx,
           gridloc.lo(), gridloc.hi());
     }
  }
\end{lstlisting}
here {\tt BL\_TO\_FORTRAN} is a special \boxlib\ macro that converts the
C++ multifab into a Fortran array, and {\tt BL\_FORT\_PROC\_CALL}
is a BoxLib macro that is used to interface with Fortran routines.
\MarginPar{what is the purpose of mfi.isValid()?}

The corresponding Fortran function will look like:
{\color{red} Need to write the Fortran version here}


\subsection{Derived Variables}

\subsection{Error Estimators}


\subsection{Gravity class}


\subsection{Fortran Helper Modules}

There are a number of modules that make data available to the Fortran
side of \castro\ or perform other useful tasks.

\begin{itemize}
\item {\tt prob\_params\_module}:

  This module makes the physical domain lower left and upper right
  coordinates ({\tt problo()} and {\tt probhi}) and problem center
  ({\tt center}) available.

  Here {\tt center} is the center to
  be used for the multipole expansion in gravity, certain diagnostics,
  construction of the radial velocity, etc.  It is not necessarily
  the center of the domain (e.g., for a problem modeling an octant
  of a star, it would be the origin).  This is either set by
  {\tt castro.center} or in the problem's {\tt probinit()}
  subroutine.

  This module also provides the type of physical boundary
  condition in force at each domain edge (and the integer
  keys needed to interpret them).

\item {\tt meth\_params\_module}:

  This module provides the integer keys used to access the state
  arrays for both the conserved variables ({\tt URHO}, {\tt UMX}, $\ldots$)
  and primitive variables ({\tt QRHO}, {\tt QU}, $\ldots$), as well
  as the number of scalar variables.

  It also provides the values of most of the {\tt castro.{\em xxxx}}
  runtime parameters.

\item {\tt model\_parser\_module}:

  This module is built if {\tt USE\_MODELPARSER = TRUE} is set in the
  problem's {\tt GNUmakefile}.  It then provides storage for the an
  initial model and routines to read it in and interpolate onto the
  \castro\ grid.

\item {\tt fundamental\_constants\_module}:

  This provides the CGS values of many physical constants.
  
\end{itemize}

  

\section{Setting Up Your Own Problem}

To define a new problem, we create a new directory under {\tt Exec/},
and place in it a {\tt Prob\_2d.f90} file (or {\tt 1d}/{\tt 3d}, depending on the
dimensionality of the problem), a {\tt probdata.f90} file, the {\tt
  inputs} and {\tt probin} files, and a {\tt Make.package} file that
tells the build system what problem-specific routines exist.  The
simplest way to get started is to copy these files from an existing
problem.  Here we describe how to customize your problem.

The purpose of these files is:
\begin{itemize}
\item {\tt probdata.f90}: this holds the {\tt probdata\_module} Fortran module
  that allocates storage for all the problem-specific runtime parameters that
  are used by the problem (including those that are read from the {\tt probin}
  file.

\item {\tt Prob\_?d.f90}: this holds the main routines to
  initialize the problem and grid and perform problem-specific boundary
  conditions:

  \begin{itemize}
  \item {\tt probinit()}:

    This routine is primarily responsible for reading in the {\tt
      probin} file (by defining the {\tt \&fortin} namelist and
    reading in an initial model (usually through the {\tt
      model\_parser\_module}---see the {\tt toy\_convect} problem
    setup for an example).  The parameters that are initialized
    here are those stored in the {\tt probdata\_module}.

  \item {\tt ca\_initdata()}:

    This routine will initialize the state data for a single grid.
    The inputs to this routine are:
    \begin{itemize}
    \item {\tt level}: the level of refinement of the grid we are filling

    \item {\tt time}: the simulation time

    \item {\tt lo()}, {\tt hi()}: the integer indices of the box's {\em
      valid data region} lower left and upper right corners.  These
      integers refer to a global index space for the level and
      identify where in the computational domain the box lives.

    \item {\tt nscal}: the number of scalar quantities---this is not typically
      used in \castro.

    \item {\tt state\_l1}, {\tt state\_l2}, ({\tt state\_l3}): the
      integer indices of the lower left corner of the box in each
      coordinate direction.  These are for the box as allocated in memory,
      so they include any ghost cells as well as the valid data regions.

    \item {\tt state\_h1}, {\tt state\_h2}, ({\tt state\_h3}): the
      integer indices of the upper right corner of the box in each
      coordinate direction.  These are for the box as allocated in memory,
      so they include any ghost cells as well as the valid data regions.
      
    \item {\tt state()}: the main state array.  This is dimensioned as:
\begin{verbatim}
double precision state(state_l1:state_h1,state_l2:state_h2,NVAR)
\end{verbatim}
    (in 2-d), where {\tt NVAR} comes from the {\tt meth\_params\_module}.

    When accessing this array, we use the index keys provided by
    {\tt meth\_params\_module} (e.g., {\tt URHO}) to refer to specific
    quantities
    
    \item {\tt delta()}: this is an array containing the zone width ($\Delta x$)
      in each coordinate direction: $\mathtt{delta(1)} = \Delta x$,
      $\mathtt{delta(2)} = \Delta y$, $\ldots$.

    \item {\tt xlo()}, {\tt xhi()}: these are the physical coordinates of the
      lower left and upper right corners of the {\em valid region}
      of the box.  These can be used to compute the coordinates of the
      cell-centers of a zone as:
\begin{lstlisting}
  do j = lo(2), hi(2)
     y = xlo(2) + delta(2)*(dble(j-lo(2)) + 0.5d0)
     ...
\end{lstlisting}
     (Note: this method works fine for the problem initialization
     stuff, but for routines that implement tiling, as discussed below,
     {\tt lo} and {\tt xlo} may not refer to the same corner, and instead
     coordinates should be computed using {\tt problo()} from the {\tt
     prob\_params\_module}.)
     
    \end{itemize}
      

  \item the {\tt *fill} routines:

    The following routines handle how \castro\ fills ghostcells {\em
      at physical boundaries} for specific data.

    The idea is that
    these routines are registered in {\tt Castro\_setup.cpp}, and
    called as needed.  By default, they just pass the arguments
    through to {\tt filcc}, which handles all of the generic boundary
    conditions (like reflecting, extrapolation, etc.).  The specific
    `{\tt fill}' routines can then supply the problem-specific
    boundary conditions, which are typically just Dirichlet boundary
    conditions.  The code implementing these specific conditions
    should {\em follow} the {\tt filcc} call.

    \begin{itemize}
    \item {\tt ca\_hypfill}:
      This handles the boundary filling for the hyperbolic system.

    \item {\tt ca\_denfill}: At times, we need to fill just the density
      (always assumed to be the first element in the hyperbolic state)
      instead of the entire state.  When the fill patch routine is called
      with {\tt first\_comp = Density} and {\tt num\_comp = 1}, then we
      use {\tt ca\_denfill} instead of {\tt ca\_hypfill}.

    \item {\tt ca\_grav?fill}: These routines will the ghostcells with the
      gravitational acceleration.  By default, they will just do something
      like a first-order extrapolation.  These are needed for the hydro
      routines to have the gravitational acceleration needed for the 
      source terms to the interface states.

    \item {\tt ca\_reactfill}
    \end{itemize}

    These routines that the following arguments:
    \begin{itemize}
    \item {\tt adv\_l1}, {\tt adv\_l2}, ({\tt adv\_l3}): the indicies of
      the lower left corner of the box holding the data we are working on.
      These indices refer to the entire box, including ghost cells.

    \item {\tt adv\_h1}, {\tt adv\_h2}, ({\tt adv\_h3}): the indicies of
      the upper right corner of the box holding the data we are working on.
      These indices refer to the entire box, including ghost cells.

    \item {\tt adv()}: the array of data whose ghost cells we are filling.
      Depending on the routine, this may have an additional index refering
      to the variable.
    
      This is dimensioned as:
\begin{verbatim}
  double precision adv(adv_l1:adv_h1,adv_l2:adv_h2)                             
\end{verbatim}

    \item {\tt domlo()}, {\tt domhi()}: the integer indices of the lower
      left and upper right corners of the valid region of the {\em entire
        domain}.  These are used to test against to see if we are filling
      physical boundary ghost cells.

      This changes according to refinement level: level-0 will
      range from {\tt 0} to {\tt castro.max\_grid\_size},
      and level-n will range from {\tt 0} to
      $\mathtt{castro.max\_grid\_size} \cdot \prod_n \mathtt{castro.ref\_ratio(n)}$.

    \item {\tt delta()}: is the zone width in each coordinate direction,
      as in {\tt initdata()} above.

    \item {\tt xlo()}: this is the physical coordinate of the lower
      left corner of the box we are filling---including the ghost cells.

      Note: this is different than how {\tt xlo()} was defined in
      {\tt initdata()} above.

    \item {\tt time}: the simulation time

    \item {\tt bc()}: an array that holds the type of boundary conditions
      to enforce at the physical boundaries for {\tt adv}.

      Sometimes it appears of the form {\tt bc(:,:)} and sometimes
      {\tt bc(:,:,:)}---the last index of the latter holds the variable
      index, i.e., density, pressure, species, etc.

      The first index is the coordinate direction and the second index
      is the domain face ({\tt 1} is low, {\tt 2} is hi), so {\tt
        bc(1,1)} is the lower $x$ boundary type, {\tt bc(1,2)} is
      the upper $x$ boundary type, {\tt bc(2,1)} is the lower
      $y$ boundary type, etc.

      To interpret the array values, we test against the quantities
      defined in {\tt bc\_types.fi} included in each subroutine,
      for example, {\tt EXT\_DIR}, {\tt FOEXTRAP}, $\ldots$.  The
      meaning of these are explained below.
      
    \end{itemize}
    
  \end{itemize}

\end{itemize}


\section{Boundaries}
\subsection{Boundaries Between Grids}
Boundaries between grids are of two types. The first we call
``fine-fine'', which is two grids at the same level.  Filling ghost
cells at the same level is also part of the fillpatch operation---it's
just a straight copy from ``valid regions'' to ghost cells. The second
type is "coarse-fine", which needs interpolation from the coarse grid
to fill the fine grid ghost cells.  This also happens as part of the
FillPatch operation, which is why arrays aren't just arrays, they're
``State Data'', which means that the data knows how to interpolate
itself (in an anthropomorphical sense).  The type of interpolation to
use is defined in {\tt Castro\_setup.cpp} as well---search for
{\tt cell\_cons\_interp}, for example---that's ``cell conservative
interpolation'', i.e., the data is cell-based (as opposed to node-based
or edge-based) and the interpolation is such that the average of the
fine values created is equal to the coarse value from which they came.
(This wouldn't be the case with straight linear interpolation, for
example.)

A {\tt FillPatchIterator} is used to loop over the grids and fill
ghostcells.  One should never assume that ghostcells are valid.  A key
thing to keep in mind about the {\tt FillPatchIterator} is that you
operate on a copy of the data---the data is disconnected from the
original source.  If you want to update the data in the source,
you need to explicitly copy it back.  Also note: {\tt FillPatchIterator}
takes a multifab, but this is not filled---this is only used to
get the grid layout.  \MarginPar{did I say that right?} 

{\color{red}simple example}


\subsection{Physical Boundaries}

The following boundary conditions have already been implemented in
\castro\ See {\tt BoxLib/Src/C\_AMRLib/amrlib/BC\_TYPES.H} for some
more details.
\begin{itemize}
\item {\it Outflow}:
  \begin{itemize}
    \item velocity: {\tt FOEXTRAP}
    \item temperature: {\tt FOEXTRAP}
    \item scalars: {\tt FOEXTRAP}
  \end{itemize}
  
\item {\it No Slip Wall with Adiabatic Temp}:
  \begin{itemize}
  \item velocity: {\tt EXT\_DIR}, $u=v=0$
  \item temperature: {\tt REFLECT\_EVEN}, $dT/dt=0$
  \item scalars: {\tt HOEXTRAP}
  \end{itemize}

\item {\it No Slip Wall with Fixed Temp}:
  \begin{itemize}
  \item velocity: {\tt EXT\_DIR}, $u=v=0$
  \item temperature: {\tt EXT\_DIR}
  \item scalars: {\tt HOEXTRAP}
  \end{itemize}
    
\item {\it Slip Wall with Adiabatic Temp}:
  \begin{itemize}
  \item velocity: {\tt EXT\_DIR}, $u_n=0$; {\tt HOEXTRAP}, $u_t$
  \item temperature: {\tt REFLECT\_EVEN}, $dT/dn=0$
  \item scalars: {\tt HOEXTRAP}
  \end{itemize}
  
\item {\it Slip Wall with Fixed Temp}:
  \begin{itemize}
  \item velocity: {\tt EXT\_DIR}, $u_n=0$
  \item temperature: {\tt EXT\_DIR}
  \item scalars: {\tt HOEXTRAP}
  \end{itemize}

\end{itemize}

Here's definitions of some of the funny-sounding ``all-caps''
words from above:
\begin{itemize}
\item {\tt INT\_DIR}: data taken from other grids or interpolated

\item {\tt EXT\_DIR}: data specified on EDGE (FACE) of bndry

\item {\tt HOEXTRAP}: higher order extrapolation to EDGE of bndry

\item {\tt FOEXTRAP}: first order extrapolation from last cell in interior

\item {\tt REFLECT\_EVEN}: $F(-n) = F(n)$ true reflection from interior cells

\item {\tt REFLECT\_ODD}: $F(-n) = -F(n)$ true reflection from interior cells
\end{itemize}


\subsection{Filling the boundaries}

Basically, boundary conditions are imposed on ``state variables'' every
time that they're ``fillpatched'', as part of the fillpatch operation.

For example, the loop that calls {\tt CA\_UMDRV} (all the integration stuff) starts with
\begin{lstlisting}
   for (FillPatchIterator fpi(*this, S_new, NUM_GROW,
                              time, State_Type, strtComp, NUM_STATE);
         fpi.isValid(); ++fpi)
   {
     FArrayBox &state = fpi();

     // work on the state FAB
   }           
\end{lstlisting}
Here the {\tt FillPatchIterator} is the thing that distributes the
grids over processors and makes parallel ``just work''. This fills the
single patch ``{\tt fpi}'' , which has {\tt NUM\_GROW} ghost cells,
with data of type ``{\tt State\_Type}'' at time ``{\tt time}'',
starting with component {\tt strtComp} and including a total of {\tt
  NUM\_STATE} components.

The way that you tell the code what kind of physical boundary
condition to use is given in {\tt Castro\_setup.cpp}. At the top we
define arrays such as ``{\tt scalar\_bc}'', ``{\tt norm\_vel\_bc}'',
etc, which say which kind of bc to use on which kind of physical
boundary.  Boundary conditions are set in functions like ``{\tt
  set\_scalar\_bc}'', which uses the {\tt scalar\_bc} pre-defined
arrays.

If you want to specify a value at a function (like at an inflow
boundary), there are routines in {\tt Prob\_1d.f90}, for example, which do
that. Which routine is called for which variable is again defined in
{\tt Castro\_setup.cpp}.

\section{Parallel I/O}

Both checkpoint files and plotfiles are really directories containing
subdirectories: one subdirectory for each level of the AMR hierarchy.
The fundamental data structure we read/write to disk is a MultiFab,
which is made up of multiple FAB's, one FAB per grid.  Multiple
MultiFabs may be written to each directory in a checkpoint file.
MultiFabs of course are shared across CPUs; a single MultiFab may be
shared across thousands of CPUs.  Each CPU writes the part of the
MultiFab that it owns to disk, but they don't each write to their own
distinct file.  Instead each MultiFab is written to a runtime
configurable number of files N (N can be set in the inputs file as the
parameter {\tt amr.checkpoint\_nfiles} and {\tt amr.plot\_nfiles}; the
default is 64).  That is to say, each MultiFab is written to disk
across at most N files, plus a small amount of data that gets written
to a header file describing how the file is laid out in those N files.

What happens is $N$ CPUs each opens a unique one of the $N$ files into
which the MultiFab is being written, seeks to the end, and writes
their data.  The other CPUs are waiting at a barrier for those $N$
writing CPUs to finish.  This repeats for another $N$ CPUs until all the
data in the MultiFab is written to disk.  All CPUs then pass some data
to CPU {\tt 0} which writes a header file describing how the MultiFab is
laid out on disk.

We also read MultiFabs from disk in a ``chunky'' manner, opening only $N$
files for reading at a time.  The number $N$, when the MultiFabs were
written, does not have to match the number $N$ when the MultiFabs are
being read from disk.  Nor does the number of CPUs running while
reading in the MultiFab need to match the number of CPUs running when
the MultiFab was written to disk.

Think of the number $N$ as the number of independent I/O pathways in
your underlying parallel filesystem.  Of course a ``real'' parallel
filesytem should be able to handle any reasonable value of $N$.  The
value {\tt -1} forces $N$ to the number of CPUs on which you're
running, which means that each CPU writes to a unique file, which can
create a very large number of files, which can lead to inode issues.


