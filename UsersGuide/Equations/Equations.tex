\section{Conservation Forms}
We begin with the fully compressible equations for the conserved state vector, 
$\Ub = (\rho, \rho \ub, \rho E, \rho A_k, \rho X_k, \rho Y_k):$
\begin{eqnarray}
\frac{\partial \rho}{\partial t} &=& - \nabla \cdot (\rho \ub) + S_{{\rm ext},\rho}, \\
\frac{\partial (\rho \ub)}{\partial t} &=& - \nabla \cdot (\rho \ub \ub) - \nabla p +\rho \gb + \Sb_{{\rm ext},\rho\ub}, \\
\frac{\partial (\rho E)}{\partial t} &=& - \nabla \cdot (\rho \ub E + p \ub) + \rho \ub \cdot \gb - \sum_k {\rho q_k \dot\omega_k} + \nabla\cdot\kappa\nabla T + S_{{\rm ext},\rho E}, \\
\frac{\partial (\rho A_k)}{\partial t} &=& - \nabla \cdot (\rho \ub A_k) + S_{{\rm ext},\rho A_k}, \\
\frac{\partial (\rho X_k)}{\partial t} &=& - \nabla \cdot (\rho \ub X_k) + \rho \dot\omega_k + S_{{\rm ext},\rho X_k}, \\
\frac{\partial (\rho Y_k)}{\partial t} &=& - \nabla \cdot (\rho \ub Y_k) + S_{{\rm ext},\rho Y_k}.\label{eq:compressible-equations}
\end{eqnarray}
Here $\rho, \ub, T, p$, and $\kappa$ are the density, velocity,
temperature, pressure, and thermal conductivity, respectively, and 
$E = e + \ub \cdot \ub / 2$ is the total energy with $e$ representing the internal
energy.  In addition, $X_k$ is the abundance of the $k^{\rm th}$ isotope,
with associated production rate, $\dot\omega_k$, and
energy release, $q_k$.  Here $\gb$ is the gravitational vector,
and $S_{{\rm ext},\rho}, \Sb_{{\rm ext}\rho\ub}$, etc.,
are user-specified source terms.  $A_k$ is an advected quantity, i.e., 
a tracer.  We also carry around auxiliary variables, $Y_k$, which have a 
user-defined evolution equation, but by default are treated as advected
quantities.

In the code we also carry around $T$ and $\rho e$ in the conservative state vector
even though they are derived from the other conserved quantities.
The ordering of the elements within $\Ub$ is defined (in 3D) by integer
variables into the array---see Table~\ref{table:consints}
\begin{table}[t]
  \centering
\begin{tabular}{llp{3.5in}}
  \hline
      {\bf variable} & {\bf quantity} & {\bf note} \\
  \hline
{\tt URHO} & $\rho$ \\
{\tt UMX} & $\rho u$ \\
{\tt UMY} & $\rho v$ \\
{\tt UMZ} & $\rho w$ \\
{\tt UEDEN} & $\rho E$ \\
{\tt UEINT} & $\rho e$ & this is computed from the other quantities using
             $\rho e = \rho E - \rho \ub \cdot \ub / 2$ \\
{\tt UTEMP} & $T$  & this is computed from the other quantities using the EOS \\
{\tt UFA} & $\rho A_1$ & the first advected quantity \\
{\tt UFS} & $\rho X_1$ & the first species \\
{\tt UFX} & $\rho Y_1$ & the first auxiliary variable \\
\hline
\end{tabular}
\caption{\label{table:consints} The integer variables to index the conservative state array}
\end{table}

Some notes:
\begin{itemize}
\item There are {\tt nadv} advected quantities, which range from {\tt UFA: UFA+nadv-1}.
The advected quantities have no effect at all on the rest of the solution
but can be useful as tracer quantities.

\item There are {\tt nspec} species (defined in the {\tt network}
  directory), which range from {\tt UFS: UFS+nspec-1}.

\item There are {\it naux} auxiliary variables, from {\tt
  UFX:UFX+naux-1} The auxiliary variables are passed into the equation
  of state routines along with the species; An example of an auxiliary
  variable is the electron fraction, $Y_e$, in core collapse simulations.
\end{itemize}


\section{Primitive Forms}
\castro\ uses the primitive form of the fluid equations, defined in terms of
the state $\Qb = (\rho, \ub, p, \rho e, A_k, X_k, Y_k)$, to construct the
interface states that are input to the Riemann problem.

The primitive variable equations for density, velocity, and pressure are:
\begin{eqnarray}
  \frac{\partial\rho}{\partial t} &=& -\ub\cdot\nabla\rho - \rho\nabla\cdot\ub + S_{{\rm ext},\rho} \\
%
  \frac{\partial\ub}{\partial t} &=& -\ub\cdot\nabla\ub - \frac{1}{\rho}\nabla p + \gb + 
\frac{1}{\rho} (\Sb_{{\rm ext},\rho\ub} - \ub \; S_{{\rm ext},\rho}) \\
\frac{\partial p}{\partial t} &=& -\ub\cdot\nabla p - \rho c^2\nabla\cdot\ub +
\left(\frac{\partial p}{\partial \rho}\right)_{e,X}S_{{\rm ext},\rho}\nonumber\\
&&+\  \frac{1}{\rho}\sum_k\left(\frac{\partial p}{\partial X_k}\right)_{\rho,e,X_j,j\neq k}\left(\rho\dot\omega_k + S_{{\rm ext},\rho X_k} - X_kS_{{\rm ext},\rho}\right)\nonumber\\
&& +\  \frac{1}{\rho}\left(\frac{\partial p}{\partial e}\right)_{\rho,X}\left[-eS_{{\rm ext},\rho} - \sum_k\rho q_k\dot\omega_k + \nabla\cdot\kappa\nabla T \right.\nonumber\\
&& \quad\qquad\qquad\qquad+\ S_{{\rm ext},\rho E} - \ub\cdot\left(\Sb_{{\rm ext},\rho\ub} - \frac{\ub}{2}S_{{\rm ext},\rho}\right)\Biggr] 
\end{eqnarray}
We augment these with an internal energy equation:
\begin{eqnarray}
\frac{\partial(\rho e)}{\partial t} &=& - \ub\cdot\nabla(\rho e) - (\rho e+p)\nabla\cdot\ub - \sum_k \rho q_k\dot\omega_k 
                                        + \nabla\cdot\kappa\nabla T + S_{{\rm ext},\rho E} \nonumber\\
&& -\  \ub\cdot\left(\Sb_{{\rm ext},\rho\ub}-\frac{1}{2}S_{{\rm ext},\rho}\ub\right), 
\end{eqnarray}
Technically, this is redundant, but for a general equation of state, carrying around an additional
the thermodynamic quantity allows us to avoid equation of state calls.

The advected quantities appear as:
\begin{eqnarray}
\frac{\partial A_k}{\partial t} &=& -\ub\cdot\nabla A_k + \frac{1}{\rho}
                                     ( S_{{\rm ext},\rho A_k} - A_k S_{{\rm ext},\rho} ), \\
\frac{\partial X_k}{\partial t} &=& -\ub\cdot\nabla X_k + \dot\omega_k + \frac{1}{\rho}
                                     ( S_{{\rm ext},\rho X_k}  - X_k S_{{\rm ext},\rho} ), \\
\frac{\partial Y_k}{\partial t} &=& -\ub\cdot\nabla Y_k + \frac{1}{\rho} 
                                     ( S_{{\rm ext},\rho Y_k}  - Y_k S_{{\rm ext},\rho} ).
\end{eqnarray}

In the code we also carry around $T$ in the primitive state vector.
All of the primitive variables are derived from the conservative state
vector, as described in Section \ref{Sec:Compute Primitive Variables}.
When accessing the primitive variable state vector, the integer variable
keys for the different quantities are listed in Table~\ref{table:primlist}.

\begin{table}[t]
  \centering
\begin{tabular}{llp{3.5in}}
  \hline
      {\bf variable} & {\bf quantity} & {\bf note} \\
  \hline
{\tt  QRHO} & $\rho$ \\
{\tt  QU} & $u$ \\
{\tt  QV} & $v$ \\
{\tt  QW} & $w$ \\
{\tt  QPRES} & $p$ \\
{\tt  QREINT} & $\rho e$ \\
{\tt  QTEMP} & $T$ \\
{\tt  QFA} & $A_1$ & the first advected quantity \\
{\tt  QFS} & $X_1$ & the first species \\
{\tt  QFX} & $Y_1$ & the first auxiliary variable \\
 \hline
\end{tabular}
\caption{\label{table:primlist} The integer variable keys for accessing the
  primitive state vector}
\end{table}

The full primitive variable form (without the advected or auxiliary quantities) is
\begin{equation}
\frac{\partial\Qb}{\partial t} + \sum_d \Ab_d\frac{\partial\Qb}{\partial x_d} = \Sb_{\Qb}.
\end{equation}
For example, in 2D:
\begin{equation}
\left(\begin{array}{c}
\rho \\
u \\
v \\
p \\
\rho e \\
X_k
\end{array}\right)_t
+
\left(\begin{array}{cccccc}
u & \rho & 0 & 0 & 0 & 0 \\
0 & u & 0 & \frac{1}{\rho} & 0 & 0 \\
0 & 0 & u & 0 & 0 & 0 \\
0 & \rho c^2 & 0 & u & 0 & 0 \\
0 & \rho e + p & 0 & 0 & u & 0 \\
0 & 0 & 0 & 0 & 0 & u
\end{array}\right)
\left(\begin{array}{c}
\rho \\
u \\
v \\
p \\
\rho e \\
X_k
\end{array}\right)_x
+
\left(\begin{array}{cccccc}
v & 0 & \rho & 0 & 0 & 0 \\
0 & v & 0 & 0 & 0 & 0 \\
0 & 0 & v & \frac{1}{\rho} & 0 & 0 \\
0 & 0 & \rho c^2 & v & 0 & 0 \\
0 & 0 & \rho e + p & 0 & v & 0 \\
0 & 0 & 0 & 0 & 0 & v
\end{array}\right)
\left(\begin{array}{c}
\rho \\
u \\
v \\
p \\
\rho e \\
X_k
\end{array}\right)_y
=
\Sb_\Qb
\end{equation}
The eigenvalues are:
\begin{equation}
{\bf \Lambda}(\Ab_x) = \{u-c,u,u,u,u,u+c\}, \qquad {\bf \Lambda}(\Ab_y) = \{v-c,v,v,v,v,v+c\} .
\end{equation}
The right column eigenvectors are:
\begin{equation}
\Rb(\Ab_x) =
\left(\begin{array}{cccccc}
1 & 1 & 0 & 0 & 0 & 1 \\
-\frac{c}{\rho} & 0 & 0 & 0 & 0 & \frac{c}{\rho} \\
0 & 0 & 1 & 0 & 0 & 0 \\
c^2 & 0 & 0 & 0 & 0 & c^2 \\
h & 0 & 0 & 1 & 0 & h \\
0 & 0 & 0 & 0 & 1 & 0 \\
\end{array}\right),
\qquad
\Rb(\Ab_y) =
\left(\begin{array}{cccccc}
1 & 1 & 0 & 0 & 0 & 1 \\
0 & 0 & 1 & 0 & 0 & 0 \\
-\frac{c}{\rho} & 0 & 0 & 0 & 0 & \frac{c}{\rho} \\
c^2 & 0 & 0 & 0 & 0 & c^2 \\
h & 0 & 0 & 1 & 0 & h \\
0 & 0 & 0 & 0 & 1 & 0 \\
\end{array}\right).
\end{equation}
The left row eigenvectors, normalized so that $\Rb_d\cdot\Lb_d = \Ib$ are:
\begin{equation}
\Lb_x =
\left(\begin{array}{cccccc}
0 & -\frac{\rho}{2c} & 0 & \frac{1}{2c^2} & 0 & 0 \\
1 & 0 & 0 & -\frac{1}{c^2} & 0 & 0 \\
0 & 0 & 1 & 0 & 0 & 0 \\
0 & 0 & 0 & -\frac{h}{c^2} & 1 & 0 \\
0 & 0 & 0 & 0 & 0 & 1 \\
0 & \frac{\rho}{2c} & 0 & \frac{1}{2c^2} & 0 & 0
\end{array}\right),
\qquad
\Lb_y =
\left(\begin{array}{cccccc}
0 & 0 & -\frac{\rho}{2c} & \frac{1}{2c^2} & 0 & 0 \\
1 & 0 & 0 & -\frac{1}{c^2} & 0 & 0 \\
0 & 1 & 0 & 0 & 0 & 0 \\
0 & 0 & 0 & -\frac{h}{c^2} & 1 & 0 \\
0 & 0 & 0 & 0 & 0 & 1 \\
0 & 0 & \frac{\rho}{2c} & \frac{1}{2c^2} & 0 & 0
\end{array}\right).
\end{equation}

\section{Runtime Parameters}

A number of runtime parameters affect the hydrodynamics behavior.
\begin{itemize}
\item {\tt castro.do\_hydro}: time-advance the fluid dynamical
  equations (0 or 1; must be set)

\item {\tt castro.add\_ext\_src}: include additional user-specified
  source term (0 or 1; default 0)
  
\item {\tt castro.do\_sponge}: call a user-supplied sponging routine
  after the solution update (0 or 1; default: 0)

  The purpose of the sponge is to damp velocities outside of a star, to
  prevent them from dominating the timestep constraint.
  
\item {\tt castro.normalize\_species}: enforce that $\sum_i X_i = 1$
  (0 or 1; default: 0)
  
\item {\tt castro.fix\_mass\_flux}: enforce constant mass flux at
  domain boundary (0 or 1; default: 1)
  
\item {\tt castro.allow\_negative\_energy}: is internal energy allowed to be
  negative? (0 or 1; default: 1)
  
\item {\tt castro.spherical\_star}: this is used to set the boundary
  conditions by assuming the star is spherically symmetric in
  the outer regions (0 or 1; default: 0)

  When used, \castro\ averages the values at a given radius over the
  cells that are inside the domain to define a radial function.  This
  function is then used to set the values outside the domain in
  implementing the boundary conditions.
  
\item {\tt castro.show\_center\_of\_mass}: (0 or 1; default: 0)
\end{itemize}

Several floors are imposed on the thermodynamic quantities to prevet unphysical
behavior:
\begin{itemize}
\item {\tt castro.small\_dens}: (Real; default: -1.e20)
\item {\tt castro.small\_temp}: (Real; default: -1.e20)
\item {\tt castro.small\_pres}: (Real; default: -1.e20)
\end{itemize}

For the reconstruction of the interface states, the following apply:
\begin{itemize}
\item {\tt castro.ppm\_type}: use piecewise linear vs PPM algorithm
  (0, 1, 2; default: 1)

  Values of 1 and 2 are both piecewise parabolic reconstruction, with
  2 using updated limiters that better preserve extrema.

\item {\tt castro.ppm\_reference} uses the integral under the parabola for
  the region corresponding to the fastest moving wave to the interface
  for the reference state used by the characteristic projection in
  ppm (1) or the cell-centered value (0).  (Default: 1)

  The original \castro\ paper~\cite{castro_I} used the cell-centered
  value ({\tt ppm\_reference = 0}), but the method in the PPM paper~\cite{ppm}
  does the method of {\tt ppm\_refrence = 1}.
    
\item {\tt castro.ppm\_trace\_grav} reconstructs the gravitational
  acceleration as a parabola and then constructs the integrals under
  this parabola for the three characteristic waves and then puts the
  gravitational source term into the characteristic projection when
  doing the interface states (0 or 1; default: 0)

\item {\tt castro.ppm\_temp\_fix} does various attempts to use the
  temperature in the reconstruction of the interface states.  This
  is experimental.

\item {\tt castro.ppm\_tau\_in\_tracing} uses $\tau = 1/\rho$ instead of
  $\rho$ in the characteristic projection (0 or 1; default 0)

\item {\tt castro.ppm\_predict\_gammae} reconstructs $\gamma_e = p/(\rho e) + 1$
  to the interfaces and does the necessary transverse terms to aid in
  the conversion between the conserved and primitive interface states
  in the transverse flux routines (0 or 1; default 0)

\item {\tt castro.ppm\_reference\_edge\_limit} uses the limit of the
  parabolic interpolant as the reference state instead of the
  cell-center value for the case when the wave is not moving toward
  the interface (0 or 1; default 0)

\item {\tt castro.ppm\_reference\_eigenvectors} uses the reference states in
  the evaluation of the eigenvectors for the characteristic projection
  (0 or 1; default 0)
  
\item {\tt castro.use\_flattening} turns on/off the flattening of parabola
  near shocks (0 or 1; default 1)

\item {\tt castro.ppm\_flatten\_before\_integrals} does the flattening
  procedure on the parabolic coefficients before they are integrated
  to the interface states instead of the default method of flattening
  on the integrals themselves.

\end{itemize}


For the construction of the fluxes in the Riemann solver, the following
parameters apply:
\begin{itemize}
\item {\tt castro.use\_colglaz} uses the Colella \& Glaz Riemann solver~\cite{colglaz} to
  solve for the `star' state (0 or 1; default 0)

  The default is to use the solver based on an unpublished Colella,
  Glaz, \& Ferguson manuscript, as described in the original
  \castro\ paper~\cite{castro_I}.

  The Colella \& Glaz solver is iterative, and two runtime parameters are used
  to control its behavior:
  \begin{itemize}
  \item {\tt castro.cg\_maxiter}: number of iterations for CG algorithm
    (Integer; default: 12)
    
  \item {\tt castro.cg\_tol}: tolerance for CG solver when solving
     for the ``star'' state (Real; default: 1.0e-5)
  \end{itemize}

\item {castro.hybrid\_riemann}: switch to an HLL Riemann solver when we are
  in a zone with a shock (0 or 1; default 0)

  This eliminates an odd-even decoupling issue (see the {\tt oddeven}
  problem.
  
\end{itemize}



\section{Advection Step}\label{Sec:Advection Step}
There are four major steps in the advective update, detailed below.
\subsection{Compute Primitive Variables}\label{Sec:Compute Primitive Variables}
We compute the primtive variables from the conserved variables.
\begin{itemize}
\item $\rho, \rho e$ - directly copy these from the conserved state
  vector
\item $\ub, A_k, X_k, Y_k$ - copy these from the conserved state
  vector, dividing by $\rho$
\item $p,T$ - use the EOS.  First, if {\bf
  castro.allow\_negative\_energy} = 0 (it defaults to 1) and $e < 0$,
  we do the following:
\begin{enumerate}
\item Use the EOS to set $e = e(\rho,T_{\rm small},X_k)$.
\item If $e < 0$, abort the program with an error message.
\end{enumerate}
Now, use the EOS to compute $p,T = p,T(\rho,e,X_k)$.
\end{itemize}
We also compute the flattening coefficient, $\chi\in[0,1]$, used in
the edge state prediction to further limit slopes near strong shocks.
We use the same flattening procedure described in the the FLASH paper.
A flattening coefficient of 1 indicates that no additional limiting
takes place; a flattening coefficient of 0 means we effectively drop
order to a first-order Godunov scheme (this convention is opposite of
that used in the FLASH paper).  For each cell, we compute the
flattening coefficient for each spatial direction, and choose the
minimum value over all directions.  As an example, to compute the
flattening for the x-direction, here are the steps:
\begin{enumerate}
\item Define $\zeta$
\begin{equation}
\zeta_i = \frac{p_{i+1}-p_{i-1}}{\max\left(p_{\rm small},|p_{i+2}-p_{i-2}|\right)}.
\end{equation}
\item Define $\tilde\chi$
\begin{equation}
\tilde\chi_i = \min\left\{1,\max[0,a(\zeta_i - b)]\right\},
\end{equation}
where $a=10$ and $b=0.75$ are tunable parameters.  We are essentially
setting $\tilde\chi_i=a(\zeta_i-b)$, and then constraining
$\tilde\chi_i$ to lie in the range $[0,1]$.  Then, if either
$u_{i+1}-u_{i-1}<0$ or
\begin{equation}
\frac{p_{i+1}-p_{i-1}}{\min(p_{i+1},p_{i-1})} \le c,
\end{equation}
where $c=1/3$ is a tunable parameter, then set $\tilde\chi_i=0$.
\item Define $\chi$
\begin{equation}
\chi_i =
\begin{cases}
1 - \max(\tilde\chi_i,\tilde\chi_{i-1}) & p_{i+1}-p_{i-1} > 0 \\
1 - \max(\tilde\chi_i,\tilde\chi_{i+1}) & \text{otherwise}
\end{cases}.
\end{equation}
\end{enumerate}
\subsection{Edge State Prediction}
We wish to compute a left and right state of primitive variables at
each edge to be used as inputs to the Riemann problem.  We use a
version of the Colella and Sekora 2009 PPM algorithm, which has been
further modified to eliminate sensitivity due to roundoff error
(modifications via personal communication with Colella).  Note that
\castro\ also has options for the original PPM algorithm of Colella
and Woodward 1984, and piecewise-linear algorithm described in
Saltzman 1994.  We also use characteristic tracing with corner
coupling in 3D, as described in Miller and Colella 2002.  We give full
details of the PPM algorithm, as it has not appeared before in the
literature, and summarize the developments from Miller and Colella
2002.

The PPM algorithm is used to compute time-centered edge states by
extrapolating the base-time data in space and time.  The edge states
are dual-valued, i.e., at each face, there is a left state and a right
state estimate.  The spatial extrapolation is one-dimensional, i.e.,
transverse derivatives are ignored.  We also use a flattening
procedure to further limit the edge state values.  The Miller and
Colella 2002 algorithm, which we describe later, incorporates the
transverse terms, and also describes the modifications required for
equations with additional characteristics besides the fluid velocity.
There are four steps to compute these dual-valued edge states (here,
we use $s$ to denote an arbitrary scalar from $\Qb$, and we write the
equations in 1D, for simplicity):
\begin{itemize}
\item {\bf Step 1}: Compute $s_{i,+}$ and $s_{i,-}$, which are spatial
  interpolations of $s$ to the hi and lo side of the face with special
  limiters, respectively.  Begin by interpolating $s$ to edges using a
  4th-order interpolation in space:
\begin{equation}
s_{i+\myhalf} = \frac{7}{12}\left(s_{i+1}+s_i\right) - \frac{1}{12}\left(s_{i+2}+s_{i-1}\right).
\end{equation}
Then, if $(s_{i+\myhalf}-s_i)(s_{i+1}-s_{i+\myhalf}) < 0$, we limit
$s_{i+\myhalf}$ a nonlinear combination of approximations to the
second derivative.  The steps are as follows:
\begin{enumerate}
\item Define:
\begin{eqnarray}
(D^2s)_{i+\myhalf} &=& 3\left(s_{i}-2s_{i+\myhalf}+s_{i+1}\right) \\
(D^2s)_{i+\myhalf,L} &=& s_{i-1}-2s_{i}+s_{i+1} \\
(D^2s)_{i+\myhalf,R} &=& s_{i}-2s_{i+1}+s_{i+2}
\end{eqnarray}
\item Define
\begin{equation}
s = \text{sign}\left[(D^2s)_{i+\myhalf}\right],
\end{equation}
\begin{equation}
(D^2s)_{i+\myhalf,\text{lim}} = s\max\left\{\min\left[Cs\left|(D^2s)_{i+\myhalf,L}\right|,Cs\left|(D^2s)_{i+\myhalf,R}\right|,s\left|(D^2s)_{i+\myhalf}\right|\right],0\right\},
\end{equation}
where $C=1.25$ as used in Colella and Sekora 2009.  The limited value
of $s_{i+\myhalf}$ is
\begin{equation}
s_{i+\myhalf} = \frac{1}{2}\left(s_{i}+s_{i+1}\right) - \frac{1}{6}(D^2s)_{i+\myhalf,\text{lim}}.
\end{equation}
\end{enumerate}
Now we implement an updated implementation of the Colella and Sekora
2009 algorithm which eliminates sensitivity to roundoff.  First we
need to detect whether a particular cell corresponds to an
``extremum''.  There are two tests.
\begin{itemize}
\item For the first test, define
\begin{equation}
\alpha_{i,\pm} = s_{i\pm\myhalf} - s_i.
\end{equation}
If $\alpha_{i,+}\alpha_{i,-} \ge 0$, then we are at an extremum.
\item We only apply the second test if either $|\alpha_{i,\pm}| >
  2|\alpha_{i,\mp}|$.  If so, we define:
\begin{eqnarray}
(Ds)_{i,{\rm face},-} &=& s_{i-\myhalf} - s_{i-\sfrac{3}{2}} \\
(Ds)_{i,{\rm face},+} &=& s_{i+\sfrac{3}{2}} - s_{i-\myhalf}
\end{eqnarray}
\begin{equation}
(Ds)_{i,{\rm face,min}} = \min\left[\left|(Ds)_{i,{\rm face},-}\right|,\left|(Ds)_{i,{\rm face},+}\right|\right].
\end{equation}
\begin{eqnarray}
(Ds)_{i,{\rm cc},-} &=& s_{i} - s_{i-1} \\
(Ds)_{i,{\rm cc},+} &=& s_{i+1} - s_{i}
\end{eqnarray}
\begin{equation}
(Ds)_{i,{\rm cc,min}} = \min\left[\left|(Ds)_{i,{\rm cc},-}\right|,\left|(Ds)_{i,{\rm cc},+}\right|\right].
\end{equation}
If $(Ds)_{i,{\rm face,min}} \ge (Ds)_{i,{\rm cc,min}}$, set 
$(Ds)_{i,\pm} = (Ds)_{i,{\rm face},\pm}$.  Otherwise, set 
$(Ds)_{i,\pm} = (Ds)_{i,{\rm cc},\pm}$.  Finally, we are at an extreumum if
$(Ds)_{i,+}(Ds)_{i,-} \le 0$.
\end{itemize}
Thus concludes the extremum tests.  The remaining limiters depend on
whether we are at an extremum.
\begin{itemize}
\item If we are at an extremum, we modify $\alpha_{i,\pm}$.  First, we
  define
\begin{eqnarray}
(D^2s)_{i} &=& 6(\alpha_{i,+}+\alpha_{i,-}) \\
(D^2s)_{i,L} &=& s_{i-2}-2s_{i-1}+s_{i} \\
(D^2s)_{i,R} &=& s_{i}-2s_{i+1}+s_{i+2} \\
(D^2s)_{i,C} &=& s_{i-1}-2s_{i}+s_{i+1}
\end{eqnarray}
Then, define
\begin{equation}
s = \text{sign}\left[(D^2s)_{i}\right],
\end{equation}
\begin{equation}
(D^2s)_{i,\text{lim}} = \max\left\{\min\left[s(D^2s)_{i},Cs\left|(D^2s)_{i,L}\right|,Cs\left|(D^2s)_{i,R}\right|,Cs\left|(D^2s)_{i,C}\right|\right],0\right\}.
\end{equation}
Then,
\begin{equation}
\alpha_{i,\pm} = \frac{\alpha_{i,\pm}(D^2s)_{i,\text{lim}}}{\max\left[(D^2s)_{i},1\times 10^{-10}\right]}
\end{equation}
\item If we are not at an extremum and $|\alpha_{i,\pm}| >
  2|\alpha_{i,\mp}|$, then define
\begin{equation}
s = \text{sign}(\alpha_{i,\mp})
\end{equation}
\begin{equation}
\delta\mathcal{I}_{\text{ext}} = \frac{-\alpha_{i,\pm}^2}{4\left(\alpha_{j,+}+\alpha_{j,-}\right)},
\end{equation}
\begin{equation}
\delta s = s_{i\mp 1} - s_i,
\end{equation}
If $s\delta\mathcal{I}_{\text{ext}} \ge s\delta s$, then we perform
the following test.  If $s\delta s - \alpha_{i,\mp} \ge 1\times
10^{-10}$, then
\begin{equation}
\alpha_{i,\pm} =  -2\delta s - 2s\left[(\delta s)^2 - \delta s \alpha_{i,\mp}\right]^{\myhalf}
\end{equation}
otherwise,
\begin{equation}
\alpha_{i,\pm} =  -2\alpha_{i,\mp}
\end{equation}
\end{itemize}
Finally, $s_{i,\pm} = s_i + \alpha_{i,\pm}$.
\item {\bf Step 2}: Construct a quadratic profile using $s_{i,-},s_i$,
  and $s_{i,+}$.
\begin{equation}
s_i^I(x) = s_{i,-} + \xi\left[s_{i,+} - s_{i,-} + s_{6,i}(1-\xi)\right],\label{Quadratic Interp}
\end{equation}
\begin{equation}
s_6 = 6s_{i} - 3\left(s_{i,-}+s_{i,+}\right),
\end{equation}
\begin{equation}
\xi = \frac{x - ih}{h}, ~ 0 \le \xi \le 1.
\end{equation}
\item {\bf Step 3:} Integrate quadratic profiles.  We are essentially
  computing the average value swept out by the quadratic profile
  across the face assuming the profile is moving at a speed
  $\lambda_k$.\\ \\ Define the following integrals, where $\sigma_k =
  |\lambda_k|\Delta t/h$:
\begin{eqnarray}
\mathcal{I}_{i,+}(\sigma_k) &=& \frac{1}{\sigma_k h}\int_{(i+\myhalf)h-\sigma_k h}^{(i+\myhalf)h}s_i^I(x)dx \\
\mathcal{I}_{i,-}(\sigma_k) &=& \frac{1}{\sigma_k h}\int_{(i-\myhalf)h}^{(i-\myhalf)h+\sigma_k h}s_i^I(x)dx
\end{eqnarray}
Plugging in (\ref{Quadratic Interp}) gives:
\begin{eqnarray}
\mathcal{I}_{i,+}(\sigma_k) &=& s_{i,+} - \frac{\sigma_k}{2}\left[s_{i,+}-s_{i,-}-\left(1-\frac{2}{3}\sigma_k\right)s_{6,i}\right], \\
\mathcal{I}_{i,-}(\sigma_k) &=& s_{i,-} + \frac{\sigma_k}{2}\left[s_{i,+}-s_{i,-}+\left(1-\frac{2}{3}\sigma_k\right)s_{6,i}\right].
\end{eqnarray}
\item {\bf Step 4:} Obtain 1D edge states by performing a 1D
  extrapolation to get left and right edge states.  Note that we
  include an explicit source term contribution.
\begin{eqnarray}
s_{L,i+\myhalf} &=& s_i - \chi_i\sum_{k:\lambda_k \ge 0}\lb_k\cdot\left[s_i-\mathcal{I}_{i,+}(\sigma_k)\right]\rb_k + \frac{\dt}{2}S_i^n, \\
s_{R,i-\myhalf} &=& s_i - \chi_i\sum_{k:\lambda_k < 0}\lb_k\cdot\left[s_i-\mathcal{I}_{i,-}(\sigma_k)\right]\rb_k + \frac{\dt}{2}S_i^n.
\end{eqnarray}
Here, $\rb_k$ is the $k^{\rm th}$ right column eigenvector of
$\Rb(\Ab_d)$ and $\lb_k$ is the $k^{\rm th}$ left row eigenvector lf
$\Lb(\Ab_d)$.  The flattening coefficient is $\chi_i$.
\end{itemize}
In order to add the transverse terms in an spatial operator unsplit
framework, the details follow exactly as given in Section 4.2.1 in
Miller and Colella 2002, except for the details of the Riemann solver,
which are given below.
\subsection{Riemann Problem}
Inputs from the edge state prediction are $\rho_{L/R}, u_{L/R},
v_{L/R}, p_{L/R}$, and $(\rho e)_{L/R}$ ($v$ represents all of the
transverse velocity components).  We also compute $\gamma$ at cell
centers and copy these to edges directly to get the left and right
states, $\gamma_{L/R}$.  We also define $c_{\rm avg}$ as a
face-centered value that is the average of the neighboring
cell-centered values of $c$.  We have also computed $\rho_{\rm small},
p_{\rm small}$, and $c_{\rm small}$ using cell-centered data.

Here are the steps.  First, define $(\rho c)_{\rm small} = \rho_{\rm
  small}c_{\rm small}$. Then, define:
\begin{equation}
(\rho c)_{L/R} = \max\left[(\rho c)_{\rm small},\left|\gamma_{L/R},p_{L/R},\rho_{L/R}\right|\right].
\end{equation}
Define star states:
\begin{equation}
p^* = \max\left[p_{\rm small},\frac{\left[(\rho c)_L p_R + (\rho c)_R p_L\right] + (\rho c)_L(\rho c)_R(u_L-u_R)}{(\rho c)_L + (\rho c)_R}\right],
\end{equation}
\begin{equation}
u^* = \frac{\left[(\rho c)_L u_L + (\rho c)_R u_R\right]+ (p_L - p_R)}{(\rho c)_L + (\rho c)_R}.
\end{equation}
If $u^* \ge 0$ then define $\rho_0, u_0, p_0, (\rho e)_0$ and $\gamma_0$ to be the left state.  Otherwise, define them to be the right state.  Then, set
\begin{equation}
\rho_0 = \max(\rho_{\rm small},\rho_0),
\end{equation}
and define
\begin{equation}
c_0 = \max\left(c_{\rm small},\sqrt{\frac{\gamma_0 p_0}{\rho_0}}\right),
\end{equation}
\begin{equation}
\rho^* = \rho_0 + \frac{p^* - p_0}{c_0^2},
\end{equation}
\begin{equation}
(\rho e)^* = (\rho e)_0 + (p^* - p_0)\frac{(\rho e)_0 + p_0}{\rho_0 c_0^2},
\end{equation}
\begin{equation}
c^* = \max\left(c_{\rm small},\sqrt{\left|\frac{\gamma_0 p^*}{\rho^*}\right|}\right)
\end{equation}
Then,
\begin{eqnarray}
c_{\rm out} &=& c_0 - {\rm sign}(u^*)u_0, \\
c_{\rm in} &=& c^* - {\rm sign}(u^*)u^*, \\
c_{\rm shock} &=& \frac{c_{\rm in} + c_{\rm out}}{2}.
\end{eqnarray}
If $p^* - p_0 \ge 0$, then $c_{\rm in} = c_{\rm out} = c_{\rm shock}$.
Then, if $c_{\rm out} = c_{\rm in}$, we define $c_{\rm temp} =
\epsilon c_{\rm avg}$.  Otherwise, $c_{\rm temp} = c_{\rm out} -
c_{\rm in}$.  We define the fraction
\begin{equation}
f = \half\left[1 + \frac{c_{\rm out} + c_{\rm in}}{c_{\rm temp}}\right],
\end{equation}
and constrain $f$ to lie in the range $f\in[0,1]$.

To get the final ``Godunov'' state, for the transverse velocity, we
upwind based on $u^*$.
\begin{equation}
v_{\rm gdnv} =
\begin{cases}
v_L, & u^* \ge 0 \\
v_R, & {\rm otherwise}
\end{cases}.
\end{equation}
Then, define
\begin{eqnarray}
\rho_{\rm gdnv} &=& f\rho^* + (1-f)\rho_0, \\
u_{\rm gdnv} &=& f u^* + (1-f)u_0, \\
p_{\rm gdnv} &=& f p^* + (1-f)p_0, \\
(\rho e)_{\rm gdnv} &=& f(\rho e)^* + (1-f)(\rho e)_0.
\end{eqnarray}
Finally, if $c_{\rm out} < 0$, set $\rho_{\rm gdnv}=\rho_0, u_{\rm
  gdnv}=u_0, p_{\rm gdnv}=p_0$, and $(\rho e)_{\rm gdnv}=(\rho e)_0$.
If $c_{\rm in}\ge 0$, set $\rho_{\rm gdnv}=\rho^*, u_{\rm gdnv}=u^*,
p_{\rm gdnv}=p^*$, and $(\rho e)_{\rm gdnv}=(\rho e)^*$.
\subsection{Compute Fluxes and Update}
Compute the fluxes as a function of the primitive variables, and then
advance the solution:
\begin{equation}
\Ub^{n+1} = \Ub^n - \dt\nabla\cdot\Fb^\nph + \dt\Sb^n.
\end{equation}
Again, note that since the source term is not time centered, this is
not a second-order method.  After the advective update, we correct the
solution, effectively time-centering the source term.
