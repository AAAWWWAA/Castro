
\subsection{List of Parameters}

\begin{table*}[h]
\begin{scriptsize}
\begin{center}
\begin{tabular}{|l|l|l|l|}\hline
Parameter & Definition & Acceptable Values & Default\\
\hline
{\bf castro.do\_rotation} & Include rotation as a forcing term & 0 if false, 1 if true & 0 \\
{\bf castro.rotational\_period} & Period (s) of rotation & Real & 0.0 \\
{\bf castro.rotational\_dPdt} & d(Period) / dt for rotation & Real & 0.0 \\
\hline
\end{tabular}
\end{center}
\end{scriptsize}
\end{table*}

\subsection{Notes}
This is for constant, solid-body rotation about a fixed axis.  See the Rotation section for more detail.
\begin{itemize}
\item To include rotation you must set
\begin{itemize}
\item USE\_ROTATION = TRUE in the GNUMakefile
\item {\bf castro.do\_rotation} = 1 in the inputs file
\end{itemize}
\item {\bf castro.rotational\_period} is only relevant if {\bf castro.do\_rotation} = 1, and must be specified in that case.
\end{itemize}

Currently, CASTRO supports contant, solid-body rotation about a fixed
(in space and time) axis in 2D and 3D by transforming the evolution
equations to the rotating frame of reference.  The rotational
period (in s) is specified by setting {\bf
  castro.rotational\_period} in the inputs file.  Note that this
parameter is only relevant if USE\_ROTATION = TRUE in the GNUmakefile
and {\bf castro.do\_rotation} = 1 in the inputs file.  The rotational
period specified by {\bf castro.rotational\_period} is internally
converted to an angular frequency for use in the source term
equations.

The axis of rotation currently depends on the dimensionality of the
problem and the value of {\bf coord\_sys}; in all cases, however, the
default axis of rotation points from {\bf center}, which is typically
defined in a Prob\_\$(DIM)d.f90 routine, to the typical ``vertical
direction.''  The vertical direction is defined as follows
\begin{itemize}
\item 2D
\begin{itemize}
\item {\bf coord\_sys} = 0, (x,y): out of the (x,y)-plane along the ``z''-axis
\item {\bf coord\_sys} = 1, (r,z): along the z-axis
\end{itemize}
\item 3D
\begin{itemize}
\item {\bf coord\_sys} = 0, (x,y,z): along the z-axis
\end{itemize}
\end{itemize}
To change these defaults, the {\bf omega} vector in the ca\_rotate
routine found in the Rotate\_\$(DIM)d.f90 file.

For completeness, we show below a derivation of the source terms that
appear in the momentum and total energy evolution equations upon
switching to a rotating reference frame.

\section{Coordinate transformation to rotating frame}
Consider an intertial reference frame \(C\) and a non-inertial
reference frame \(\widetilde{C}\) whose origins are separated by the
vector \(\boldsymbol{l}\).  The non-inertial frame is rotating about the axis
\(\ob\) with a \emph{constant} angular velocity \(\omega\);
furthermore, we assume the \emph{direction} of the rotational axis is
fixed.  Consider a fluid element at the point \(P\) whose location is
given by \(\rb\) in \(C\) and by \(\rbt\) in
\(\widetilde{C}\):
  \begin{equation}
    \rb = \rbt + \boldsymbol{l},
  \end{equation}
or in component notation
  \begin{equation}\label{eq:r}
    r_i\boldsymbol{e_i} = \widetilde{r_i}\widetilde{\boldsymbol{e_i}} + l_i\boldsymbol{e_i},
  \end{equation}
where \(\boldsymbol{e_i}\) and \(\widetilde{\boldsymbol{e_i}}\) are the \(i\)th unit
vectors in the \(C\) and \(\widetilde{C}\) coordinate systems,
respectively.  The total time rate of change of \ref{eq:r} is given by
  \begin{equation}\label{eq:vcomp}
    \frac{Dr_i}{Dt}\boldsymbol{e_i} = \frac{D\widetilde{r_i}}{Dt}\widetilde{\boldsymbol{e_i}} + \widetilde{r_i}\frac{D\widetilde{\boldsymbol{e_i}}}{Dt} + \frac{Dl_i}{Dt}\boldsymbol{e_i},
  \end{equation}
where we have used the fact that the unit vectors of the inertial
frame \(C\) are not moving (or at least can be considered stationary,
and the change in \(\boldsymbol{l}\) gives the relative motion of the two
coordinate systems).  By definition, a unit vector can not change its
length, and therefore the only change of \(\widetilde{\boldsymbol{e_i}}\) with
time can come from changing direction.  This change is carried out by
a rotation about the \(\ob\) axis, and the tip of the unit
vector moves circumferentially, that is
  \begin{equation}\label{eq:etilde-rot}
    \frac{D\widetilde{\boldsymbol{e_i}}}{Dt} = \ob\times\widetilde{\boldsymbol{e_i}}.
  \end{equation}
Plugging \ref{eq:etilde-rot} into \ref{eq:vcomp} and switching back to
vector notation, we have
  \begin{equation}\label{eq:r-dot}
    \frac{D\rb}{Dt} = \frac{D\rbt}{Dt} + \ob\times\rbt + \frac{D\boldsymbol{l}}{Dt}.
  \end{equation}
The left hand side of \ref{eq:r-dot} is interpretted as the velocity
of the fluid element as seen in the intertial frame; the first term on the
right hand side is the velocity of the fluid element as seen by a
stationary observer in the rotating frame \(\widetilde{C}\).  The second
and third terms on the right hand side of \ref{eq:r-dot} describe the
additional velocity due to rotation and translation of the frame
\(\widetilde{C}\) as seen in \(C\).  In other words, 
  \begin{equation}\label{eq:v}
    \vb = \vbt + \ob\times\rbt + \boldsymbol{v_l},
  \end{equation}
where we use \(\boldsymbol{v_l}\) to represent the translational velocity.  

Similarly, by taking a second time derivative of \ref{eq:v} we have
  \begin{equation}\label{eq:a}
    \frac{D\vb}{Dt} = \frac{D\vbt}{Dt} + 2\ob\times\vbt + \ob\times\left[\ob\times\rbt\right] + \frac{D\boldsymbol{v_l}}{Dt}.
  \end{equation}

Henceforth we will assume the two coordinate systems are not
translating relative to one another, \(\boldsymbol{v_l} = 0\).  It is
also worth mentioning that derivatives with respect to spatial
coordinates do not involve additional terms due to rotation,
i.e. \(\nablab\cdot\vb = \nablab\cdot\vbt\).
Because of this, the continuity equation remains unchanged in the
rotating frame:
  \begin{equation}\label{eq:cont-rot}
    \frac{\partial \rho}{\partial t} = -\nablab\cdot\left(\rho\vbt\right),
  \end{equation}
or 
  \begin{equation}\label{eq:cont-rot-total}
    \frac{D\rho}{Dt} = -\rho\nablab\cdot\vbt.
  \end{equation}

\section{Momentum equation in rotating frame}
The usual momentum equation applies in an inertial frame:
  \begin{equation}\label{eq:mom1}
    \frac{D\left(\rho\vb\right)}{Dt} = -\rho\vb\cdot\nablab\vb - \nablab p + \rho\gb.
  \end{equation}
Using the continuity equation, \ref{eq:cont-rot-total}, and substituting for
the terms in the rotating frame from \ref{eq:a}, we have from \ref{eq:mom1}:
  \begin{eqnarray}
    \rho\left(\frac{D\vbt}{Dt} + 2\ob\times\vbt + \ob\times\left[\ob\times\rbt\right]\right) - \rho\vb\nablab\cdot\vb &=& -\rho\vb\cdot\nablab\vb - \nablab p + \rho\gb \nonumber \\
    \rho\left(\frac{\partial\vbt}{\partial t} + \vbt\cdot\nablab\vbt\right) &=& -\nablab p + \rho\gb - 2\rho\ob\times\vbt - \rho\ob\times\left[\ob\times\rbt\right] \nonumber \\
  \frac{\partial\left(\rho\vbt\right)}{\partial t} &=& -\nablab\cdot\left(\rho\vbt\vbt\right) - \nablab p + \rho\gb - 2\rho\ob\times\vbt \nonumber \\
  & & -\ \rho\ob\times\left[\ob\times\rbt\right]\label{eq:mom-rot}
  \end{eqnarray}
or
  \begin{equation}\label{eq:mom-rot-tot}
    \frac{D\left(\rho\vbt\right)}{Dt} = -\rho\vbt\cdot\nablab\vbt - \nablab p + \rho\gb - 2\rho\ob\times\vbt - \rho\ob\times\left[\ob\times\rbt\right].
  \end{equation}

\section{Energy equations in rotating frame}
From \ref{eq:mom-rot-tot}, we have the velocity evolution equation in
a rotating frame
  \begin{equation}\label{eq:v-rot}
    \frac{D\vbt}{Dt} = -\frac{1}{\rho}\nablab p + \gb - 2\ob\times\vbt - \ob\times\left[\ob\times\rbt\right].
  \end{equation}
The kinetic energy equation can be obtained from \ref{eq:v-rot} by
mulitplying by \(\rho\vbt\):
  \begin{eqnarray}
    \rho\vbt\cdot\frac{D\vbt}{Dt} &=& -\vbt\cdot\nablab p + \rho\vbt\cdot\gb - 2\rho\vbt\cdot\left[\ob\times\vbt\right] - \rho\vbt\cdot\left\{\ob\times\left[\ob\times\rbt\right]\right\} \nonumber \\
    \frac{1}{2}\frac{D\left(\rho\vbt\cdot\vbt\right)}{Dt} - \frac{1}{2}\vbt\cdot\vbt\frac{D\rho}{Dt} &=& -\vbt\cdot\nablab p + \rho\vbt\cdot\gb - \rho\vbt\cdot\left[\left(\ob\cdot\rbt\right)\ob - \rho\omega^2\rbt\right] \nonumber \\
    \frac{1}{2}\frac{D\left(\rho\vbt\cdot\vbt\right)}{Dt} &=& -\frac{1}{2}\rho\vbt\cdot\vbt\nablab\cdot\vbt - \vbt\cdot\nablab p + \rho\vbt\cdot\gb - \rho\vbt\cdot\left[\left(\ob\cdot\rbt\right)\ob - \rho\omega^2\rbt\right]. \label{eq:ekin-rot-total}
  \end{eqnarray}
The internal energy is simply advected, and, from the first law of
thermodynamics, can change due to \(pdV\) work:
  \begin{equation}\label{eq:eint-rot-total}
    \frac{D\left(\rho e\right)}{Dt} = -\left(\rho e + p\right)\nablab\cdot\vbt.
  \end{equation}
Combining \ref{eq:ekin-rot-total} and \ref{eq:eint-rot-total} we can
get the evolution of the total specific energy in the rotating frame,
\(\rho \widetilde{E} = \rho e + \frac{1}{2}\rho\vbt\cdot\vbt\): 
  \begin{eqnarray}
    \frac{D\left(\rho e\right)}{Dt} + \frac{1}{2}\frac{D\left(\rho\vbt\cdot\vbt\right)}{Dt} &=& -\left(\rho e + p + \frac{1}{2}\rho\vbt\cdot\vbt\right)\nablab\cdot\vbt - \vbt\cdot\nablab p + \rho\vbt\cdot\gb -\rho\vbt\cdot\left[\left(\ob\cdot\rbt\right)\ob - \rho\omega^2\rbt\right]\nonumber \\
    \frac{D\left(\rho \widetilde{E}\right)}{Dt} &=& -\rho\widetilde{E}\nablab\cdot\vbt - \nablab\cdot\left(p\vbt\right) + \rho\vbt\cdot\gb - \rho\vbt\cdot\left[\left(\ob\cdot\rbt\right)\ob - \rho\omega^2\rbt\right] \label{eq:etot-rot-total}
  \end{eqnarray}
or
  \begin{equation}\label{eq:etot-rot}
    \frac{\partial\left(\rho\widetilde{E}\right)}{\partial t} = -\nablab\cdot\left(\rho\widetilde{E}\vbt + p\vbt\right) + \rho\vbt\cdot\gb - \rho\vbt\cdot\left[\left(\ob\cdot\rbt\right)\ob - \rho\omega^2\rbt\right].
  \end{equation}

\section{Switching to the rotating reference frame}
If we choose to be a stationary observer in the rotating reference
frame, we can drop all of the tildes, which indicated terms in the
non-inertial frame \(\widetilde{C}\).  Doing so, and making sure we
account for the offset, \(\boldsymbol{l}\), between the two coordinate systems, we obtain
the following equations for hydrodynamics in a rotating frame of
reference:
  \begin{eqnarray}
    \frac{\partial\rho}{\partial t} &=& -\nablab\cdot\left(\rho\vb\right) \label{eq:cont-rot-switch} \\
    \frac{\partial \left(\rho\vb\right)}{\partial t} &=& -\nablab\cdot\left(\rho\vb\vb\right) - \nablab p + \rho\gb - 2\rho\ob\times\vb - \rho\left(\ob\cdot\rb\right)\ob + \rho\omega^2\rb \label{eq:mom-rot-switch} \\
    \frac{\partial\left(\rho E\right)}{\partial t} &=& -\nablab\cdot\left(\rho E\vb + p\vb\right) + \rho\vb\cdot\gb - \rho\left(\ob\cdot\rb\right)\left(\ob\cdot\vb\right) + \rho\omega^2\left(\vb\cdot\rb\right). \label{eq:etot-rot-switch}
  \end{eqnarray}

\section{Adding the forcing to the hydrodynamics}

There are several ways to incorporate the effect of the rotation forcing
on the hydrodynamical evolution. We control this through the use of the
runtime parameter {\tt castro.rot\_source\_type}. This is an integer
with values currently ranging from 1 through 4, and these values are all
analogous to the way that gravity is used to update the momentum and
energy. For the most part, the differences are in how the energy update is done.

When {\tt castro.rot\_source\_type == 1}, we use a standard predictor-corrector
formalism for updating the momentum and energy. Specifically, our first update
is equal to $\Delta t \times \mathbf{S}^n$ , where $\mathbf{S}^n$ is the value
of the source terms at the old-time (which is usually called time-level $n$).
At the end of the timestep, we do a corrector step where we subtract off
$\Delta t / 2 \times \mathbf{S}^n$ and add on $\Delta t / 2 \times \mathbf{S}^{n+1}$,
so that at the end of the timestep the source term is properly time centered.

When {\tt castro.rot\_source\_type == 2}, we do something very similar. The
major difference is that when evaluating the energy source term at the
new time (which is equal to $\mathbf{u} \cdot \mathbf{S}^{n+1}_{\rho \mathbf{u}}$,
where the latter is the momentum source term evaluated at the new time), we first update
the momentum, rather than using the value of $\mathbf{u}$ before entering the
rotation source terms. This permits a tighter coupling between the momentum and
energy update and we have seen that it usually results in a more accurate evolution.

When {\tt castro.rot\_source\_type == 3}, we do the same momentum update as the
previous two, but for the energy update, we put all of the work into updating
the kinetic energy alone. In particular, we explicitly ensure that $(rho e)$ maintains
the same, and update $(rho K)$ with the work due to rotation, adding the new
kinetic energy to the old internal energy to determine the final total gas energy.
The physical motivation is that work should be done on the velocity, and should not
directly update the temperature -- only indirectly through things like shocks.

When {\tt castro.rot\_source\_type == 4}, the energy update is done in a ``conservative'' fashion.
The previous methods all evaluate the value of the source term at the cell center,
but this method evaluates the change in energy at cell edges, using the hydrodynamical mass
fluxes, permitting total energy to be conserved (excluding possible losses at open domain
boundaries). Additionally, the velocity update is slightly different -- for the corrector step,
we note that there is an implicit coupling between the velocity components, and we directly solve
this coupled equation, which results in a slightly better coupling and a more accurate evolution.
