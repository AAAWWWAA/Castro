\castro\ uses a hybrid MPI + OpenMP approach to parallelism.  The
basic idea is that MPI is used to distribute individual boxes across
nodes while OpenMP is used to distribute the work in local boxes to
the cores within a node.  The OpenMP approach in \castro\ has evolved
considerably since the original paper was written, with the modern
approach, called {\em tiling}, gearing up to meet the demands of
many-core processors in the next-generation of supercomputers.


\section{\boxlib\'s Non-Tiling Approach In C++}
\label{sec:boxlib0}

At the highest abstraction level, we have {\tt MultiFab} (mulitple
{\tt FArrayBox}es).  A {\tt MultiFab} contains an array of {\tt
  Box}es (a {\tt Box} contains integers specifying the index space it
covers), including {\tt Box}es owned by other processors for the
purpose of communication, an array of MPI ranks specifying which MPI
processor owns each {\tt Box}, and an array of pointers to {\tt
  FArrayBox}es owned by this MPI processor.  The real floating point
data are stored as one-dimensional arrays in {\tt FArrayBox}es.  An
{\tt FArrayBox} also contains a box that can be used to reshape the 1D
array into multi-dimensional arrays to be used by Fortran subroutines.

A typical usage of {\tt MultiFab} is as follows,

\begin{lstlisting}
  for (MFIter mfi(mf); mfi.isValid(); ++mfi) // Loop over boxes
  {
    // Get the index space of this iteration
    const Box& box = mfi.validbox(); 

    // Get a reference to the FAB, which contains data and box  
    FArrayBox& fab = mf[mfi];  

    // Get double* of the FAB 
    double* a = fab.dataPtr();

    // Get the index space for the data pointed by the double* 
    // Note "abox" may have ghost cells, and is thus larger than 
    // or equal to "box" obtained using mfi.validbox().
    const Box& abox = fab.box();

    // We can now pass the information to a Fortran routine,
    // which reshapes double* a into a multi-dimensional array 
    // with dimensions specified by the information in "abox".
    // We will also pass "box", which specifies our "work" region.
  }
\end{lstlisting}
A few comments about this code
\begin{itemize}
\item Here the iterator, {\tt mfi}, will perform the loop only over the
   boxes that are local to the MPI task.  If there are 3 boxes on the
   processor, then this loop has 3 iterations.
      
\item {\tt box} as returned from {\tt mfi.validbox()} does not include
   ghost cells.  We can get the indices of the valid zones as {\tt
   box.loVect} and {\tt box.hiVect}.

\item Instead of getting the data pointer explicitly (via {\tt fab.dataPtr()}),
   \castro\ often uses the preprocessor macro {\tt BL\_TO\_FORTRAN(x)} (defined
   in {\tt ArrayLim.H}) to substitute in the data pointer and the {\tt lo}
   and {\tt hi} indices of the valid region.
\end{itemize}


\section{\boxlib\'s Current Tiling Approach In C++}
\label{sec:boxlib1}

There are two types of tiling that people discuss.  In {\em logical
tiling}, the data storage in memory is unchanged from how we do things
now in pure MPI.  In a given box, the data region is stored
contiguously).  But when we loop in OpenMP over a box, the tiling
changes how we loop over the data.  The alternative is called {\em
separate tiling}---here the data storage in memory itself is changed
to reflect how the tiling will be performed.  This is not considered
in \boxlib.

We have recently introduced logical tiling into parts of \boxlib\.  It
is turned off by default, because this makes the transition smooth and
because not everything should be tiled.  Examples can be found at {\tt
  Tutorials/Tiling\_C}, and {\tt Src/LinearSolvers/C\_CellMG/}.

In our logical tiling approach, a box is logically split into tiles,
and a {\tt MFIter} loops over each tile in each box.  Note that the
non-tiling iteration approach can be considered as a special case of
tiling with the tile size equal to the box size.

An example of using tiling is shown below.

\begin{lstlisting}
  bool tiling = true;
  for (MFIter mfi(mf,tiling); mfi.isValid(); ++mfi) // Loop over tiles
  {
    // Get the index space of this iteration
    const Box& box = mfi.tilebox(); 

    // Get a reference to the FAB, which contains data and box  
    FArrayBox& fab = mf[mfi];  

    // Get double* of the FAB 
    double* a = fab.dataPtr();

    // Get the index space for the data pointed by the double*.
    const Box& abox = fab.box();

    // We can now pass the information to a Fortran routine.
  }
\end{lstlisting}
Note that the code is almost identical to the one in \S~\ref{sec:boxlib0}.
Some comments:
\begin{itemize}
\item The iterator now takes an extra argument to turn on tiling
(set to {\tt true}).  There is another interface fo {\tt MFIter}
that can take an {\tt IntVect} that explicitly gives the tile size
in each coordinate direction.

If we don't explictly specify the tile size at the loop, then the
runtime parameter {\tt mfiter\_tile\_size} can be used to set it
globally.

\item {\tt .validBox()} has the same meaning as in the non-tile approach,
so we don't use it.  Instead, we use {\tt .tilebox()} to get the
{\tt Box} (and corresponding {\tt lo} and {\tt hi}) for the {\em
current tile}, not the entire data region.

\item When passing into the Fortran routine, we still use the
index space of the entire fab (including ghost cells), as seen in
the {\tt abox} construction.

The Fortran routine will declare a multidimensional array that is of
the same size as the entire box, but only work on the index space
identified by the tile-box ({\tt box}).
\end{itemize}

Let us consider an example.  Suppose there are four boxes.  The first
box is divided into 4 logical tiles, the second and third are divided
into 2 tiles each (because they are small), and the fourth into 4 tiles.
So there are 12 tiles in total.  In the tiling version, the loop body
will be run 12 times.  Note that {\tt tilebox} is different for each
tile, whereas {\tt fab} might be referencing the same object if the tiles
belong to the same box.  In the non-tiling version (by constructing
{\tt MFIter} without the optional second argument or setting to {\tt
  false}), the loop body will be run 4 times because there are four
boxes, and a call to {\tt mfi.tilebox()} will return the traditional
{\tt validbox}.  The non-tiling case is essentially having one tile
per box.

Tiling provides us the opportunity of a coarse-grained approach for
OpenMP.  Threading can be turned on by inserting the following line
above the {\tt for (MFIter...)} line.
\begin{lstlisting}
  #pragma omp parallel
\end{lstlisting}
Assuming four threads are used in the above example, thread 0 will
work on 3 tiles from the first box, thread 1 on 1 tile from the first
box and 2 tiles from the second box, and so forth.  It should be noted
OpeMP can be used even when tiling is turned off.  In that case, the
OpenMP granularity is at the box level.  It should also be noted that
--- independent of whether or not tiling is on --- OpenMP threading
can also be started within the function called inside the {\tt MFIter}
loop, rather than at the {\tt MFIter} loop level.

The tile size for the three spatial dimensions can be set by a parameter
{\tt fabarray.mfiter\_tile\_size = 1024000 8 8}.  A huge number like
1024000 will turn off tiling in that direction.  The {\tt MFIter}
constructor can also take an explicit tile size: {\tt
  MFIter(mfi(mf,IntVect(128,16,32)))}. 

The {\tt MFIter} class provides some other useful functions:
\begin{lstlisting}
 mfi.validbox()       : The same meaning as before independent of tiling.
 mfi.growntilebox(int): A grown tile box that includes ghost cells at box
                        boundaries only.  Thus the returned boxes for a
                        Fab are non-overlapping.
 mfi.fluxbox(int)     : Returns non-overlapping edge-type boxes for tiles.
                        The argument is for direction.
 mfi.fabbox()         : Same as mf[mfi].box().
\end{lstlisting}

Finally we note that tiling is not always desired or better.  This
traditional fine-grained approach coupled with dynamic scheduling is
more appropriate for work with unbalanced loads, such as chemistry
burning in cells by an implicit solver.  Tiling can also create extra
work in the ghost cells of tiles.


\section{Practical Details in Working with Tiling}

