The Castro executables read run-time information from an "inputs" file (which you put on the command line) 
and from a "probin" file, the name of which is usually defined in the inputs file, but which defaults to "probin".  
To set the "probin" file name in the inputs file:\\

{\bf amr.probin\_file} = my\_special\_probin\\ \\
for example, has the Fortran code read a file called "my\_special\_probin"

\section{Problem Geometry}

\subsection{List of Parameters}

\begin{table*}[h]
\begin{scriptsize}
\begin{tabular}{|l|l|l|l|} \hline
Parameter & Definition & Acceptable Values & Default \\
\hline
{\bf geometry.prob\_lo} & physical location of low corner of the domain & Real & must be set\\
{\bf geometry.prob\_hi} & physical location of high corner of the domain & Real  & must be set\\
{\bf geometry.coord\_sys} & coordinate system & 0 = Cartesian, 1 = r-z, 2 = spherical & must be set \\
{\bf geometry.is\_periodic} & is the domain periodic in this direction & 0 if false, 1 if true  & 0 0 0 \\
\hline
\end{tabular}
\label{Table:Geometry}
\end{scriptsize}
\end{table*}

\subsection{Examples of Usage}

\begin{itemize}

\item {\bf geometry.prob\_lo} = 0 0 0 \\
defines the low corner of the domain at (0,0,0) in physical space.  

\item {\bf geometry.prob\_hi} = 1.e8 2.e8 2.e8 \\
defines the high corner of the domain at (1.e8,2.e8,2.e8) in physical space.  

\item {\bf geometry.coord\_sys} = 0 \\
defines the coordinate system as Cartesian 

\item {\bf geometry.is\_periodic} = 0 1 0 \\
says the domain is periodic in the y-direction only. 

\end{itemize}

\section{Domain Boundary Conditions}

\subsection{List of Parameters}

\begin{table*}[h]
\begin{scriptsize}
\begin{center}
\begin{tabular}{|l|l|l|l|} \hline
Parameter & Definition & Acceptable Values &Default\\
\hline
{\bf castro.lo\_bc} & boundary type of each low face  & 0,1,2,3,4,5 & must be set \\
{\bf castro.hi\_bc} & boundary type of each high face & 0,1,2,3,4,5 & must be set \\
\hline
\end{tabular}
\label{Table:BC}
\end{center}
\end{scriptsize}
\end{table*}

\subsection{Notes}

Boundary types are:

\begin{table*}[h]
\begin{center}
\begin{tabular}{llll} 
0 --  Interior / Periodic \hspace{1.in} & 3  --  Symmetry     \hspace{1.in} & \\
1 --  Inflow              \hspace{1.in} & 4  --  Slip Wall    \hspace{1.in}& \\
2 --  Outflow             \hspace{1.in} & 5  --  No Slip Wall \hspace{1.in}& \\
\end{tabular}
\end{center}
\end{table*}

\noindent Note -- {\bf castro.lo\_bc} and {\bf castro.hi\_bc} must be consistent with 
{\bf geometry.is\_periodic} -- if the domain is periodic in a particular
direction then the low and high bc's must be set to 0 for that direction.

\subsection{Examples of Usage}

\begin{itemize}

\item {\bf castro.lo\_bc} = 1 4 0 

\item {\bf castro.hi\_bc} = 2 4 0 

\item {\bf geometry.is\_periodic} = 0 0 1

\end{itemize}

\noindent would define a problem with inflow (1) in the low-x direction, 
outflow(2) in the high-x direction, slip wall (4) on the low and high y-faces, 
and periodic in the z-direction. 

\section{Resolution}
\subsection{List of Parameters}

\begin{table*}[h]
\begin{scriptsize}
\begin{center}
\begin{tabular}{|l|l|l|l|} \hline
Parameter & Definition & Acceptable Values &Default\\
\hline
{\bf amr.n\_cell} &  number of cells in each direction at the coarsest level & Integer $> 0$ & must be set \\
{\bf amr.max\_level} &  number of levels of refinement above the coarsest level& Integer $\geq 0$  & must be set \\
{\bf amr.ref\_ratio} & ratio of coarse to fine grid spacing between subsequent levels & 2 or 4 & must be set \\
{\bf amr.regrid\_int} & how often to regrid & Integer $> 0$  & must be set \\
{\bf amr.regrid\_on\_restart} & should we regrid immediately after restarting & 0 or 1 & 0 \\
\hline
\end{tabular}
%\caption{Input Parameters -- Resolution}
\label{Table:ResInputs}
\end{center}
\end{scriptsize}
\end{table*}

\noindent Note: if {\bf amr.max\_level} = 0 then you do not need to set {\bf amr.ref\_ratio} or {\bf amr.regrid\_int}.

\subsection{Examples of Usage}

\begin{itemize}

\item {\bf amr.n\_cell} = 32 64 64

would define the domain to have 32 cells in the x-direction, 64 cells in the y-direction, 
and 64 cells in the z-direction {\em{at the coarsest level}}. 
(If this line appears in a 2D inputs file then the final number will be ignored.)

\item {\bf amr.max\_level} = 2 \\ 
would allow a maximum  of 2 refined levels in addition to the coarse level.   
Note that these additional levels will only be created only if the tagging criteria 
are such that cells are flagged as needing refinement.  The number of refined levels 
in a calculation must be $\leq$ {\bf amr.max\_level}, but can change in time and need
not always be equal to {\bf amr.max\_level}.
 
\item {\bf amr.ref\_ratio} = 2 4 \\ 
would set factor 2 refinement between levels 0 and 1, and factor 4 refinement between levels 1 and 2. 
Note that you must have at least {\bf amr.max\_level} values of {\bf amr.ref\_ratio} 
(Additional values may appear in that line and they will be ignored).

\item {\bf amr.regrid\_int} = 2 2 \\
tells the code to regrid every 2 steps.  Thus in this example, new level 1 grids will be created every 2 level 0 time steps, and
new level 2 grids will be created every 2 level 1 time steps.

\end{itemize}

\section{Tagging}

\subsection{List of Parameters}

\begin{table*}[h]
\begin{scriptsize}
\begin{center}
\begin{tabular}{|l|l|l|l|} \hline
Parameter & Definition & Acceptable Values &Default\\
\hline
{\bf castro.allow\_untagging} & are cells allowed to be "untagged" & 0 or 1 & 0 \\
\hline
\end{tabular}
\label{Table:Tagging}
\end{center}
\end{scriptsize}
\end{table*}

\subsection{Notes}

\begin{itemize}
\item Typically cells at a given level can be tagged as needing refinement by any of a number
of criteria, but cannot be "untagged", i.e. once tagged no other criteria can untag them.   If
we set {\bf castro.allow\_untagging} = 1 then the user is allowed to "untag" cells in the
Fortran tagging routines
\end{itemize}

\section{Regridding}
\subsection{Overview}

The details of the regridding strategy are described in a later section; here we 
cover how the input parameters can control the gridding.

As described later, the user defines Fortran subroutines which tag individual
cells at a given level if they need refinement.  This list of tagged cells is
sent to a grid generation routine, which uses the Berger-Rigoutsis algorithm
to create rectangular grids that contain the tagged cells.   

\subsection{List of Parameters}

\begin{table*}[h]
\begin{scriptsize}
\begin{center}
\begin{tabular}{|l|l|l|l|} \hline
Parameter & Definition & Acceptable Values &Default\\
\hline
{\bf amr.regrid\_file} & name of file from which to read the grids & text & no file  \\ 
{\bf amr.grid\_eff} & grid efficiency & Real $>0$ and $<1$ & 0.7 \\ 
{\bf amr.n\_error\_buf} & radius of additional tagging around already tagged cells & Integer $\geq 0$ & 1 \\ 
{\bf amr.max\_grid\_size} & maximum size of a grid in any direction & Integer $> 0$ & 128 in 2D, 32 in 3D \\ 
{\bf amr.blocking\_factor} &  grid size must be a multiple of this & Integer $> 0$ & 2\\
{\bf amr.refine\_grid\_layout} & refine grids more if \# of processors $>$ \# of grids &  0 if false, 1 if true & 1 \\
\hline
\end{tabular}
\label{Table:GriddingInputs}
\end{center}
\end{scriptsize}
\end{table*}

\subsection{Notes}

\begin{itemize}
\item {\bf amr.n\_error\_buf}, {\bf amr.max\_grid\_size} and {\bf amr.blocking\_factor} 
can be read in as a single value which is assigned to every level, 
or as multiple values, one for each level
\item {\bf amr.max\_grid\_size} at every level must be even 
\item {\bf amr.blocking\_factor} at every level must be a power of 2
\item the domain size must be a multiple of  {\bf amr.blocking\_factor} at level 0
\item {\bf amr.max\_grid\_size} must be a multiple of {\bf amr.blocking\_factor} at every level
\end{itemize}

\subsection{Examples of Usage}

\begin{itemize} 

\item {\bf amr.regrid\_file} = fixed\_grids \\ 
In this case the list of grids at each fine level are contained in the file, {\em fixed\_grids},
which will be read during the gridding procedure.  These grids must not violate the
{\bf amr.max\_grid\_size} criterion.   The rest of the gridding procedure described below
will not occur if {\bf amr.regrid\_file} is set.

\item {\bf amr.grid\_eff} = 0.9\\ 
During the grid creation process, at least 90\% of the cells in each grid at the
level at which the grid creation occurs must be tagged cells.  

\item {\bf amr.max\_grid\_size} = 64 \\ 
The final grids will be no longer than 64 cells on a side at every level.

\item {\bf amr.max\_grid\_size} = 64 32 16 \\ 
The final grids will be no longer than 64 cells on a side
at level 0, 32 cells on a side at level 1, and 16 cells on a side at level 2.

\item {\bf amr.blocking\_factor} = 32 \\
The dimensions of all the final grids will be multiples of 32 at all levels.

\item {\bf amr.blocking\_factor} = 32 16 8\\
The dimensions of all the final grids will be multiples of 32
at level 0, multiples of 16 at level 1, and multiples of 8 at level 2..  

Having grids that are large enough to coarsen multiple levels in a V-cycle is essential
for good multigrid performance in simulations that use self-gravity.

\end{itemize}

\subsection{How Grids are Created}

\noindent The gridding algorithm proceeds in this order:

\begin{enumerate}
\item Grids are created using the Berger-Rigoutsis clustering algorithm 
modified to ensure that all new fine grids are divisible by {\bf amr.blocking\_factor}.

\item Next, the grid list is chopped up if any grids are larger than {\bf max\_grid\_size}.
Note that because {\bf amr.max\_grid\_size} is a multiple of {\bf amr.blocking\_factor}
the {\bf amr.blocking\_factor} criterion is still satisfied. 

\item Next, if {\bf amr.refine\_grid\_layout} = 1 and there are more processors than grids, and
\begin{itemize}
\item if {\bf amr.max\_grid\_size} / 2 is a multiple of {\bf amr.blocking\_factor}
\end{itemize}
then the grids will be redefined, at each level independently, so that the 
maximum length of a grid at level $\ell$, in any dimension, 
is {\bf amr.max\_grid\_size}[$\ell$] / 2.  

\item Finally, if {\bf amr.refine\_grid\_layout} = 1,  and there are still more processors
than grids, and 
\begin{itemize}
\item if {\bf amr.max\_grid\_size} / 4 is a multiple of {\bf amr.blocking\_factor}
\end{itemize}
then the grids will be redefined, at each level independently, so that the 
maximum length of a grid at level $\ell$, in any dimension, 
is {\bf amr.max\_grid\_size}[$\ell$] / 4.


\end{enumerate}

\section{Simulation Time}

\subsection{List of Parameters}
\begin{table*}[h]
\begin{scriptsize}
\begin{center}
\begin{tabular}{|l|l|l|l|} \hline
Parameter & Definition & Acceptable Values &Default \\
\hline
{\bf max\_step} & maximum number of level 0 time steps & Integer $\geq 0$ & -1 \\
{\bf stop\_time} & final simulation time & Real $\geq 0$ & -1.0 \\
\hline
\end{tabular}
\label{Table:TimeInputs}
\end{center}
\end{scriptsize}
\end{table*}

\subsection{Notes}
To control the number of time steps, you can limit by the maximum number of 
level 0 time steps ({\bf max\_step}) or by the final simulation time ({\bf stop\_time}), 
or both. The code will stop at whichever criterion comes first. 
Note that if the code reaches {\bf stop\_time} then the final time step will be shortened so as to end 
exactly at {\bf stop\_time}, not pass it.

\subsection{Examples of Usage}
\begin{itemize}
\item {\bf max\_step}  = 1000
\item {\bf stop\_time}  = 1.0
\end{itemize}
will end the calculation when either the simulation time reaches 1.0 or 
the number of level 0 steps taken equals 1000, whichever comes first.

\section{Time Step}
If {\bf castro.do\_hydro}$ = 1$, then typically 
the code chooses a time step based on the CFL number (dt = cfl * dx / max(u+c) ).

\subsection{List of Parameters}

\begin{table*}[h]
\begin{scriptsize}
\begin{center}
\begin{tabular}{|l|l|l|l|} \hline
Parameter & Definition & Acceptable Values &Default\\
\hline
{\bf castro.cfl} & CFL number & Real $> 0$ and $\leq 1$ & 0.8 \\ 
{\bf castro.init\_shrink} & factor by which to shrink the initial time step & Real $> 0$ and $\leq 1$ & 1.0 \\ 
{\bf castro.change\_max} & factor by which the time step can grow in subsequent steps & Real $\geq 1$ & 1.1 \\ 
{\bf castro.fixed\_dt} &  level 0 time step regardless of cfl or other settings & Real $> 0$ & unused if not set \\
{\bf castro.initial\_dt} &  initial level 0 time step regardless of other settings & Real $> 0$ & unused if not set \\
{\bf castro.dt\_cutoff} & time step below which calculation will abort & Real $> 0$ & 0.0 \\
\hline
\end{tabular}
\label{Table:TimeStepInputs}
\end{center}
\end{scriptsize}
\end{table*}

\subsection{Examples of Usage}

\begin{itemize}

\item {\bf castro.cfl} = 0.9 \\
defines the timestep as dt = cfl * dx / umax.

\item {\bf castro.init\_shrink} = 0.01 \\
sets the initial time step to 1\% of what it would be otherwise.  

\item {\bf castro.change\_max} = 1.1\\
allows the time step to increase by no more than 10\% in this case.    Note that the time step
can shrink by any factor; this only controls the extent to which it can grow.

\item {\bf castro.fixed\_dt} = 1.e-4\\
sets the level 0 time step to be 1.e-4 for the entire simulation, 
ignoring the other timestep controls.   Note that if {\bf castro.init\_shrink} $\neq 1$
then the first time step will in fact be {\bf castro.init\_shrink} * {\bf castro.fixed\_dt}.

\item {\bf castro.initial\_dt} = 1.e-4\\
sets the {\it initial} level 0 time step to be 1.e-4 regardless of 
{\bf castro.cfl} or {\bf castro.fixed\_dt}.  The time step can
grow in subsequent steps by a factor of {\bf castro.change\_max} each step.

\item {\bf castro.dt\_cutoff} = 1.e-20\\
tells the code to abort if the time step ever gets below 1.e-20. 
This is a safety mechanism so that if things go nuts you don't burn through your 
entire computer allocation because you don't realize the code is misbehaving.

\end{itemize}

\section{Subcycling}
Castro supports a number of different modes for subcycling in time.

\begin{itemize}
\item If {\bf amr.subcycling\_mode}$ = $Auto (default), then the code 
will run with equal refinement in space and time. In other words, if 
level $n+1$ is a factor of 2 refinement above level $n$, then $n+1$ 
will take 2 steps of half the duration for every level $n$ step.

\item If {\bf amr.subcycling\_mode}$ = $None, then the code 
will not refine in time. All levels will advance together with a 
timestep dictated by the level with the strictest $dt$. Note that this 
is identical to the deprecated command {\bf amr.nosub = 1}.

\item If {\bf amr.subcycling\_mode}$ = $Manual, then the code will 
subcycle according to the values supplied by {\bf 
subcycling\_iterations}.

\end{itemize}

\subsection{List of Parameters}

\begin{table*}[h]
\begin{scriptsize}
\begin{center}
\begin{tabular}{|l|l|l|l|} \hline
Parameter & Definition & Acceptable Values &Default\\
\hline
{\bf amr.subcycling\_mode}& How shall we subcycle &Auto, None or Manual & Auto \\ 
{\bf amr.subcycling\_iterations} & Number of cycles at each level & 1 
or {\tt ref\_ratio} & must be set in Manual mode \\ 
\hline
\end{tabular}
\end{center}
\end{scriptsize}
\end{table*}

\subsection{Examples of Usage}

\begin{itemize}

\item {\bf amr.subcycling\_mode}$ = $Manual \\
Subcycle in manual mode with largest allowable timestep.

\item {\bf amr.subcycling\_iterations} = 1 2 1 2\\
Take 1 level 0 timestep at a time (required). Take 2 level 1 timesteps for each level 0 step, 1 timestep at level 2 
for each level 1 step, and take 2 timesteps at level 3 for each level 
2 step.

\item {\bf amr.subcycling\_iterations} = 2\\
Alternative form. Subcycle twice at every level (except level 0).

\end{itemize}

\section{Restart Capability}

Castro has a standard sort of checkpointing and restarting capability. 
In the inputs file, the following options control the generation of checkpoint files (which are really
directories):\\

\subsection{List of Parameters}

\begin{table*}[h]
\begin{scriptsize}
\begin{center}
\begin{tabular}{|l|l|l|l|} \hline
Parameter & Definition & Acceptable Values &Default\\
\hline
{\bf amr.check\_file} & prefix for restart files & Text & "chk" \\
{\bf amr.check\_int}  & how often (by level 0 time steps) to write restart files & Integer $> 0$ & -1  \\
{\bf amr.check\_per}  & how often (by simulation time) to write restart files & Real $> 0$ & -1.0 \\
{\bf amr.restart}  & name of the file (directory) from which to restart & Text & not used if not set \\
{\bf amr.checkpoint\_files\_output} & should we write checkpoint files & 0 or 1 & 1 \\
{\bf amr.check\_nfiles}  & how parallel is the writing of the checkpoint files & Integer $\geq 1$ & 64 \\
{\bf amr.checkpoint\_on\_restart} & should we write a checkpoint immediately after restarting 
  & 0 or 1 & 0 \\
{\bf castro.grown\_factor} & Factor by which domain has been grown & Integer $\geq 1$ & 1 \\
\hline
\end{tabular}
\end{center}
\end{scriptsize}
\end{table*}

\subsection{Notes}

\begin{itemize}

\item You should specify either {\bf amr.check\_int} or {\bf amr.check\_per}.  Do not try to specify both. 

\item Note that if {\bf amr.check\_per} is used then in order to hit that exact time the code 
may modify the time step slightly, which will change your results ever so slightly than if 
you didn't set this flag.

\item Note that {\bf amr.plotfile\_on\_restart} and {\bf amr.checkpoint\_on\_restart} 
only take effect if {\bf amr.regrid\_on\_restart} is in effect.

\item See the Software Section for more details on parallel I/O and the 
{\bf amr.check\_nfiles} parameter.

\item If you are doing a scaling study then set {\bf amr.checkpoint\_files\_output} = 0
so you can test scaling of the algorithm without I/O.

\end{itemize}

\subsection{Examples of Usage}

\begin{itemize}

\item {\bf amr.check\_file} = chk\_run
\item {\bf amr.check\_int} = 10

means that restart files (really directories) starting with the prefix "chk\_run" will be
generated every 10 level 0 time steps.  The directory names will be {\it chk\_run00000}, 
{\it chk\_run00010}, {\it chk\_run00020}, etc.

\end{itemize}

If instead you specify

\begin{itemize}

\item {\bf amr.check\_file} = chk\_run
\item {\bf amr.check\_per} = 0.5

then restart files (really directories) starting with the prefix "chk\_run" will be
generated every 0.1 units of simulation time.  The directory names will be {\it chk\_run00000}, 
{\it chk\_run00043}, {\it chk\_run00061}, etc, where $t = 0.1$ after 43 level 0 steps, 
$t = 0.2$ after 61 level 0 steps, etc.

\end{itemize}

To restart from {\it chk\_run00061},for example, then set 

\begin{itemize}
\item {\bf amr.restart} = chk\_run00061
\end{itemize}

\section{Controlling PlotFile Generation}
\label{sec:PlotFiles}
The main output from Castro is in the form of plotfiles (which are really directories).
The following options in the inputs file control the generation of plotfiles 

\subsection{List of Parameters}

\begin{table*}[h]
\begin{scriptsize}
\begin{center}
\begin{tabular}{|l|l|l|l|} \hline
Parameter & Definition & Acceptable Values &Default\\
\hline
{\bf amr.plot\_file} & prefix for plotfiles & Text & "plt" \\
{\bf amr.plot\_int}  & how often (by level 0 time steps) to write plot files & Integer $> 0$ & -1  \\
{\bf amr.plot\_per}  & how often (by simulation time) to write plot files & Real $> 0$ & -1.0 \\
{\bf amr.plot\_vars}  & name of state variables to include in plotfiles 
 & ALL, NONE or list & ALL \\
{\bf amr.derive\_plot\_vars}  & name of derived variables to include in plotfiles 
 & ALL, NONE or list & NONE \\
{\bf castro.plot\_X}  & include all the species mass fractions in the plotfile 
 & 0 or 1 & 0 \\
{\bf amr.plot\_files\_output} & should we write plot files & 0 or 1 & 1 \\
{\bf amr.plotfile\_on\_restart} & should we write a plotfile immediately after restarting 
  & 0 or 1 & 0 \\
{\bf amr.plot\_nfiles}  & how parallel is the writing of the plotfiles & Integer $\geq 1$ & 64 \\
{\bf castro.plot\_phiGrav} & Should we plot the gravitational potential & 0 or 1 & 0 \\
\hline
\end{tabular}
\end{center}
\end{scriptsize}
\end{table*}

All the options for {\bf amr.derive\_plot\_vars} are kept in \texttt{derive\_lst} in Castro\_setup.cpp.  Feel free to look at it and see what's there. 


\subsection{Notes}

\begin{itemize}

\item You should specify either {\bf amr.plot\_int} or {\bf amr.plot\_per}.  Do not try to specify both. 

\item Note that if {\bf amr.plot\_per} is used then in order to hit that exact time the 
code may modify the time step slightly, which will change your results ever so slightly 
than if you didn't set this flag.

\item See the Software Section for more details on parallel I/O and the 
{\bf amr.plot\_nfiles} parameter.

\item If you are doing a scaling study then set {\bf amr.plot\_files\_output} = 0
so you can test scaling of the algorithm without I/O.

\item {\bf castro.plot\_phiGrav} is only relevant if 
{\bf castro.do\_grav} = 1 and {\bf gravity.gravity\_type} = PoissonGrav

\end{itemize}

\subsection{Examples of Usage}

\begin{itemize}

\item {\bf amr.plot\_file} = plt\_run
\item {\bf amr.plot\_int} = 10

means that plot files (really directories) starting with the prefix "plt\_run" will be
generated every 10 level 0 time steps.  The directory names will be {\it plt\_run00000}, 
{\it plt\_run00010}, {\it plt\_run00020}, etc.

\end{itemize}

If instead you specify

\begin{itemize}

\item {\bf amr.plot\_file} = plt\_run
\item {\bf amr.plot\_per} = 0.5

then restart files (really directories) starting with the prefix "plt\_run" will be
generated every 0.1 units of simulation time.  The directory names will be {\it plt\_run00000}, 
{\it plt\_run00043}, {\it plt\_run00061}, etc, where $t = 0.1$ after 43 level 0 steps, 
$t = 0.2$ after 61 level 0 steps, etc.

\end{itemize}

\section{Screen Output}

\subsection{List of Parameters}

\begin{table*}[h]
\begin{scriptsize}
\begin{center}
\begin{tabular}{|l|l|l|l|} \hline
Parameter & Definition & Acceptable Values &Default\\
\hline
{\bf amr.v} & verbosity of Amr.cpp & 0 or 1 & 0 \\
{\bf castro.v} & verbosity of Castro.cpp & 0 or 1 & 0 \\
{\bf gravity.v} & verbosity of Gravity.cpp & 0 or 1 & 0 \\
{\bf diffusion.v} & verbosity of Diffusion.cpp & 0 or 1 & 0 \\
{\bf mg.v} & verbosity of multigrid solver (for gravity) & 0,1,2,3,4 & 0 \\
{\bf amr.grid\_log}       & name of the file to which the grids are written & Text & not used if not set \\
{\bf amr.run\_log}        & name of the file to which certain output is written & Text & not used if not set \\
{\bf amr.run\_log\_terse} & name of the file to which certain (terser) output is written & Text & not used if not set \\
{\bf amr.sum\_interval}   & if $> 0,$ how often (in level 0 time steps) & &  \\  
                          & to compute and print integral quantities & Integer & -1 \\  
{\bf castro.do\_special\_tagging} & & 0 or 1 & 1 \\
\hline
\end{tabular}
\end{center}
\end{scriptsize}
\end{table*}

\subsection{Notes}

\begin{itemize}

\item {\bf castro.do\_special\_tagging} = 1 allows the user to set a special flag based on
user-specified criteria.  This can be used, for example, to calculate the bounce time in a 
core collapse simulation; the bounce time is defined as the first time at which the maximum
density in the domain exceeds a user-specified value.   This time can then be printed into
a special file as a useful diagnostic.

\end{itemize}

\subsection{Examples of Usage}

\begin{itemize}

\item {\bf amr.grid\_log} = grdlog \\
Every time the code regrids it prints a list of grids at all relevant levels.  
Here the code will write these grids lists into the file {\em grdlog}.

\item {\bf amr.run\_log} = runlog \\ 
Every time step the code prints certain statements to the screen (if {\bf amr.v} = 1), such as \\
STEP = 1 TIME = 1.91717746 DT = 1.91717746 \\
PLOTFILE: file = plt00001 \\
Here these statements will be written into {\em runlog} as well.

\item {\bf amr.run\_log\_terse} = runlogterse \\ 
This file, {\em runlogterse} differs from {\em runlog}, in that it only contains lines
of the form \\ 
10  0.2  0.005 \\
in which "10" is the number of steps taken, "0.2" is the simulation time, and "0.005" is the 
level 0 time step.  This file can be plotted very easily to monitor the time step.

\item {\bf castro.sum\_interval} = 2 \\
if {\bf castro.sum\_interval} $> 0$ then the code computes and prints certain integral quantities, 
such as total mass, momentum and energy in the domain every {\bf castro.sum\_interval} level 0 steps. 
In this example the code will print these quantities every two coarse time steps.  The print 
statements have the form \\
TIME= 1.91717746 MASS= 1.792410279e+34 \\
for example. 
If this line is commented out then it will not compute and print these quanitities.

\end{itemize}

\section{Gravity}

\subsection{List of Parameters}

\begin{table*}[h]
\begin{scriptsize}
\begin{center}
\begin{tabular}{|l|l|l|l|} \hline
Parameter & Definition & Acceptable Values &Default\\
\hline
{\bf castro.do\_grav}  & Include gravity as a forcing term & 0 if false, 1 if true & must be set \\
& & & if USE\_GRAV = TRUE  \\
\hline
{\bf gravity.gravity\_type} & if {\bf castro.do\_grav} = 1, & ConstantGrav, & \\
& how shall gravity be calculated & PoissonGrav, & \\
& & or MonopoleGrav & must be set \\
\hline
{\bf gravity.const\_grav} & if {\bf gravity.gravity\_type} = ConstantGrav, & & \\
& set the value of constant gravity & Real & 0.0 \\
\hline
{\bf gravity.no\_sync} & if {\bf gravity.gravity\_type} = PoissonGrav, & & \\
& whether to perform the ``sync solve" &  0 or 1 & 0 \\
\hline
{\bf gravity.no\_composite} & if {\bf gravity.gravity\_type} = PoissonGrav, & & \\
& whether to perform a composite solve & 0 or 1 & 0 \\
\hline
{\bf gravity.max\_multipole\_order} & if {\bf gravity.gravity\_type} = PoissonGrav, & & \\
& max $\ell$ value to use for multipole BCs & Integer $\geq 0$ & 0 \\
\hline
{\bf gravity.direct\_sum\_bcs} & if {\bf gravity.gravity\_type} = PoissonGrav, & & \\
& evaluate BCs using exact sum & 0 or 1 & 0 \\
\hline
\end{tabular}
\label{Table:Gravity}
\end{center}
\end{scriptsize}
\end{table*}

\subsection{Notes}

Gravity types are ConstantGrav, PoissonGrav, or MonopoleGrav.  See the Gravity section for more detail.

\begin{itemize}
\item To include gravity you must set 
\begin{itemize}
\item USE\_GRAV  = TRUE in the GNUmakefile 
\item {\bf castro.do\_grav} = 1 in the inputs file
\end{itemize}
\item {\bf gravity.gravity\_type} is  only relevant if {\bf castro.do\_grav} = 1 
\item {\bf gravity.no\_sync} and {\bf gravity.no\_composite} are only relevant if {\bf gravity.gravity\_type} = PoissonGrav,
i.e. the code does a full Poisson solve for self-gravity.
\item {\bf gravity.max\_multipole\_order} and {\bf gravity.direct\_sum\_bcs} are only relevant if {\bf gravity.gravity\_type} = PoissonGrav,
and they are described in Section \ref{sec-poisson-3d-bcs}.
\end{itemize}

\section{Diffusion}

\subsection{List of Parameters}

\begin{table*}[h]
\begin{scriptsize}
\begin{center}
\begin{tabular}{|l|l|l|l|} \hline
Parameter & Definition & Acceptable Values &Default\\
\hline
{\bf castro.diffuse\_temp} & Include thermal diffusion & 0 if false, 1 if true & 0 \\
{\bf diffusion.diff\_coeff} & & Real $> 0$ & 0.0 \\
\hline
\end{tabular}
\label{Table:Diffusion}
\end{center}
\end{scriptsize}
\end{table*}

\subsection{Notes}
\begin{itemize}
\item To include diffusion you must set
\begin{itemize}
\item USE\_DIFFUSION  = TRUE in the GNUmakefile
\item {\bf castro.diffuse\_temp} = 1 in the inputs file
\end{itemize}
\item You can run a pure diffusion problem (with no hydrodynamics) by setting 
\begin{itemize}
\item {\bf castro.diffuse\_temp} = 1
\item {\bf castro.do\_hydro} = 0 
\end{itemize}
\item {\bf diffusion.diff\_coeff} is only relevant if {\bf castro.diffuse\_temp} = 1 
\end{itemize}

\section{Rotation}

\subsection{List of Parameters}

\begin{table*}[h]
\begin{scriptsize}
\begin{center}
\begin{tabular}{|l|l|l|l|}\hline
Parameter & Definition & Acceptable Values & Default\\
\hline
{\bf castro.do\_rotation} & Include rotation as a forcing term & 0 if false, 1 if true & 0 \\
{\bf castro.rotational\_frequency} & Frequency (Hz) of rotation & Real & 0.0 \\
\hline
\end{tabular}
\end{center}
\end{scriptsize}
\end{table*}

\subsection{Notes}
This is for constant, solid-body rotation about a fixed axis.  See the Rotation section for more detail.
\begin{itemize}
\item To include rotation you must set
\begin{itemize}
\item USE\_ROTATION = TRUE in the GNUMakefile
\item {\bf castro.do\_rotation} = 1 in the inputs file
\end{itemize}
\item {\bf castro.rotational\_frequency} is only relevant if {\bf castro.do\_rotation} = 1
\end{itemize}

\section{Physics}

\subsection{List of Parameters}

\begin{table*}[h]
\begin{scriptsize}
\begin{center}
\begin{tabular}{|l|p{3.0in}|l|l|} \hline
Parameter & Definition & Acceptable Values &Default\\
\hline
{\bf castro.do\_hydro} & Time-advance the fluid dynamical equations & 0 if false, 1 if true &
 must be set \\
{\bf castro.do\_react} & Include reactions                          & 0 if false, 1 if true & 
 must be set \\
% {\bf castro.do\_radiation} & Include radiation                      & 0 if false, 1 if true &
%  must be set if USE\_RAD = TRUE \\
{\bf castro.add\_ext\_src} & Include additional user-specified source term & 0 if false, 1 if true & 0 \\
{\bf castro.point\_mass}   & Point mass at the center of the star & Real $\geq 0$ & 0.0 \\
{\bf castro.do\_sponge} & Call a user-supplied sponging routine after the solution update & 0 or 1 & 0 \\
{\bf castro.normalize\_species} & Enforce that $\sum_i X_i = 1$ & 0 or 1 & 0 \\
{\bf castro.fix\_mass\_flux} & Enforce constant mass flux at domain boundary & 0 or 1 & 1 \\
{\bf castro.allow\_negative\_energy} & Is internal energy allowed to be negative & 0 or 1 & 1 \\
{\bf castro.ppm\_type} & Use piecewise linear vs PPM algorithm & 0,1,2 & 1 \\
{\bf castro.ppm\_reference} & Use the integral under the parabolic interpolant as the reference state (=1; original PPM algorithm),
                              or the cell-centered quantity (=0; Castro paper) & 0,1 & 1 \\
{\bf castro.ppm\_temp\_fix} & Different methods of reconstructing using temperature information (experimental) & 0,1,2 & 0 \\
{\bf castro.ppm\_trace\_grav} & Reconstruct the gravitational acceleration as a parabola and do characteristic tracing on it? (1=yes) & 0,1 & 0 \\
{\bf castro.use\_colglaz} & Use the Colella/Glaz Riemann solver (2- and 3-d) or the original C-G algorithm (1-d)? & 0 or 1 & 0 \\
{\bf castro.cg\_maxiter} & Number of iterations for C/G algorithm & Integer & 12 \\
{\bf castro.cg\_tol} & Tolerance for C/G solver & Real & 1.0d-5 \\
{\bf castro.gamma} & Sets the value of $\gamma$ & Real & 0.0 \\
{\bf castro.spherical\_star} & & 0 or 1 & 0 \\
{\bf castro.show\_center\_of\_mass} & & 0 or 1 & 0 \\
{\bf castro.small\_dens} & & Real & -1.e20 \\
{\bf castro.small\_temp} & & Real & -1.e20 \\
{\bf castro.small\_pres} & & Real & -1.e20 \\
\hline
\end{tabular}
\label{Table:Physics}
\end{center}
\end{scriptsize}
\end{table*}

\subsection{Notes}

\begin{itemize}
\item You must have USE\_POINTMASS  = TRUE in the GNUmakefile for {\bf castro.point\_mass} to be relevant.
\item {\bf castro.gamma} is only relevant for a gamma law gas.
\item {\bf castro.use\_colglaz}  = 1 is only implemented in 1D
\end{itemize}

% \section{Radiation}

% \begin{table*}[h]
% \begin{scriptsize}
% \begin{center}
% \begin{tabular}{|l|l|l|l|} \hline
% Parameter & Definition & Acceptable Values &Default\\
% \hline
% {\bf radiation.do\_timing} & & & \\
% {\bf radiation.do\_clouds} & & & \\
% {\bf radiation.do\_shadow} & & & \\
% {\bf radiation.do\_thermal\_wave} & & & \\
% {\bf radiation.do\_crooked\_pipe} & & & \\
% {\bf radiation.do\_linearMGD\_tp} & & & \\
% {\bf radiation.do\_neutrino\_test} & & & \\
% {\bf radiation.do\_rad\_sphere} & & & \\
% {\bf radiation.do\_light\_front} & & & \\
% {\bf radiation.do\_multigroup} & & & \\
% {\bf radiation.ngroups} & & & \\
% {\bf radiation.nNeutrinoSpecies} & & & \\
% {\bf radiation.nNeutrinoGroups} & & & \\
% {\bf radiation.plot\_neutrino\_group\_energies\_total} & & & \\
% \hline
% \end{tabular}
% \label{Table:Radiation}
% \end{center}
% \end{scriptsize}
% \end{table*}

% \begin{itemize}
% \item You must have USE\_RAD  = TRUE in the GNUmakefile for {\bf castro.do\_radiation} to be relevant.
% \end{itemize}

