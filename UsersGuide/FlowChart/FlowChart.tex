Our equations look like:
\begin{equation}
\frac{\partial\Ub}{\partial t} = \nabla\cdot\Fb + \Sb_{\rm react} + \Sb,
\end{equation}
where $\Fb$ is the flux vector, $\Sb_{\rm react}$ are the reaction
source terms, and $\Sb$ are the non-reaction source terms, which
includes any user-defined external sources, $\Sb_{\rm ext}$.  We use
Strang splitting to discretize the advection-reaction equations.  In
summary, for each time step, we update the conservative variables,
$\Ub$, by reacting for half a time step, advecting for a full time
step (ignoring the reaction terms), and reacting for half a time step.
In summary,
\begin{equation}
\Ub^n = \Ub^n + \frac{\dt}{2}\Sb_{\rm react}^n,
\end{equation}
\begin{equation}
\Ub^{n+1} = \Ub^n - \Delta t \nabla \cdot\Fb^\nph + \dt\frac{\Sb^n + \Sb^{n+1}}{2},
\end{equation}
\begin{equation}
\Ub^{n+1} = \Ub^{n+1} + \frac{\dt}{2}\Sb_{\rm react}^{n+1},
\end{equation}
The construction of $F$ is purely explicit, and based on an unsplit
second-order Godunov method.  We predict the standard primitive
variables, as well as $\rho e$, at time-centered edges and use an
approximate Riemann solver construct fluxes.  At the beginning of the
time step, we assume that $\Ub$ and $\phi$ are defined consistently,
i.e., $\rho^n$ and $\phi^n$ satisfy equation (\ref{eq:Self
  Gravity}).\\

\castro\ also supports radiation (Chapter \ref{Chap:Radiation}) and
level sets (Chapter \ref{Chap:Level Sets}).  We omit the details in
this section.  Here is the single-level algorithm:
\begin{description}
\item[Step 1:] {\em React $\Delta t/2$.}

Update the solution due to the effect of reactions over half a time step.
\begin{eqnarray}
(\rho E)^n &=& (\rho E)^n - \frac{\dt}{2}\sum_k(\rho q_k\omegadot_k)^n,\\
(\rho X_k)^n &=& (\rho X_k)^n + \frac{\dt}{2}(\rho\omegadot_k)^n.
\end{eqnarray}
\item[Step 2:] {\em Solve for gravity.}

Solve for gravity using:
\begin{equation}
\gb^n = \nabla\phi^n, \qquad 
\Delta\phi^n = -4\pi G\rho^n,
\end{equation}
or use one of the simpler gravity types.

\item[Step 3:] {\em Compute explicit source terms.}

We now compute explicit source terms for each variable in $\Qb$ and
$\Ub$.  The primitive variable source terms will be used to construct
time-centered fluxes.  The conserved variable source will be used to
advance the solution.  We neglect reaction source terms since they are
accounted for in {\bf Steps 1} and {\bf 6}.  The source terms are:
\begin{equation}
\Sb_{\Qb}^n =
\left(\begin{array}{c}
S_\rho \\
\Sb_{\ub} \\
S_p \\
S_{\rho e} \\
S_{A_k} \\
S_{X_k} \\
S_{Y_k}
\end{array}\right)^n
=
\left(\begin{array}{c}
S_{{\rm ext},\rho} \\
\gb + \frac{1}{\rho}\Sb_{{\rm ext},\rho\ub} \\
\frac{1}{\rho}\frac{\partial p}{\partial e}S_{{\rm ext},\rho E} + \frac{\partial p}{\partial\rho}S_{{\rm ext}\rho} \\
\nabla\cdot\kappa\nabla T + S_{{\rm ext},\rho E} \\
\frac{1}{\rho}S_{{\rm ext},\rho A_k} \\
\frac{1}{\rho}S_{{\rm ext},\rho X_k} \\
\frac{1}{\rho}S_{{\rm ext},\rho Y_k}
\end{array}\right)^n,
\end{equation}
\begin{equation}
\Sb_{\Ub}^n =
\left(\begin{array}{c}
\Sb_{\rho\ub} \\
S_{\rho E} \\
S_{\rho A_k} \\
S_{\rho X_k} \\
S_{\rho Y_k}
\end{array}\right)^n
=
\left(\begin{array}{c}
\rho \gb + \Sb_{{\rm ext},\rho\ub} \\
\rho \ub \cdot \gb + \nabla\cdot\kappa\nabla T + S_{{\rm ext},\rho E} \\
S_{{\rm ext},\rho A_k} \\
S_{{\rm ext},\rho X_k} \\
S_{{\rm ext},\rho Y_k}
\end{array}\right)^n.
\end{equation}

\item[Step 3:] {\em Advect $\Delta t$.}

The goal is to advance
\begin{equation}
\Ub^{n+1} = \Ub^n - \dt\nabla\cdot\Fb^\nph + \dt\Sb^n.
\end{equation}
neglecting reaction terms.  Note that since the source term is not
time centered, this is not a second-order method.  After the advective
update, we correct the solution, effectively time-centering the source
term. The advection step is complicated, and more detail is given in
Section \ref{Sec:Advection Step}.  Here is the summarized version:
\begin{enumerate}
\item Compute primitive variables.
\item Predict primitive variables to time-centered edges.
\item Solve the Riemann problem.
\item Compute fluxes and update.
\end{enumerate}
\item[Step 4:] {\em Solve for updated gravity.}

Solve for gravity using:
\begin{equation}
\gb^{n+1} = \nabla\phi^{n+1}; \qquad \Delta\phi^{n+1} = -4\pi G\rho^{n+1},
\end{equation}
or use one of the simpler gravity types.
\item[Step 5:] {\em Correct solution with time-centered source terms.}

We need to correct the solution by effectively time-centering the
source terms.  These corrections are to be performed sequentially
since new source term evaluations may depend on previous corrections.

First, we correct the solution with the updated gravity:
\begin{eqnarray}
(\rho\ub)^{n+1} &=& (\rho\ub)^{n+1} + \frac{\dt}{2}\left[(\rho\gb)^{n+1} - (\rho\gb)^n\right], \\
(\rho E)^{n+1} &=& (\rho E)^{n+1} + \frac{\dt}{2}\left[\left(\rho\ub\cdot\gb\right)^{n+1} - \left(\rho\ub\cdot\gb\right)^n\right].
\end{eqnarray}

Next, we correct $\Ub$ with updated external sources.  For example,
for the momentum, we correct using
\begin{equation}
(\rho\ub)^{n+1} = (\rho\ub)^{n+1} + \frac{\dt}{2}\left(\Sb_{{\rm ext},\rho\ub}^{n+1} - \Sb_{{\rm ext},\rho\ub}^n\right).
\end{equation}
We correct $\rho E, \rho A_k, \rho X_k$, and $\rho Y_k$ in an
analogous manner.

Finally, we correct the solution with updated thermal diffusion using
\begin{equation}
(\rho E)^{n+1} = (\rho E)^{n+1} + \frac{\dt}{2}\left(\nabla\cdot\kappa\nabla T^{n+1} - \nabla\cdot\kappa\nabla T^n\right).
\end{equation}
\item[Step 6:] {\em React $\Delta t/2$.}

Update the solution due to the effect of reactions over half a time step.
\begin{eqnarray}
(\rho E)^{n+1} &=& (\rho E)^{n+1} - \frac{\dt}{2}\sum_k(\rho q_k\omegadot_k)^{n+1},\\
(\rho X_k)^{n+1} &=& (\rho X_k)^{n+1} + \frac{\dt}{2}(\rho\omegadot_k)^{n+1}.
\end{eqnarray}
\item[Step 7:] {\em Modify auxiliary variables.}

This is problem-dependent.  By default we treat the auxiliary
variables as advected quantities, so no additional steps are required.
\end{description}
Thus concludes the single-level algorithm description.


There are a number of parameters that affect which physics is turned on:
\begin{itemize}
\item {\tt castro.do\_hydro}
\item {\tt castro.do\_react}
\end{itemize}


\subsection{Castro::advance()}

\noindent

if (doReact)

\hspace{.1in}  strangChem()

end if

if (doGrav)

\hspace{.1in}  define oldGravityVector

end if

if (Diffusion)

\hspace{.1in}  getOldDiffusionTerm()

end if

if (addExtSource)

\hspace{.1in}  getSource() at old time

end if

AdvanceSolution()

if (doGrav)

\hspace{.1in}  define newGravityVector

\hspace{.1in}  correct solution due to new gravity

end if

if (addExtSource)

\hspace{.1in}  getSource() at new time

\hspace{.1in}  correct solution due to new source

end if

if (Diffusion)

\hspace{.1in}  getNewDiffusionTerm()

\hspace{.1in}  correct solution due to new diffusion term

\hspace{.1in}  computeTemp()

end if

if (doReact)

\hspace{.1in}  strangChem()

end if

if (advanceAux)

\hspace{.1in}  advanceAux()

end if

if (LevelSet)

\hspace{.1in}  advanceLevelSet()

end if

