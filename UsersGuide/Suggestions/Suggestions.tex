\begin{enumerate}

\item 3/4/16: The default version loaded of Python on Mira is not recent enough to support the Python scripts in our build system. Add {\tt +python} to your {\tt .soft} to fix this.

\item 3/1/16: The default Intel Compiler loaded on Cori (16.0.0.109)
  cannot compile BoxLib. To fix this, switch to an earlier version
  (e.g., at the terminal, do {\tt module swap intel intel/15.0.1.133}).

\item 3/1/16: The default version loaded of Python on Edison and Cori (2.6.9) is not recent enough to support the Python scripts in our build system. At the terminal, do {\tt module load python} to fix this.

\item 2/18/16: The default version loaded of Python on Titan (2.6.9) is not recent enough to support the Python scripts in our build system. At the terminal, do {\tt module load python} to fix this.

\item 2/12/09: If running on more than 64 processors on pleiades.ucsc.edu, you must set

amr.plot\_nfiles = NUMBER

amr.checkpoint\_nfiles = NUMBER

where NUMBER is a number greater than or equal to the number of processors (putting a large integer here is fine). The default value for plot\_nfiles and checkpoint\_nfiles is 64, so if you are running on 64 or fewer processors you are ok.

There is a problem with writing to the file system on pleiades. We have not yet been able to come up with a fix but this appears to be a good solution for now.

\item 2/2/09: Intel version 10.1.015 for Fortran has a known bug when run in non-DEBUG mode. We've tried the new version 11 as well and parts of the code no longer compile. So beware the Intel compilers. Best suggestion at this point is to return to version 9.

\item 1/25/09: on franklin.nersc.gov, the PathScale compiler with full optimization appears to have bug. Please use PGI for now, or reduce the optimization from -O to -O1 with PathScale.

\item 1/25/09: Very rough estimate on franklin-- for a code with 1 species, one can run a $256^3$ calculation on 16 processors but not 8. Keep this in mind as you choose the number of processors for a given job. Suggested settings for large 3-d runs: blocking\_factor = 16, max\_grid\_size = 32. 
\end{enumerate}

\newpage
\section{Compilers}

This is a brief collection of our experiences with compilers on various machines.  This is {\em not} an exhaustive list.

\subsection{Those that compile...}

\begin{table*}[!h]
%\begin{scriptsize}
\begin{tabular}{|l|l|l|l|} \hline
Machine & Compiler & Modules & Remarks \\
\hline
{\bf hopper} & {\tt PathScale} & PrgEnv-pathscale/3.1.61(default) & Date: 1/12/12\\
& & pathscale/4.0.9 & \\
\cline{2-4}
& {\tt PGI} & PrgEnv-pgi/3.1.61(default) & Date: 1/12/12\\
& & pgi/11.7.0(default) & \\
\hline
{\bf titan \& blue waters} & {\tt Cray} & PrgEnv-craye/4.1.20(default) & Date: 1/23/13\\
& & cce/8.1.2 & \\
\hline
\end{tabular}
\label{Table:Those-that-compile}
%\end{scriptsize}
\end{table*}
{\em Note: Cray compilers version 8.1.2 need the } {\bf -hnopgas\_runtime} {\em flag to be set.}

\subsection{Those that {\em don't} compile...}
\begin{table*}[!h]
%\begin{scriptsize}
\begin{tabular}{|l|l|l|l|} \hline
Machine & Compiler & Modules & Remarks \\
\hline
{\bf hopper} & {\tt PathScale} & PrgEnv-pathscale/3.1.61(default) & Date: 1/12/12\\
& & pathscale/3.2.99(default) & \\
\hline
{\bf titan \& blue waters} & {\tt Cray} & PrgEnv-cray/4.1.20(default) & \\
& & cce/8.1.1(default) & Date: 1/23/13\\
\hline
\end{tabular}
\label{Table:Those-that-dont-compile}
%\end{scriptsize}
\end{table*}
