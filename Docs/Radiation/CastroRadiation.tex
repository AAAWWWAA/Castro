\section{Introduction}

\castro\ has three radiation solvers: 
\begin{itemize}
\item {\tt SingleGroupSolver}: this solver does not have radiation
  pressre.  It is pure hydro plus radiation diffusion.  This is only
  applicable when the medium is optically thick and the pressure is small.

\item {\tt SGFLDSolver}: this is the gray flux-limited diffusion
  radiation hydrodymamics solver.  Here the radiation pressure is
  separate from the gas pressure, and both pressures participate in
  the Riemann solver.

\item {\tt MGFLDSolver}: this is the multigroup flux-limited diffusion
  radiation hydrodynamics solver.  As with the gray solver, radiation
  pressure contributes to the pressure in the Riemann solver.  Here a
  number of energy groups are used to represent the radiation field,
  and the opacity can be frequency-dependent.

\end{itemize}

The gray solver has a comoving frame mode and a mixed frame mode,
whereas the MG solver uses the comoving frame approach.  More details
about the formulation and algorithm can be found in the series of
\castro\ papers.

\section{Getting Started}

\subsection{Getting the Code}

The \castro\ radiation solver is part of the main \castro\ git repo,
so you already have all the \castro\ code and problem setups
to exercise radiation.  The only other requirement is a copy
of the \hypre\ library. \hypre\ provides the algebraic multigrid
solvers used by the implicit radiation update.  You can get
a copy at \url{https://computation.llnl.gov/casc/linear_solvers/sls_hypre.html}.  You will need to follow their installation instructions.

In addition to the environment variables you set for the main
\castro\ hydrodynamics problems, you also need to tell the code
where to find \hypre.  This is done via one of two variables:
\begin{itemize}
\item the environment variable {\tt HYPRE\_DIR} should
  point to the location of your Hypre installation
  (e.g., {\tt\seqsplit{/home/J.Doe/hypre}}) or
  can be set directly in 
  {\tt\seqsplit{CastroRadiation/Exec/Make.Castro}}.
  This applies is you build with {\tt USE\_OMP=FALSE}

\item the variable {\tt HYPRE\_OMP\_DIR} should be set (either as an
  environment variable or in
  {\tt\seqsplit{Castro/Exec/Make.Castro}}) to the directory
  for openmp enabled Hypre (e.g.,
  {\tt\seqsplit{/home/J.Doe/hypre-omp}}) if you build with {\tt
    USE\_OMP=TRUE}
\end{itemize}

Now go to a ``run'' directory, say
{\tt\seqsplit{Castro/Exec/radiation\_tests/RadThermalWave}},
edit the file {\tt GNUmakefile}, and set
\begin{itemize}
\item {\tt COMP} = your favorite compiler suite (e.g., {\tt gnu}, {\tt pgi}, {\tt intel})
\item {\tt DIM}   = 1 or 2 or 3
\item \makevar{USE\_RAD} {\tt = TRUE}---this is important.  This tells the build system to
  compile in, and link the the radiation code.
\end{itemize}
Then type {\tt make} to generate an executable file.

\section{Microphysics: EOS, Network, and Opacity}

\subsection{EOS}

\castro\ provides several types of equation of state (EOS), including
gamma-law and Helmholtz.  To use the gamma-law EOS, set
\begin{verbatim}
EOS_DIR := gamma_law_general
\end{verbatim}
in the {\tt GNUmakefile}.

The original Helmholtz EOS for stellar interiors includes a radiation
contribution.  However, for radiation hydrodynamics calculations, the
radition contribution should be taken out of EOS because radiation has
been treated in other places.  To use Helmholtz EOS, we will use the
version in \microphysics, as with the pure hydrodynamics code, but
this will interpret the \ifdef{RADIATION} preprocessor variable and
disable the radiation portion of the EOS\footnote{at the moment, we
  don't have a way to allow for the EOS to provide radiation pressure
  if the \castro\ radiation is used solely for neutrinos, but this is
  something that could be added easily.}  If you have your own EOS, you
can put it in \microphysics.


\subsubsection{EOS Parameters}

The following parameters affect how the radiation solver used the EOS:
\begin{itemize}
  \item {\tt radiation.do\_real\_eos = 1}

    Usually you do not want to change this from the default.  Setting
    this to {\tt 0} is only for contrived tests that assume the
    specific heat is in the form of a power-law,
    \begin{equation}
      c_v = \mathrm{const}\ \rho^m T^{-n}
    \end{equation}

\end{itemize}
  
  

\subsection{Network}

The radiation solver uses the same networks as we saw for pure hydro,
so nothing needs to change here.  Again, if you are not modeling
reactions, then the {\tt general\_null} network can be used to define
the appropriate composition for your problem.

\subsection{Opacity}

By default, we assume that
\begin{equation}
  \kappa = \mathrm{const}\ \rho^{m} T^{-n} \nu^{p} , \label{eq:kappa}
\end{equation}
where $\kappa$ is either Planck or Rosseland mean absorption
coefficients, $\rho$ is density, $T$ is temperature, $\nu$ is
frequency, and $m$, $n$ and $p$ are constants.  For the gray solver,
$p = 0$.  If Equation~(\ref{eq:kappa}) is sufficient, set
\begin{verbatim}
Opacity_dir := null
\end{verbatim}
in {\tt GNUmakefile}.  Otherwise, put your own opacity in
{\tt\seqsplit{Castro/Microphysics/opacity}} and set
the input parameter, {\tt radiation.use\_opacity\_table\_module = 1} (see
\S~\ref{sec:opacpars}).

Some notes:
\begin{itemize}
\item Here, $\kappa$ has units of $\mathrm{cm}^{-1}$.  Some papers or
  texts may instead have an implicit density factor in $\kappa$,
  yielding units $\mathrm{cm}^2~\mathrm{g}^{-1}$.

\item \castro\ allows for two temperatures (different radiation and gas
  temperature, so $E_\mathrm{r} \ne a T_\mathrm{gas}^4$).
  Correspondingly,  \castro\ cares about both the Planck mean,
  $\kappa_P$, and Rosseland mean, $\kappa_R$, opacities---these have
  different weightings.

  If we set $\kappa_P \Delta x \gg 1$ ($\kappa_P$ is really large),
  then the two temperatures become the same.

  If we set $\kappa_P = \kappa_R$, then we can see how different the
  two temperature are.

  In an optically thick medium, we would not expect the two temperatures
  to be very different.

\end{itemize}


\subsubsection{Opacity Parameters}
\label{sec:opacpars}

The parameters describing the opacity include:
\begin{itemize}

\item {\tt radiation.use\_opacity\_table\_module = 0}
  
  For neutrino problems, this parameter is not ignored.  For photon
  problems, this determines whether the opacity module at {\tt
    Opacity\_dir} (which is set in {\tt GNUmakefile}) will be used to
  compute opacities.  If this is set to 1, the following parameters
  for opacities will be ignored.

\item For the Planck mean opacity of the form in Eq.~(\ref{eq:kappa}),
  the following parameters set the coefficient and exponents:
  \begin{itemize}
  \item {\tt radiation.const\_kappa\_p = -1.0}
  \item {\tt radiation.kappa\_p\_exp\_m = 0.0}
  \item {\tt radiation.kappa\_p\_exp\_n = 0.0}
  \item {\tt radiation.kappa\_p\_exp\_p = 0.0}
  \end{itemize}

\item For the Rosseland mean opacity of the form in Eq.~(\ref{eq:kappa}),
  the following parameters set the coefficient and exponents:
  \begin{itemize}
  \item {\tt radiation.const\_kappa\_r = -1.0}
  \item {\tt radiation.kappa\_r\_exp\_m = 0.0}
  \item {\tt radiation.kappa\_r\_exp\_n = 0.0}
  \item {\tt radiation.kappa\_r\_exp\_p = 0.0}
  \end{itemize}
  
\item For the scattering coefficient of the form in Eq.~(\ref{eq:kappa}),
  the following parameters set the coefficient and exponents:
  \begin{itemize}
  \item {\tt radiation.const\_scattering = 0.0}
  \item {\tt radiation.scattering\_exp\_m = 0.0}
  \item {\tt radiation.scattering\_exp\_n = 0.0}
  \item {\tt radiation.scattering\_exp\_p = 0.0}
  \end{itemize}

\item {\tt radiation.kappa\_r\_floor = 0.0}

  Floor for Rosseland mean.

\item{\tt radiation.do\_kappa\_stm\_emission = 0}

  If it is 1, correction for stimulated emission is applied to Planck mean as
  follows
  \begin{equation}
    \kappa = \mathrm{const}\ \rho^{m} T^{-n} \nu^{p}
    \left [1-\exp{\left (-\frac{h\nu}{k T} \right )} \right ].
  \end{equation}

\item {\tt radiation.surface\_average = 2}

 How the averaging of opacity is done from faces to center for
 the radiation solver.  {\tt 0} is arithmetic averaging, {\tt 1}
 is harmonic averaging, and {\tt 2} is a combination of the two.
 This is implemented in {\tt RAD\_?D.F} in {\tt kavg}.


\end{itemize}

\noindent Note that the unit for opacities is $\mathrm{cm}^{-1}$.  For
the gray solver, the total opacity in the diffusion coefficient is the sum
of {\tt kappa\_r} and {\tt scattering}, whereas for the MG solver,
there are two possibilities.  If {\tt const\_kappa\_r} is greater than
0, then the total opacity is set by {\tt kappa\_r} alone, otherwise
the total opacity is the sum of {\tt kappa\_p} and {\tt scattering}.


\section{Radiation Solver Physics}

In this section, we list some radiation related parameters that you
can set in an {\tt inputs} file.  Here are some important parameters:
\begin{itemize}
\item {\tt radiation.SolverType}:

  Set it to {\tt 5} for the gray solver, and {\tt 6} for the MG solver.

\item {\tt castro.do\_hydro}

  Usually you want to set it to {\tt 1}.  If it is set to {\tt 0},
  hydro will be turned off, and the calculation will only solve
  radiation diffusion equation.

\item {\tt castro.do\_radiation}

  If it is {\tt 0}, the calculation will be pure hydro.
\end{itemize}

Below are more parameters.  For each parameter, the default value is
on the right-hand side of the equal sign.


\subsection{Verbosity and I/O}
\label{sec:bothpar}

\begin{itemize}
\item {\tt radiation.v = 0}
  
  Verbosity

\item {\tt radiation.verbose = 0}
  
  Verbosity

\item {\tt radiation.plot\_lambda = 0}
  
  If {\tt 1}, save flux limiter in plotfiles.

\item {\tt radiation.plot\_kappa\_p = 0}

  If {\tt 1}, save Planck mean opacity in plotfiles.

\item {\tt radiation.plot\_kappa\_r = 0}

  If {\tt 1}, save Rosseland mean opacity in plotfiles.

\item {\tt radiation.plot\_lab\_Er = 0}
  
  If {\tt 1}, save lab frame radiation energy density in plotfiles.
  This flag is ignored when the mixed-frame gray solver is used.

\item {\tt radiation.plot\_com\_flux = 0}
  
  If {\tt 1}, save comoving frame radiation flux in plotfiles.

\item {\tt radiation.plot\_lab\_flux = 0}

  If {\tt 1}, save lab frame radiation flux in plotfiles.

\end{itemize}


\subsection{Flux Limiter and Closure}

\label{sec:fluxlimiter}

\begin{itemize}

\item {\tt radiation.limiter = 2}

  Possible values are:
  \begin{itemize}
  \item {\tt ~0}: No flux limiter

  \item {\tt ~2}: Approximate limiter of Levermore \& Pomraning

  \item {\tt 12}: Bruenn's limiter

  \item {\tt 22}: Larsen's square root limiter

  \item {\tt 32}: Minerbo's limiter
  \end{itemize}

\item {\tt radiation.closure = 3}

  Possible values are:
  \begin{itemize}
  \item {\tt 0}: $f = \lambda$, where $f$ is the scalar Eddington factor
    and $\lambda$ is the flux limiter.

  \item {\tt 1}: $f = \frac{1}{3}$
  \item {\tt 2}: $f = 1 - 2 \lambda$
  \item {\tt 3}: $f = \lambda + (\lambda R)^2$, where $R$ is the radiation
    Knudsen number.
  \item {\tt 4}: $f = \frac{1}{3} + \frac{2}{3} (\frac{F}{cE})^2$, where
      $F$ is the radiation flux, $E$ is the radiation energy density,
      and $c$ is the speed of light.
  \end{itemize}
\end{itemize}

Note the behavior of the radiative flux in the optically thin and
optically thick limits.  The flux limiter, $\lambda = \lambda(R)$,
where
\begin{equation}
  R = \frac{|\nabla E_r^{(0)}|}{\chi_R E_r^{(0)}}
\end{equation}
Regardless of the limiter chosen, when we are optically thick,
$\chi_R \rightarrow \infty$, $R \rightarrow 0$, and $\lambda \rightarrow 1/3$.
The radiative flux then becomes
\begin{equation}
  F_r^{(0)} = -\frac{c\lambda}{\chi_R} \nabla E_r^{(0)} \rightarrow
  \frac{1}{3} \frac{c}{\chi_R} \nabla E_r^{(0)}
\end{equation}
And when we are optically thin, $\chi_R \rightarrow 0$, $R \rightarrow \infty$,
and $\lambda \rightarrow 1/R = \chi_R E_r^{(0)}/{|\nabla E_r^{0}|}$, and
the radiative flux then becomes
\begin{equation}
  F_r^{(0)} = -\frac{c\lambda}{\chi_R} \nabla E_r^{(0)} \rightarrow
  -\frac{c}{\chi_R}\frac{\chi_R E_r^{(0)}}{|\nabla E_r^{0}|}
    \nabla E_r^{(0)} = -c E_r^{0}
\end{equation}
See Krumholz et al.\ 2007 for some discussion on this.
\subsection{Boundary Conditions}

\castro\ needs to know about the boundary conditions for both
the hydrodynamics and radiation portions of the evolution.

\subsubsection{Hydrodynamics Evolution}

For the hydrodynamics portion of the solve, the boundary conditions
for the normal hydrodynamic state values will be set by the problem's
{\tt hypfill} routine (which typically just calls {\tt filcc} to handle
the usual hydrodynamics boundary types: outflow, symmetry, etc.).

A corresponding {\tt radfill} routine needs to be written to fill the
ghost cells for the radiation energy density during the hydrodynamics
evolution.  Again, this usually will just default to calling {\tt
  filcc}.

Note: if any of the hydrodynamic boundary conditions types are set
to {\tt Inflow}, then you will need to ensure that the {\tt radfill}
routine explicitly handles the boundary condition implementation
for the radiation energy density in that case---the {\tt filcc}
routine will not do a hydrodynamic {\tt Inflow} boundary.


\subsubsection{Radiation Evolution}

The following parameters are for radiation boundary in the diffusion
equation. They do not affect hydrodynamic boundaries.
\begin{itemize}
\item {\tt radiation.lo\_bc}
  
  This sets the action to take at the lower edge of the domain in
  each coordinate direction.  Possible values are:
  \begin{itemize}

  \item {\tt 101} {\em Dirchlet}:

    Specify the radiation energy density on the boundary.
    For gray radiation, this could be $E_r = a T^4$.

    For multigroup radiation,  \castro\ stores the energy density as
    $\mathrm{erg}~\mathrm{cm}^{-3}$, so the total radiation energy
    can be found by simply summing over the groups.  So if you want
    to set the radiation BCs using the Planck function, you simply
    multiply by the group width---see {\tt Exec/radiation\_tests/RadSphere/Tools/radbc.f90}
    for an example.

  \item {\tt 102} {\em Neumann}:

    Here, you specify the radiation flux on the boundary.  For gray
    radiation, this is the expression given in the gray \castro\ paper
    (Eq. 7, 8),
    \begin{equation}
      F_r = - \frac{c\lambda}{\kappa_R} \nabla E_r
    \end{equation}
    where $\lambda$ is the flux limiter.

    Note that if your boundary represents an incoming flux through
    a vacuum (like stellar irradiation), then $\kappa \rightarrow 0$, leaving
    \begin{equation}
      F_r = -c E_r
    \end{equation}
    (see \S~\ref{sec:fluxlimiter}) in that case.  

  \item {\tt 104} {\em Marshak} (vacuum):
      
    Here, you specify the incident flux and the outside is a vacuum.
    This differs from the Neumann condition because there is also a
    flux coming from inside, for the net flux across the boundary is
    different than the incident flux.

  \item {\tt 105} {\em Sanchez-Pomraning}:

    This is a modified form of the Marshak boundary condition that works with FLD.
    This is like the Marshak condition, but $\lambda = 1/3$ is not assumed inside
    the boundary (optical thickness).

  \end{itemize}
  
\item {\tt radiation.hi\_bc}
  
  See {\tt radiation.lo\_bc}.

\item {\tt radiation.lo\_bcflag = 0 0 0}
  
  If it is 0, {\tt bcval} is used for that dimension, otherwise
  subroutine {\tt rbndry} in {\tt RadBndry\_1d.f90} is called to set
  boundary conditions.

\item {\tt radiation.hi\_bcflag = 0 0 0}
  
  See {\tt radiation.lo\_bcflag}

\item {\tt radiation.lo\_bcval = 0.0 0.0 0.0}
  
  The actual value to impose for the boundary condition type set by
  {\tt radiation.lo\_bc}.  This parameter is interpreted differently
  depending on the boundary condition:
  \begin{itemize}
    \item Dirchlet: Dirichlet value of rad energy density
    \item Neumann:  inward flux of rad energy
    \item Marshak:  incident flux
    \item Sanchez-Pomraning: incident flux
  \end{itemize}

\item {\tt radiation.hi\_bcval = 0.0 0.0 0.0}
  
  See {\tt radiation.lo\_bcval}
\end{itemize}


\subsection{Convergence}

For the gray solver, there is only one iteration in the scheme,
whereas for the MG solver, there are two iterations with an inner
iteration embedded inside an outer iteration.  In the following, the
iteration in the gray solver will also be referred as the outer
iteration for convenience.  The parameters for the inner iteration are
irrelevant to the gray solver.

\begin{description}
\item[radiation.maxiter = 50] \hfill \\
  Maximal number of outer iteration steps.
\item[radiation.miniter = 1] \hfill \\
  Minimal number of outer iteration steps.
\item[radiation.reltol = 1.e-6] \hfill \\
  Relative tolerance for the outer iteration.
\item[radiation.abstol = 0.0] \hfill \\
  Absolute tolerance for the outer iteration.
\item[radiation.maxInIter = 30] \hfill \\
  Maximal number of inner iteration steps.
\item[radiation.minInIter = 1] \hfill \\
  Minimal number of inner iteration steps.
\item[radiation.relInTol = 1.e-4] \hfill \\
  Relative tolerance for the inner iteration.
\item[radiation.absInTol = 0.0] \hfill \\
  Absolute tolerance for the inner iteration.
\item[radiation.convergence\_check\_type = 0] \hfill \\
  For the MG solver only.  This specifiy the way of checking the
  convergence of an outer iteration.  Possible values are
  \begin{itemize}
    \item 0: Check $T$, $Y_e$, and the residues of the equations for
      $\rho e$ and $\rho Y_e$
    \item 1: Check $\rho e$
    \item 2: Check the residues of the equations for $\rho e$ and $\rho Y_e$
    \item 3: Check $T$ and $Y_e$
  \end{itemize}
\end{description}


\subsection{Parameters for Gray Solver}
\label{sec:graypar}

\begin{description}
\item[radiation.comoving = 1] \hfill \\
  Do we use the comoving frame approach?
\item[radiation.Er\_Lorentz\_term = 1] \hfill \\
  If the mixed-frame approach is taken, this parameter decides whether
  Lorentz transformation terms are retained.
\item[radiation.delta\_temp = 1.0] \hfill \\
  This is used in computing numerical derivativas with respect to $T$.
  So it should be a small number compared with $T$, but not too small.
\item[radiation.update\_limiter = 1000] \hfill \\
  Stop updating flux limiter after {\tt update\_limiter} iteration steps.
\item[radiation.update\_planck = 1000] \hfill \\
  Stop updating Planck mean opacity after {\tt update\_planck} iteration steps.
\item[radiation.update\_rosseland = 1000] \hfill \\
  Stop updating Rosseland mean opacity after {\tt update\_rosseland} iteration steps.
\end{description}

\subsection{Grouping in the MG Solver}

We provide two methods of setting up groups based upon logarithmic
spacing.  In both methods, you must provide:
\begin{description}
\item[radiation.nGroups] \hfill \\
  Number of groups.
\item[radiation.lowestGroupHz] \hfill \\
  Frequency of the lower bound for the first group.
\end{description}

In addition, if the parameter {\tt groupGrowFactor} is provided, then
the first method will be used, otherwise the second method will be
used.  In the first way, you must also provide {\tt firstGroupWidthHz}
(the width of the first group).  The width of other groups is set to
be {\tt groupGrowFactor} times the width of its immediately preceding
group.  In the second way, you must provide {\tt highestGroupHz} as
the upper bound of the last group.  It should be noted that {\tt
  lowestGroupHz} can be 0 in the first method, but not the second
method.  However, when we compute the group-integrated Planck
function, the lower bound for the first group and the upper bound for
the last group are assumed to be 0 and $\infty$, respectively.

\subsection{Parameters for MG Solver}
\label{sec:mgpar}

\begin{description}
\item[radiation.delta\_e\_rat\_dt\_tol = 100.0] \hfill \\
  Maximally allowed relative change in $e$ during one time step.
\item[radiation.delta\_T\_rat\_dt\_tol = 100.0] \hfill \\
  Maximally allowed relative change in $T$ during one time step.
\item[radiation.delta\_Ye\_dt\_tol = 100.0] \hfill \\
  Maximally allowed absolute change in $Y_e$ during one tim estep.
\item[radiation.fspace\_advection\_type = 2] \hfill \\
  Possible value is 1 or 2.  The latter is better.
\item[radiation.integrate\_Planck = 1] \hfill \\
  If 1, integrate Planck function for each group.  For the first
  group, the lower bound in the integration is assumed to be 0 no
  matter what the grouping is.  For the last group, the upper bound in
  the integration is assumed to be $\infty$.
\item[radiation.matter\_update\_type = 0] \hfill \\
  How to update matter.  0 is proabaly the best.
\item[radiation.accelerate = 2] \hfill \\
  The inner iteration of the MG solver usually requires an
  acceleration scheme.  Choices are
  \begin{itemize}
    \item 0: No acceleration
    \item 1: Local acceleration
    \item 2: Gray acceleration
  \end{itemize}
\item[radiation.skipAccelAllowed = 0] \hfill \\
  If it is set to 1, skip acceleration if it does not help.
\item[radiation.n\_bisect = 1000] \hfill \\
  Do bisection for the outer iteration after {\tt n\_bisec} iteration steps.
\item[radiation.use\_dkdT = 1] \hfill \\
  If it is 1, $\frac{\partial \kappa}{\partial T}$ is retained in the
  Jacobi matrix for the outer (Newton) iteration.
\item[radiation.update\_opacity = 1000] \hfill \\
  Stop updating opacities after {\tt update\_opacity} outer iteration steps.
\item[radiation.inner\_update\_limiter = 0] \hfill \\
  Stop updating flux limiter after {\tt inner\_update\_limiter} inner
  iteration steps.  If it is 0, the limiter is lagged by one outer
  iteration.  If it is -1, the limiter is lagged by one time step.  If
  the inner iteration has difficulty in converging, setting this
  parameter it to -1 can help.  Since the flux limiter is only a
  kludge, it is justified to lag it.
\end{description}


\subsection{Linear System Solver}
\label{sec:hypre}

There are a number of choices for the linear system solver.  The
performance of the solvers usually depends on problems and the
computer.  So it is worth trying a few solvers to find out which one
is best for your problem and computer.

\begin{description}
\item \runparam{radsolve.level\_solver\_flag}: the linear solver
  in \hypre\ to use.  The available choices are:
  \begin{itemize}
    \item {\tt 0}: SMG
    \item {\tt 1}: PFMG  ($\ge$ 2-d only)
    \item {\tt 100}: AMG using ParCSR ObjectType
    \item {\tt 102}: GMRES using ParCSR ObjectType
    \item {\tt 103}: GMRES using SStruct ObjectType
    \item {\tt 104}: GMRES using AMG as preconditioner
    \item {\tt 109}: GMRES using Struct SMG/PFMG as preconditioner
    \item {\tt 150}: AMG using ParCSR ObjectType
    \item {\tt 1002}: PCG using ParCSR ObjectType
    \item {\tt 1003}: PCG using SStruct ObjectType
  \end{itemize}

  As a general rule, the SMG is the most stable solver, but is usually
  the slowest.  The asymmetry in the linear system comes from the
  adaptive mesh, so the PFMG should be your first choice.  Note: in
  you cannot use PFMG.

  Setting this to 109 (GMRES using Struct SMG/PFMG as preconditioner)
  should work reasonably well for most problems.

\item \runparam{radsolve.maxiter} (default: {\tt 40}): 
  Maximal number of iteration in Hypre.

\item \runparam{radsolve.reltol} (default: {\tt 1.e-10}):
  Relative tolerance in Hypre

\item \runparam{radsolve.abstol} (default: {\tt 0}):
  Absolute tolerance in Hypre

\item \runparam{radsolve.v} (default: {\tt 0}):
  Verbosity

\item \runparam{radsolve.verbos} (default: {\tt 0}):
  Verbosity

\item \runparam{habec.verbose} (default: {\tt 0}):
  Verbosity for {\tt level\_solver\_flag} $<$ 100

\item \runparam{hmabec.verbose} (default: {\tt 0}):
  Verbosity for {\tt level\_solver\_flag} $>=$ 100
\end{description}


\section{Output}

\subsection{Gray Solver}

For the gray radiation solver, the radiation energy density is stored in plotfiles
as {\tt rad}.  Note that this quantity has units of $\mathrm{erg~cm^{-3}}$, which
is different that the specify internal energy of the gas $\mathrm{erg~g^{-1}}$.


