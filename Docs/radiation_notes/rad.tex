\documentclass[12pt]{article}

\usepackage[margin=1in]{geometry}

\usepackage{amsmath}
\usepackage{amssymb}

\usepackage{hyperref}

\usepackage{mathpazo}

\newcommand{\nup}{{(\nu)}}
\newcommand{\evm}{{(-)}}
\newcommand{\evz}{{(0)}}
\newcommand{\evp}{{(+)}}

\begin{document}

\begin{center}
{\Large Treatment of gas pressure interface state in Castro + radiation}
\end{center}

\subsection*{Background}

The interface states for radiation work in the primitive variable
system, $q = (\rho, u, p, (\rho e)_g, E_r)^\intercal$, where $p$ is
the gas pressure only, and $(\rho e)_g$ is the gas energy density.

Written in the form:
\begin{equation}
q_t + A(q) q_x = 0
\end{equation}
The matrix $A$ takes the form:
\begin{equation}
A = \left (
\begin{matrix}
u & \rho & 0 & 0 & 0\\
0 & u & {1}/{\rho} & 0 & \lambda_f/{\rho}\\
0 & c_{g}^{2} \rho & u & 0 & 0\\
0 & h_{g} \rho & 0 & u & 0\\
0 & E_{r} \left(\lambda_f + 1\right) & 0 & 0 & u
\end{matrix}\right )
\end{equation}
here, $c_g$ is the gas sound speed and $h_g = e_g + p/\rho$ is the 
gas specific enthalpy.
A has eigenvalues, $\lambda^\evm = u - c$, $\lambda^\evz = u$, 
$\lambda^\evp = u + c$, where the total sound speed, $c$ is related
to the gas sound speed, $c_g$, as:
\begin{equation}
c^2 = c_g^2 + (\lambda_f + 1)\frac{\lambda_f E_r}{\rho}
\end{equation}
In constructing the interface states, start with a reference state,
$q_\mathrm{ref}$ and define the jumps carried by each wave as the
integral under the parabolic profile with respect to this reference
state:
\begin{equation}
\Delta q^\nup \equiv q_\mathrm{ref} - \mathcal{I}^\nup(q)
\end{equation}
and then the interface states are:
\begin{equation}
q_\mathrm{int} = q_\mathrm{ref} - \sum_\nu (l^\nup \cdot \Delta q^\nup) r^\nup
\end{equation}
Defining:
\begin{align}
\beta^\evm = ( l^\evm \cdot \Delta q^\evm ) &= \frac{1}{2 c^{2}} \left(
    \Delta E_r^\evm \lambda_f + \Delta p^\evm - \Delta u^\evm c \rho\right)\\
&= \frac{1}{2 c^{2}} \left(
    \Delta p_\mathrm{tot}^\evm - \Delta u^\evm c \rho\right)
\end{align}
where we recognized that $p_\mathrm{tot} = p + \lambda_f E_r$.
Similarly, we have:
\begin{align}
\beta^\evp = ( l^\evp \cdot \Delta q^\evp ) &= \frac{1}{2 c^{2}} \left(
    \Delta E_r^\evp \lambda_f + \Delta p^\evp + \Delta u^\evp c \rho\right)\\
           &= \frac{1}{2 c^{2}} \left(
    \Delta p_\mathrm{tot}^\evp + \Delta u^\evp c \rho\right)
\end{align}
and for the 0-wave, we have three degenerate eigenvalues and
eigenvectors.  We label these with the subscripts $\rho$, $\rho e_g$,
and $E_r$, indicating which of the corresponding components in our
primitive variable vector, $q$, is non-zero in the right eigenvector,
$r^\nup$.  Then we have:
\begin{align}
\beta^\evz_\rho &= 
    \Delta\rho^\evz  - \frac{\Delta p^\evz_\mathrm{tot}}{c^2} \\
%
\beta^\evz_{{\rho e}_g} &= \Delta(\rho e)^\evz_g - \frac{\Delta p_\mathrm{tot}^\evz}{c^2} h_g \\
%
\beta^\evz_{E_r} &= \Delta E_r^\evz - \frac{\Delta p_\mathrm{tot}^\evz}{c^2} h_r
\end{align}
where $h_r = (\lambda_f + 1)E_r/\rho$.  Note, these match the derivation
done in the Jupyter/SymPy notebook:\newline
{\footnotesize \url{https://github.com/zingale/hydro_examples/blob/master/compressible/euler-generaleos.ipynb}}

The gas pressure update in these terms is:
\begin{equation}
p_\mathrm{int} = p_\mathrm{ref} - (\beta^\evp + \beta^\evm) c_g^2 + \lambda_f \beta^\evz_{E_r}
\end{equation}
This matches the expression in Castro.  Notice that this expression is
unusual for pressure, since it jumps across the $\evz$-wave, whereas
the total pressure would not.  

Castro computes the edge state of the radiation energy density as:
\begin{equation}
{E_r}_\mathrm{int} = {E_r}_\mathrm{ref} - (\beta^\evp + \beta^\evm) h_r - \beta^\evz_{E_r}
\end{equation}
and we see that the total pressure can be constructued as
${p_\mathrm{tot}}_\mathrm{int} = p_\mathrm{int} + \lambda_f
{E_r}_\mathrm{int}$, giving:
\begin{equation}
{p_\mathrm{tot}}_\mathrm{int} = {p_\mathrm{tot}}_\mathrm{ref} -
   (\beta^\evp + \beta^\evm) c^2
\end{equation}
This looks, as it should, analogous to the pure hydrodynamics case.

\subsection*{The Interface States}

What is the interface state? for this we need to choose a reference
state.  The choice of reference state should not matter if we counted
the contributions of all waves, then:
\begin{equation}
q_\mathrm{int} = q_\mathrm{ref} - \sum_\nu [ l^\nup \cdot (q_\mathrm{ref} - \mathcal{I}^\nup(q)) ] r^\nup
\end{equation}
and $q_\mathrm{ref}$ cancels out.  However, when we do characteristic
tracing, this is not guaranteed.

In Castro, the quantity $\Delta p^\evz_\mathrm{tot}$ is defined as:
\begin{equation}
\Delta p^\evz_\mathrm{tot} =
  p_\mathrm{tot,ref} - \mathcal{I}^\evz(p_\mathrm{tot})
\end{equation}
and we adopt the common strategy of picking the reference state to be
the $\mathcal{I}$ corresponding to the fastest wave moving toward the
interface.

Consider a zone, $i$, and tracing to the right edge of the zone to
form the interface state, $i+1/2,L$, where $L$ indicates that it is
immediately to the left of the $i+1/2$ interface.  The fastest wave
that can potentially move toward that interface is the $\evp$ wave,
so we pick:
\begin{equation}
q_\mathrm{ref} = \left (
   \begin{array}{c}
     \mathcal{I}^\evp(\rho) \\
     \mathcal{I}^\evp(u) \\
     \mathcal{I}^\evp(p_\mathrm{tot}) \\
     \mathcal{I}^\evp((\rho e)_g) \\
     \mathcal{I}^\evp(E_r)
   \end{array}
\right )
\end{equation}
Looking at the gas pressure interface state, we have:
\begin{equation}
p_\mathrm{int} = p_\mathrm{ref} - \frac{1}{2} \frac{c_g^2}{c^2} \left \{
   \left [ \Delta p_\mathrm{tot}^\evp + \Delta u^\evp c\rho \right ]
 + \left [ \Delta p_\mathrm{tot}^\evm - \Delta u^\evm c\rho \right ] \right \}
 + \lambda_f \left [ \Delta E_r^\evz - \frac{h_r}{c^2} \Delta p^\evz_\mathrm{tot} \right ]
\end{equation}
Substituting in our choice of reference state, we have:
\begin{align}
p_\mathrm{int} = \mathcal{I}^\evp(p) &- \underbrace{\frac{1}{2} \frac{c_g^2}{c^2} \left \{
 \left [ \mathcal{I}^\evp(p_\mathrm{tot}) - \mathcal{I}^\evm(p_\mathrm{tot}) \right ] 
   - \rho c \left [ \mathcal{I}^\evp(u) - \mathcal{I}^\evm(u) \right ] \right \}}_{\mbox{\footnotesize carried by the $\evm$ wave}} \\
 &+ \underbrace{\lambda_f \left \{ \left [ \mathcal{I}^\evp(E_r) - \mathcal{I}^\evz(E_r) \right ]
   - \frac{h_r}{c^2} \left [ \mathcal{I}^\evp(p_\mathrm{tot}) - \mathcal{I}^\evz(p_\mathrm{tot}) \right ] \right \}}_{\mbox{\footnotesize carried by the $\evz$ wave}}
\end{align}
We see that the expression for
$p_\mathrm{int}$ starts with the gas pressure.  In the 
event that no other waves are moving toward the interface, then we find:
\begin{equation}
p_\mathrm{int} \rightarrow \mathcal{I}^\evp(p)
\end{equation}
(since in our algorithms, we set the $\beta$'s to $0$ if those waves
are not moving toward the interface.

\subsection*{Alternative?}

A, perhaps more consistent, way to handle this is to predict $p_\mathrm{tot}$
and $E_r$ to the interfaces.  This is consistent with our choice of
reference state, and using $p_\mathrm{tot}$ is more analogous to the
pure hydrodynamics case.  We then algebraically construct the
gas-pressure interface state as:
\begin{equation}
p_\mathrm{int} = {p_\mathrm{tot}}_\mathrm{int} - \lambda_f {E_r}_\mathrm{int}
\end{equation}

Note that this extends naturally to multigroup radiation as well, simply by
summing up the independent $E_r$.

\subsection*{$\gamma_e$ system}

The alternate approach that Castro takes to incorporate auxillary
thermodynamic information into the system is to evolve an equation for
$\gamma_e$.  We use the primitive variable system, $q = (\tau, u, p,
\gamma_e, E_r)^\intercal$, where $\tau = 1/\rho$.  The matrix $A$ now
takes the form:
\begin{equation}
A = \left (
   \begin{matrix}
   u & - \tau & 0 & 0 & 0\\
   0 & u & \tau & 0 & \lambda_f \tau\\
   0 & \frac{c_{g}^{2}}{\tau} & u & 0 & 0\\
   0 & - \alpha & 0 & u & 0\\
   0 & E_{r} \left(\lambda_f + 1\right) & 0 & 0 & u
\end{matrix}\right)
\end{equation}
The eigenvalues are unchanged and the eigenvectors are derived in the
same Jupyter notebook as with the previous system.  
Here, $\alpha = (\gamma_e - 1)(\gamma_e - \Gamma_1)$.
Now we have
\begin{align}
\beta^\evp = ( l^\evp \cdot \Delta q^\evp ) &= -\frac{1}{2C}
   \left ( \Delta u^\evp + \frac{\Delta p_\mathrm{tot}^\evp}{C} \right ) \\
\beta^\evz_\tau = ( l^\evz_\tau \cdot \Delta q^\evz ) &= 
   \Delta \tau^\evz + \frac{\Delta p_\mathrm{tot}^\evz}{C^2} \\
\beta^\evz_{\gamma_e} = ( l^\evz_{\gamma_e} \cdot \Delta q^\evz ) &= 
   \Delta \gamma_E^\evz + \alpha \frac{\Delta p_\mathrm{tot}^\evz}{\tau C^2} \\
\beta^\evz_{E_r} = ( l^\evz_{E_r} \cdot \Delta q^\evz ) &= 
   \Delta E_r^\evz - \frac{h_r}{c^2} \Delta p_\mathrm{tot}^\evz \\
\beta^\evm = ( l^\evm \cdot \Delta q^\evm ) &= \frac{1}{2C} 
   \left ( \Delta u^\evm - \frac{\Delta p_\mathrm{tot}^\evm}{C} \right ) 
\end{align}
Here we use the Lagrangian sound speed, $C = \rho c = c/\tau$.

The interface states are then:
\begin{align}
\tau_\mathrm{int} &= \tau_\mathrm{ref} - \beta^\evp - \beta^\evm - \beta_\tau^\evz \\
u_\mathrm{int} &= u_\mathrm{ref} + C (\beta^\evp - \beta^\evm) \\
p_\mathrm{int} &= p_\mathrm{ref} + \frac{c_g^2}{\tau^2} ( \beta^\evp + \beta^\evm) + \beta_{E_r} \lambda_f \\
{\gamma_e}_\mathrm{int} &= {\gamma_e}_\mathrm{ref} - \beta_{\gamma_e}^\evz 
   - \frac{\alpha}{\tau} (\beta^\evp + \beta^\evm) \\
{E_r}_\mathrm{int} &= {E_r}_\mathrm{ref} + \frac{h_r}{\tau^2} (\beta^\evp + \beta^\evm) - \beta^\evz_{E_r}
\end{align}
Again, we can also construct the total pressure on the interface:
\begin{align}
{p_\mathrm{tot}}_\mathrm{int} &= p_\mathrm{int} + \lambda_f {E_r}_\mathrm{int}\\
    &= p_\mathrm{ref} + \lambda {E_r}_\mathrm{ref} + \frac{h_r \lambda_f + c_g^2}{\tau^2} (\beta^\evp + \beta^\evm) \\
    &= {p_\mathrm{tot}}_\mathrm{ref} + C^2 (\beta^\evp + \beta^\evm)
\end{align}

\end{document}
