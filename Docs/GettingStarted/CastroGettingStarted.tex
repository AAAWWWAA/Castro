
\section{Downloading the Code}

\castro\ is built on top of the \amrex\ framework.  In order to run
\castro\, you must download two separate git modules.

\vspace{.1in}

\noindent First, make sure that {\tt git} is installed on your machine---we recommend version 1.7.x or higher.

\vspace{.1in}

\begin{enumerate}

\item Clone/fork the \amrex\ repository from the {\tt AMReX-Codes} {\sf
  github} page (\url{https://github.com/AMReX-Codes/amrex/}).  To
  clone via the command line, simply type:
\begin{verbatim}
git clone https://github.com/AMReX-Codes/amrex.git
\end{verbatim}
Alternately, if you have a {\sf github} account with your
machine's SSH-keys registered, you can do:
\begin{verbatim}
git clone ssh://git@github.com/AMReX-Codes/amrex.git
\end{verbatim}

This will create a directory called {\tt amrex/} on your machine.

You will want to periodically update \amrex\ by typing
\begin{verbatim}
git pull
\end{verbatim}
in the {\tt amrex/} directory.  

Note: actively development is done on the {\tt development} branch
in each repo, and merged into the {\tt master} branch periodically.
If you wish to use the \castro\ {\tt development} branch, then you
should also switch to the {\tt development} branch for \amrex.

\item Set the environment variable, {\tt AMREX\_HOME}, on your
  machine to point to the path name where you have put \amrex.
  You can add this to your {\tt .bashrc} as:
\begin{Verbatim}[commandchars=\\\{\}]
export AMREX_HOME={\em /path/to/amrex/}
\end{Verbatim}
where you replace \texttt{\em /path/to/amrex/} will the full path to the
{\tt amrex/} directory.

\item Clone/fork the \castro\ repository from the same {\sf
  github} organization as above, using either HTTP access:
\begin{verbatim}
git clone https://github.com/AMReX-Astro/Castro.git
\end{verbatim}
or SSH access if you have it enabled:
\begin{verbatim}
git clone ssh://git@github.com:/AMReX-Astro/Castro.git
\end{verbatim}
Or, as above, you can download a ZIP file of the code from
\href{https://github.com/AMReX-Astro}{our main {\sf github} page},
by clicking on the \castro\ link.

As with \amrex, development on \castro\ is done in the
{\tt development} branch, so you should work there if you want
the latest source.

\item We recommend setting the {\tt CASTRO\_HOME} environment
  variable to point to the path name where you have put \castro.
  Add the following to your {\tt .bashrc}:
\begin{verbatim}
export CASTRO_HOME="/path/to/Castro/"
\end{verbatim}


\item (optional) An additional repository, {\tt Microphysics.git} is
  available at the {\tt BoxLib-Codes} github page.  This add
  additional reaction networks and EOSes and can be cloned following
  the same procedure as above\footnote{Note: previously the radiation
    solver was distributed separately as {\tt CastroRadiation.git},
    but this has been merged into the main \castro\ respository}:
\begin{verbatim}
git clone ssh://git@github.com:/BoxLib-Codes/Microphysics.git
\end{verbatim}

To access the \microphysics\ routines, set the {\tt MICROPHYSICS\_HOME}
environment variable to point to the {\tt Microphysics/} directory.

\end{enumerate}

%\clearpage

\section{Building the Code}

In \castro\ each different problem setup is stored in its own
sub-directory under {\tt Castro/Exec/}.  You build the
\castro\ executable in the problem sub-directory.  Here we'll
build the {\tt Sedov} problem:

\begin{enumerate}

\item From the directory in which you checked out the Castro git repo,
  type
\begin{verbatim}
cd Castro/Exec/hydro_tests/Sedov
\end{verbatim}
This will put you into a directory in which you can run the Sedov
problem in 1-d, 2-d or 3-d.

\item In {\tt Sedov/}, edit the {\tt GNUmakefile}, and set
  \begin{itemize}
    \item {\tt DIM = 2} 

      This is the dimensionality---here we pick 2-d.

    \item {\tt COMP = gnu}

      This is the set of compilers.  {\tt gnu} are a good default
      choice (this will use {\tt g++} and {\tt gfortran}.  You can
      also choose {\tt pgi} and {\tt intel} for example.

      If you want to try other compilers than the GNU suite and they
      don't work, please let us know.

    \item {\tt DEBUG = FALSE}

      This disabled debugging checks and results in a more
      optimized executable.

    \item {\tt USE\_MPI = FALSE}

      This turns off parallelization via MPI.  Set it to {\tt TRUE} to
      build with MPI---this requires that you have the MPI library
      installed on your machine.  In this case, the build system will
      need to know about your MPI installation.  This can be done by
      editing the makefiles in the \amrex\ tree, but the default
      fallback is to look for the standard MPI wrappers (e.g.\ {\tt
        mpic++} and {\tt mpif90}) to do the build.

  \end{itemize}

\item Now type {\tt make}.

  The resulting executable will look something like {\tt
    Castro2d.Linux.gnu.ex}, which means this is a 2-d version
  of the code, made on a Linux machine, with {\tt COMP = gnu}.

\end{enumerate}

\section{Running the Code}

\begin{enumerate}

\item \castro\ takes an input file that overrides the runtime parameter defaults.
  The code is run as:
\begin{verbatim}
Castro2d.Linux.gcc.gfortran.ex inputs.2d.cyl_in_cartcoords
\end{verbatim}

This will run the 2-d cylindrical Sedov problem in Cartesian ($x$-$y$
coordinates).  You can see other possible options, which should be
clear by the names of the inputs files.

\item You will notice that running the code generates directories that
  look like {\tt plt00000/}, {\tt plt00020/}, etc, and {\tt chk00000/},
  {\tt chk00020/}, etc. These are ``plotfiles'' and ``checkpoint''
  files. The plotfiles are used for visualization, the checkpoint
  files are used for restarting the code.

\end{enumerate}

\section{Visualization of the Results}
\index{visualization}

There are several options for visualizing the data.  The popular
\visit\ package supports the \amrex\ file format natively, as does the
\yt\ python package\footnote{Each of these will recognize it as the
  \boxlib\ format.}.  The standard tool used within the
\boxlib-community is \amrvis, which we demonstrate here.

\begin{enumerate}

\item Get \amrvis:

\begin{verbatim}
git clone https://ccse.lbl.gov/pub/Downloads/Amrvis.git
\end{verbatim}

Then cd into {\tt Amrvis/}, edit the {\tt GNUmakefile} there
to set {\tt DIM = 2}, and again set {\tt COMP} to compilers that
you have. Leave {\tt DEBUG = FALSE}.

Type {\tt make} to build, resulting in an executable that
looks like {\tt amrvis2d...ex}.

If you want to build amrvis with {\tt DIM = 3}, you must first
download and build {\tt volpack}:
\begin{verbatim}
git clone https://ccse.lbl.gov/pub/Downloads/volpack.git
\end{verbatim}

Then cd into {\tt volpack/} and type {\tt make}.

Note: \amrvis\ requires the OSF/Motif libraries and headers. If you don't have these 
you will need to install the development version of motif through your package manager. 
{\tt lesstif} gives some functionality and will allow you to build the amrvis executable, 
but \amrvis\ may not run properly.

On most Linux distributions, the motif library is provided by the
{\tt openmotif} package, and its header files (like {\tt Xm.h}) are provided
by {\tt openmotif-devel}. If those packages are not installed, then use the
package management tool to install them, which varies from
distribution to distribution, but is straightforward. 

You may then want to create an alias to {\tt amrvis2d}, for example
\begin{verbatim}
alias amrvis2d /tmp/Amrvis/amrvis2d...ex
\end{verbatim}
where {\tt /tmp/Amrvis/amrvis2d...ex} is the full path and name of the \amrvis\ executable.

\item Configure \amrvis:  

  Copy the {\tt amrvis.defaults} file to your home directory (you can
  rename it to {\tt .amrvis.defaults} if you wish).  Then edit the
  file, and change the {\tt palette} line to point to the full
  path/filename of the {\tt Palette} file that comes with \amrvis.

\item Visualize:

  Return to the {\tt Castro/Exec/hydro\_tests/Sedov} directory.  You should
  have a number of output files, including some in the form {\tt *pltXXXXX},
  where {\tt XXXXX} is a number corresponding to the timestep the file
  was output.  {\tt
    amrvis2d {\em filename}} to see a single plotfile, or {\tt amrvis2d -a
  *plt*}, which will animate the sequence of plotfiles.

  Try playing
  around with this---you can change which variable you are
  looking at, select a region and click ``Dataset'' (under View)
  in order to look at the actual numbers, etc. You can also export the
  pictures in several different formats under "File/Export".

Please know that we do have a number of conversion routines to other
formats (such as matlab), but it is hard to describe them all. If you
would like to display the data in another format, please let us know
(again, {\tt asalmgren@lbl.gov}) and we will point you to whatever we have
that can help.

\end{enumerate}

You have now completed a brief introduction to \castro. 


\section{Other Distributed Problem Setups}

There are a number of standard problem setups that come with \castro.
These can be used as a starting point toward writing your own setup.
We organize these into subdirectories by broad type (radiation, hydro,
gravity, etc.): The standard categories and {\em some} of the included
problems are:
\begin{itemize}
\item {\tt gravity\_tests}:

  \begin{itemize}
  \item {\tt DustCollapse}:

    A pressureless cloud collapse that is a standard test problem for
    gravity.  An analytic solution that describes the radius of the
    sphere as a function of time is found in Colgate and
    White~\cite{colgwhite}.  This problem is also found in the FLASH
    User's Guide.
    
  \item {\tt hydrostatic\_adjust}:

    Model a 1-d stellar atmosphere (plane-parallel or
    spherical/self-gravitating) and dump energy in via an analytic
    heat source and watch the atmosphere's hydrostatic state adjust in
    response.  This is the counterpart to the \maestro\ {\tt
      test\_basestate} unit test.

  \end{itemize}


\item {\tt hydro\_tests}:

  \begin{itemize}
  \item {\tt double\_bubble}:

    Initialize 1 or 2 bubbles in a stratified atmosphere (isothermal
    or isentropic) and allow for the bubbles to have the same or a
    different $\gamma$ from one another / the background atmosphere.
    This uses the {\tt multigamma} EOS.

    An analogous problem is implemented in \maestro.
    
  \item {\tt HCBubble}:
  
  \item {\tt KH}:

    A Kelvin-Helmholtz shear instability problem.
  
  \item {\tt oddeven}:

    A grid-aligned shock hitting a very small density perturbation.
    This demonstrates the odd-even decoupling problem discussed in
    \cite{quirk1997}.  This setup serves to test the {\tt
      castro.hybrid\_riemann} option to hydrodynamics.
  
  \item {\tt reacting\_bubble}:

    A reacting bubble in a stratified white dwarf atmosphere.  This
    problem was featured in the \maestro\ reaction
    paper~\cite{maestro:III}.

  \item {\tt RT}:

    A single-model Rayleigh-Taylor instability problem.
  
  \item {\tt RT\_particles}:

  \item {\tt Sedov}:

    The standard Sedov-Taylor blast wave problem.  This setup was used
    in the first \castro\ paper~\cite{castro_I}.
    
  \item {\tt Sod}:
  
    A one-dimensional shock tube setup, including the classic Sod
    problem.  This setup was used in the original \castro\ paper.
  
  \item {\tt Sod\_stellar}:

    A version of the Sod shock tube for the general stellar equation
    of state.  This setup and the included inputs files was used
    in~\cite{zingalekatz}.

  \item {\tt toy\_convect}:

    A simple nova-like convection problem with an external heating
    source.  This problem shows how to use the model parser to
    initialize a 1-d atmosphere on the Castro grid, incorporate a
    custom tagging routine, sponge the fluid above the atmosphere, and
    write a custom diagnostics routine.

    A \maestro\ version of this problem setup also exists.
  \end{itemize}
    

\item{\tt radiation\_tests}:

\item{\tt science}:

\item{\tt unit\_tests}:


  
  
  
\end{itemize}
