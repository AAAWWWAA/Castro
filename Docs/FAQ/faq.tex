\section{Compiling}

\begin{enumerate}

\item {\em Compiling fails giving me a cryptic message about a module not
  being found, usually {\tt bl\_types} or {\tt bl\_error\_module}, like:
\begin{verbatim}
mpif90 -fno-range-check -fno-second-underscore
 -Jo/3d.Linux.gcc.gfortran.MPI.EXE -I o/3d.Linux.gcc.gfortran.MPI.EXE
 -ffixed-line-length-0 -g -O3
 -I. -I/home/zingale/development/Microphysics/util
 -I../../Microphysics/EOS -I../../Microphysics/EOS/gamma_law
 -I../../Microphysics/networks
 -I../../Microphysics/networks/general_null
 -I. -I/home/zingale/development/BoxLib//Src/C_BaseLib
 -I/home/zingale/development/BoxLib//Src/C_AMRLib
 -I/home/zingale/development/BoxLib//Src/C_BoundaryLib -I../../Source
 -I../../Source/Src_3d -I../../Source/Src_nd -I../../constants
 -I/home/zingale/development/BoxLib//Src/F_BaseLib
 -I/home/zingale/development/BoxLib//Tools/C_scripts -c
 ../../Microphysics/EOS/eos.f90 -o
 o/3d.Linux.gcc.gfortran.MPI.EXE/eos.o
../../Microphysics/EOS/eos.f90:3:6:

   use bl_types
      1
Fatal Error: Can't open module file ‘bl_types.mod’ for reading at (1):
 No such file or directory compilation terminated.

/home/zingale/development/BoxLib//Tools/C_mk/Make.rules:122: recipe
 for target 'o/3d.Linux.gcc.gfortran.MPI.EXE/eos.o' failed
\end{verbatim}
}

This usually indicates that the build system cannot find a source file
(note: the problem above is not {\tt bl\_types}, that just seems to be
the way the error manifests itself).  The source files are specified
in the various {\tt Make.package} files throughout the
\castro\ directory hierarchy.  {\tt make} will look through the
directories in the {\tt VPATH\_LOCATIONS} to find the files.

There are 2 things you can do to check what's happening.  First, inspect
the directories in {\tt VPATH\_LOCATIONS}.  This can be done via:
\begin{verbatim}
make print-VPATH_LOCATIONS
\end{verbatim}

Next, ask {\tt make} to tell you where it is finding each of the source
files.  This is done through a script {\tt find\_files\_vpath.py}
 that is hooked into \castro's build system.  You can run this as:
\begin{verbatim}
make file_locations
\end{verbatim}
At the end of the report, it will list any files it cannot find in
the {\tt vpath}.  Some of these are to be expected (like {\tt extern.f90}
and {\tt buildInfo.cpp}---these are written at compiletime.  But any
other missing files need to be investigated.

\item {\em I'm still having trouble compiling.  How can I find out what
  all of the make variables are set to?}

  Use:
\begin{verbatim}
make help
\end{verbatim}
  This will tell you the value of all the compilers and their options.



\end{enumerate}

\section{Debugging}

\begin{enumerate}

\item {\em \castro\ crashes with a floating point exception---how can
  I get more information?}

  The best thing to do is to recompile the code with {\tt TEST=TRUE}
  set in the {\tt GNUmakefile}.  This will have \amrex\ catch the
  signals raised in both \cpp\ and Fortran functions.  Behind the
  scenes, this defines the {\tt BL\_TESTING} preprocessor flag, which
  will initialize memory allocated in {\tt fab}s or {\tt multifab}s to
  signaling NaNs (sNaN), and use the {\tt BLBackTrace::handler()}
  function to handle various signals raised in both \cpp\ and Fortran
  functions.  This is a Linux/UNIX capability.  This gives us a chance
  to print out backtrace information.  The signals include seg fault,
  floating point exceptions (NaNs, divided by zero and overflow), and
  interruption by the user and system.  What signals are handed to
  \amrex\ are controlled by \amrex (e.g., using interruption by the
  user, this was once used to find an MPI deadlock.)  It also includes
  the {\tt BL\_ASSERTION} statements if {\tt USE\_ASSERTION=TRUE} or
  {\tt DEBUG=TRUE}. 

  The \amrex\ parameters that affect the behavior are:
  \begin{itemize}
    \item {\tt amrex.fpe\_trap\_invalid}
    \item {\tt amrex.fpe\_trap\_zero}
    \item {\tt amrex.fpe\_trap\_overflow}
  \end{itemize}

  For further capabilities, defining {\tt BACKTRACE=TRUE} enables you
  to get more information than the backtrace of the call stack info by
  instrumenting the code.  (This is in \cpp\ code only). Here is an
  example.  You know the line ``{\tt Real rho = state(cell,0);}'' is
  causing a segfault.  You could add a print statement before that.
  But it might print out thousands (or even millions) of line before
  it hits the segfault.  With {\tt BACKTRACE}, you could do

\begin{verbatim}
      #ifdef BL_BACKTRACING
         std::ostringstream ss;
         ss << ``state.box() = `` << state.box() << `` cell = `` << cell;
         BL_BACKTRACE_PUSH(ss.str()); // PUSH takes std::string
      #endif

      Real rho = state(cell,0);  // state is a Fab, and cell is an IntVect.
\end{verbatim}          

    The ``print'' prints to a stack of string, not stdout.  When it
    hits the segfault, you will only see the last print out.



\end{enumerate}


\section{Profiling}

\begin{enumerate}

\item {\em How can I get line-by-line profiling information?}

  With the GNU compliers, you can enabling profiling with {\tt gprof}
  by compiling with
\begin{verbatim}
  USE_GPROF=TRUE
\end{verbatim}
  in your {\tt GNUmakefile}.

  When you run, a file named {\tt gmon.out} will be produced.  This can
  be processed with {\tt gprof} by running:
\begin{verbatim}
  gprof exec-name
\end{verbatim}
  where {\em exec-name} is the name of the executable.  More detailed
  line-by-line information can be obtained by passing the {\tt -l}
  argument to {\tt gprof}.

\end{enumerate}


\section{Managing Runs}

\begin{enumerate}

\item {\em How can I force the running code to output, even it the plot or
 checkpoint interval parameters don't require it?}

Create a file called \code{dump\_and\_continue}, e.g., as:
\begin{verbatim}
touch dump_and_continue
\end{verbatim}

This will force the code to output a checkpoint file that can be used
to restart.  Other options are \code{plot\_and\_continue} to output
a plotfile, \code{dump\_and\_stop} to output a checkpoint file
and halt the code, and \code{stop\_run} to simply stop the code.
Note that the parameter \runparam{amr.message\_int} controls how often
the existence of these files is checked; by default it is 10, so the
check will be done at the end of every timestep that is a multiple of 10.
Set that to 1 in your inputs file if you'd like it to check every timestep.




\end{enumerate}
