\section{Debugging}

\begin{enumerate}

\item {\em \castro\ crashes with a floating point exception---how can
  I get more information?}

  The best thing to do is to recompile the code with {\tt TEST=TRUE}
  set in the {\tt GNUmakefile}.  This will have \boxlib\ catch the
  signals raised in both C++ and Fortran functions.  Behind the
  scenes, this defines the {\tt BL\_TESTING} preprocessor flag, which
  will initialize memory allocated in {\tt fab}s or {\tt multifab}s to
  signaling NaNs (sNaN), and use the {\tt BLBackTrace::handler()}
  function to handle various signals raised in both C++ and Fortran
  functions.  This is a Linux/UNIX capability.  This gives us a chance
  to print out backtrace information.  The signals include seg fault,
  floating point exceptions (NaNs, divided by zero and overflow), and
  interruption by the user and system.  What signals are handed to
  BoxLib are controlled by BoxLib.  (E.g., using interruption by the
  user, this was once used to find an MPI deadlock.)  It also includes
  the {\tt BL\_ASSERTION} statements if {\tt USE\_ASSERTION=TRUE} or
  {\tt DEBUG=TRUE}. 

  The BoxLib parameters that affect the behavior are:
  \begin{itemize}
    \item {\tt boxlib.fpe\_trap\_invalid}
    \item {\tt boxlib.fpe\_trap\_zero}
    \item {\tt boxlib.fpe\_trap\_overflow}
  \end{itemize}

  For further capabilities, defining {\tt BACKTRACE=TRUE} enables you
  to get more information than the backtrace of the call stack info by
  instrumenting the code.  (This is in C++ code only). Here is an
  example.  You know the line ``{\tt Real rho = state(cell,0);}'' is
  causing a segfault.  You could add a print statement before that.
  But it might print out thousands (or even millions) of line before
  it hits the segfault.  With {\tt BACKTRACE}, you could do

\begin{verbatim}
      #ifdef BL_BACKTRACING
         std::ostringstream ss;
         ss << ``state.box() = `` << state.box() << `` cell = `` << cell;
         BL_BACKTRACE_PUSH(ss.str()); // PUSH takes std::string
      #endif

      Real rho = state(cell,0);  // state is a Fab, and cell is an IntVect.
    \end{verbatim}

    The ``print'' prints to a stack of string, not stdout.  When it
    hits the segfault, you will only see the last print out.



\end{enumerate}
