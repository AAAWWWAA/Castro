\section{Overview}
We can now initialize a \castro\ simulation using data from a \maestro\
plotfile.  This should not be thought of as a restart mode, but rather
a new simulation with a special initialization.  In order to use this
feature, you must make sure the \maestro\ plotfile has the proper
variables, add some new parameters to your inputs file, and add a few
subroutines to Prob\_Xd.f90.  You need to build a special executable
with ``USE\_MAESTRO\_INIT=TRUE'', which will add ``.MAESTRO'' to the
executable string.  For multilevel problems, there are a few extra
steps relating to the fact that you have to supply a grids file
consistent with the \maestro\ grid structure.

\section{MAESTRO Plotfile Requirements}
The \maestro\ plotfile needs to have the following variables:
\begin{itemize}
\item ``x\_vel'', ``y\_vel'', (and ``z\_vel'', depending on
  dimensionality of the problem)
\item ``density'' ({\bf castro.MAESTRO\_init\_type} = 1 and 2 only)
\item Optional species (such as ``X(C12)'') - there is an option to
  not read any species from the \maestro\ plotfile.  In this case, you
  must make sure your code manually defines the species cell-by-cell
  in the initial \castro\ data
\item ``tfromp''
\item ``pi'' ({\bf castro.MAESTRO\_init\_type} = 2, 3, and 4 only)
\item ``entropy'' ({\bf castro.MAESTRO\_init\_type} = 4 only)
\end{itemize}
Also, model\_cc\_XXXXX needs to list variables in the following order,
which is the default order found in {\tt MAESTRO/Source/base\_io.f90}: r,
base\_r, rho0, p0, gamma1bar, rhoh0, div\_coeff, psi, tempbar,
etarho\_cc, tempbar\_init.

\section{List of Parameters}
Here are the additional parameters you must add to your inputs file.
\begin{table*}[h]
\begin{tiny}
\begin{tabular}{|l|l|l|l|} \hline
Parameter & Definition & Type & Default \\
\hline
    {\bf castro.MAESTRO\_plotfile} & name of the \maestro\ plotfile & std::string & must be set \\
    
    {\bf castro.MAESTRO\_modelfile} & name of the \maestro\ ``model\_cc'' file & std::string & must be set \\
    
    {\bf castro.MAESTRO\_npts\_model} & number of points in the \maestro\ model\_cc file & int & must be set \\
    
    {\bf castro.MAESTRO\_first\_species} & name of the first species & std::string & must be set or else nothing will be read in \\
    
    {\bf castro.MAESTRO\_nspec} & number of species in the \maestro\ plotfile & std::string & NumSpec in \castro \\
    
    {\bf castro.MAESTRO\_cutoff\_density} & controls how we overwrite data at the edge of the star & Real & must be set \\
    
    {\bf castro.MAESTRO\_init\_type} & determines how we initialize the \castro\ state & int & must be set \\
    
    {\bf castro.MAESTRO\_spherical} & specifies planar or spherical problem & int & must be set \\
    
\hline
\end{tabular}
\end{tiny}
\end{table*}

\subsection{Examples of Usage}
\begin{itemize}
\item {\bf castro.MAESTRO\_plotfile} = "wd\_384\_6.25e8K\_norotate\_plt120578"
\item {\bf castro.MAESTRO\_modelfile} = "./wd\_384\_6.25e8K\_norotate\_plt120578/model\_cc\_120578"
\item {\bf castro.MAESTRO\_npts\_model} = 1663\\ This is the number of
  points in {\bf castro.MAESTRO\_modelfile}.  Note that this is not
  the same thing as ``npts\_model'', which is the number of points in
  the initial model file used for standard simulations where we do not
  initialize from a \maestro\ plotfile.
\item {\bf castro.MAESTRO\_first\_species} = ``X(C12)'' If you do not
  specify this, no species will be read in.  You can always manually
  specify or overwrite the species cell-by-cell later.
\item {\bf castro.MAESTRO\_nspec} = 3\\ If you do not specify this, it
  will default to the number of species in the \castro\ network,
  ``NumSpec''.  We have this here because sometimes \maestro\ and \castro\
  will use different networks with different number of species.
\item {\bf castro.MAESTRO\_cutoff\_density} = 1.e6\\ The code will use
  this density to figure out the radial coordinate, r\_model\_start,
  which is the last radial coordinate before rho0 falls below {\bf
    castro.MAESTRO\_cutoff\_density}.  It is possible to set {\bf
    castro.MAESTRO\_cutoff\_density} to a tiny value, such that rho0
  never falls below this value, in which case we set r\_model\_start
  to $\infty$.  In INITDATA\_MAKEMODEL, we create a new 1D model
  integrating outward starting from r\_model\_start.  Then, in
  INITDATA\_OVERWRITE, we overwrite newly initialized \castro\ data in
  any cell that maps into a radial coordinate greater than
  r\_model\_start by interpolating from the new 1D model.
\item {\bf castro.MAESTRO\_init\_type} = 2\\ \castro\ will read in data
  from the \maestro\ plotfile, and then call the EOS to make sure that
  $\rho$, $e$, $T$, and $X_k$ are consistent.  The inputs to the EOS
  are based on the value of {\bf castro.MAESTRO\_init\_type}:
\begin{enumerate}
\item $e = e(\rho,T,X_k)$
\item $e,T = e,T(\rho,p_0+\pi,X_k)$
\item $\rho,e = \rho,e(p_0+\pi,T,X_k)$
\item $\rho,T,e = \rho,T,e(p_0+\pi,s,X_k)$
\end{enumerate}
\item {\bf castro.MAESTRO\_spherical} = 1\\
0 = planar; 1 = spherical.
\end{itemize}

\section{New Subroutines in Prob\_Xd.f90}
There are three routines that need to be added to your local copy of
{\tt Prob\_Xd.f90}.  See {\tt Castro/Exec/wdconvect/Prob\_3d.f90} for
a standard spherical \maestro\ initialization.
\begin{enumerate}
\item INITDATA\_MAESTRO\\ This fills in the \castro\ state by taking
  the \maestro\ data, calling the EOS, and making the proper variables
  conserved quantities.  Specifically, we need a thermodynamically
  consistent $\rho$, $T$, $e$, and $X_k$, and then algebraically
  compute $\rho{\bf u}$, $\rho e$, $\rho E$, and $\rho X_k$,
\item INITDATA\_MAKEMODEL\\
This creates a user-defined 1D initial model starting from r\_model\_start.
\item INITDATA\_OVERWRITE\\ This overwrites the initialized \castro\
  data using the new 1D initial model for all cells that map into
  radial coordinates greater than r\_model\_start.
\end{enumerate}

\section{Additional Notes}
Note that for both single-level and multilevel \maestro\ to \castro\
initialization, the \castro\ base grid structure does not have to match
the \maestro\ base grid structure, as long as the problem domain is the
same.  For example, if the coarsest level in a \maestro\ plotfile
contains $64^3$ cells divided into 8-$32^3$ grids, it is ok to use a
\castro\ base grid structure with 1-$64^3$ grid, 64-$16^3$ grids, or
anything else you can imagine - the grids don't even have to be the
same size.  As is normally the case, the \castro\ base grid structure is
created based on the parameters in the \castro\ inputs file, such as
{\bf amr.max\_grid\_size}, {\bf amr.blocking\_factor}, etc.

\subsection{Multilevel Restart}
When initialing from a multilevel \maestro\ plotfile, there are some
extra steps.  First, you need to create a \castro-compatible grids file
from the \maestro\ plotfile.  This can be done with the {\tt
  BoxLib/Tools/Postprocessing/F\_Src/fboxinfo.f90} utility.  Compile
and run this using the ``\texttt{--}castro'' option, e.g.,
``fboxinfo.Linux.gfortran.exe \texttt{--}castro pltxxxxx \texttt{|}
tee gr0.maestro'', to generate the \castro-compatible grids file.  Note
that the base grid structure is still controlled by {\bf
  amr.max\_grid\_size}, {\bf amr.blocking\_factor}, etc., since in \cpp\
BoxLib, the grids file only indicates the refined grid structure,
whereas in Fortran BoxLib the grids file contains the base grid and
refined grid structures.

Now, when you initialize the \castro\ simulation, you need to specify
the grid file using {\bf amr.regrid\_file = "gr0\_3d.128\_2levels"},
for example.  You can happily run this now, but note that the
regridding algorithm will never be called (since \castro\ thinks it's
started a new simulation from scratch with a grids file, thus
disabling the regridding).  If you wish for the grid structure to be
changed, you must do a traditional \castro\ restart from the
\castro-generated checkpoint file (you can still use the same
``.MAESTRO'' executable or an executable built with
USE\_MAESTRO\_INIT=FALSE), making sure that you {\bf do not} specity
{\bf amr.regrid\_file} (or else the grids will stay fixed).  You are
free to specify {\bf amr.regrid\_on\_restart}, {\bf
  amr.compute\_new\_dt\_on\_regrid}, and {\bf
  amr.plotfile\_on\_restart}.

Sometimes a \maestro\ plotfile will only have 1 or 2 total levels, but
you ultimately want to run a \castro\ simulation with many more levels
of refinement.  My recommended strategy is the following:
\begin{enumerate}
\item Initialize a \castro\ simulation from the \maestro\ plotfile
  while preserving the exact same grid structure and run for 10 time
  steps.
\item Do a traditional \castro\ restart from {\tt chk00010}, but do not
  increase {\bf amr.max\_level}, and run for 10 more time steps.  This
  allows a new grid structure with the same effective resolution as
  before settle in using the \cpp\ \boxlib\ regridding algorithm.
\item Do a traditional \castro\ restart from {\tt chk00020}, but increase
  {\bf amr.max\_level} by 1, and run for 10 time steps.
\item Repeat the procedure from the previous step (using the most
  updated checkpoint of course) as many times as desired.
\end{enumerate}
