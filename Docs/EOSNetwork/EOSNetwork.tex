\section{Equation of State}

\subsection{Standard \castro\ EOSes}

\castro\ is written in a modular fashion so that the EOS and network
burning routines can be supplied by the user. However, for the
examples presented later we use several EOS and network routines
that come with the \microphysics\ distribution.

\castro\ relies on routines to calculate the equation of state (EOS)
of a fluid, as well as a species network to define the components of
the fluid. The network optionally has the ability to do nuclear burning,
but for this section its main purpose is in defining the species so that
the EOS can calculate fluid properties that depend on composition, such
as electron fraction.

By default, \castro\ comes with the {\tt gamma\_law}
EOS. This represents a gamma law gas, with equation of state:
\begin{equation}
  P = (\gamma - 1) \rho e.
\end{equation}
The gas is currently assumed to be monatomic and ideal. (Only a 
restricted set of thermodynamic variables are actually calculated, 
the minimum necessary for the hydrodynamics. A fuller set of 
thermodynamic variables, for example the entropy from the 
Sackur-Tetrode equation, are calculated in the {\tt gamma\_law\_general}
EOS inside the \microphysics\ repository.)

\subsection{EOS Interfaces and Parameters}

Each EOS should have two main routines by which it interfaces to the
rest of \castro.  At the beginning of the simulation, {\tt eos\_init}
will perform any initialization steps and save EOS variables (mainly
\texttt{smallt}, the temperature floor, and \texttt{smalld}, the
density floor). Then, whenever you want to call the EOS, use
\[
  {\tt call\ eos (eos\_input, eos\_state)}.
\]
The first argument specifies the inputs to the EOS. The options
that are currently available are stored in
{\tt EOS/eos\_data.F90}, and are always a combination of two
thermodynamic quantities. For example, {\tt eos\_input\_rt} means
that we call the EOS with {\tt rho} (density) and {\tt T} (temperature)
and we expect the EOS to return the associated thermodynamic
quantities such as internal energy {\tt e} and entropy {\tt s}.

We note that for real (non-analytic) equations of state
in which $\rho$, $T$ and species are the independent variables, such
as the Helmholtz EOS, {\tt eos\_input\_rt} directly calls the EOS
and obtains the other thermodynamic variables. But for other inputs,
e.g. {\tt eos\_input\_re}, a Newton-Raphson iteration is performed
to find the density or temperature that corresponds to the given
input.

The {\tt eos\_state} variable is a Fortran derived type (similar to
a \cpp\ struct). It stores a complete set of thermodynamic
variables. When calling the EOS, you should first fill the variables
that are the inputs, for example with
\begin{verbatim}
  use eos_type_module
  ...
  type (eos_t) :: eos_state
  ...
  eos_state % rho = state(i,j,k,URHO)
  eos_state % T   = state(i,j,k,UTEMP)
  eos_state % e   = state(i,j,k,UEINT) / state(i,j,k,URHO)
  eos_state % xn  = state(i,j,k,UFS:UFS+nspec-1) / state(i,j,k,URHO)
  eos_state % aux = state(i,j,k,UFX:UFX+naux-1) / state(i,j,k,URHO)
\end{verbatim}
Whenever the \texttt{eos\_state} type is initialized, the
thermodynamic state variables are filled with unphysical numbers. If
you do not input the correct arguments to match your input quantities,
the EOS will call an error. This means that it is good
practice to fill the quantities that will be iterated over with an
initial guess. Indeed, this initial guess is typically required for
equations of state that iterate over this variable, as the values
they are initialized with will likely not
converge. Usually a prior value of the temperature or density suffices
if it's available, but if not then use \texttt{small\_temp} or
\texttt{small\_dens}.

If you are interested in using more realistic and sophisticated equations of
state, you should download the \href{https://github.com/starkiller-astro/Microphysics}{\microphysics}
repository. This is a collection of microphysics routines that are compatible with the
\boxlib\ codes. We refer you to the documentation in that repository for how to set it up
and for information on the equations of state provided. That documentation
also goes into more detail about the details of the EOS code, in case you are interested in
how it works (and in case you want to develop your own EOS).

\section{Nuclear Network}

The nuclear network serves two purposes: it defines the fluid components used
in both the equation of state and the hydrodynamics, and it evolves those
components through a nuclear burning step. \castro\ comes with a {\tt general\_null}
network (which lives in the {\tt Networks/} directory). This is a bare interface for a
nuclear reaction network. No reactions are enabled, and no auxiliary variables
are accepted. It contains several sets of isotopes; for example,
{\tt Networks/general\_null/triple\_alpha\_plus\_o.net} would describe the
isotopes needed to represent the triple-$\alpha$ reaction converting
helium into carbon, as well as oxygen and iron.

The main interface file, {\tt network.f90}, is a wrapper function. The
actual network details are defined in {\tt actual\_network.f90}, a 
file which is automatically generated in your work directory when you compile.
It supplies the number and names of species and auxiliary variables, as
well as other initializing data, such as their mass numbers, proton numbers, 
and the binding energies.

The burning front-end interface, {\tt Networks/burner.f90}, accepts a different 
derived type called the {\tt burn\_t} type. Like the {\tt eos\_t}, it has entries 
for the basic thermodynamic quantities:
\begin{verbatim}
  use burn_type_module
  ...
  type (burn_t) :: burn_state
  ...
  burn_state % rho = state(i,j,k,URHO)
  burn_state % T   = state(i,j,k,UTEMP)
  burn_state % e   = state(i,j,k,UEINT) / state(i,j,k,URHO)
  burn_state % xn  = state(i,j,k,UFS:UFS+nspec-1) / state(i,j,k,URHO)
\end{verbatim}
It takes in an input {\tt burn\_t} and returns an output {\tt burn\_t} after 
the burning has completed. The nuclear energy release can be computed by 
taking the difference of {\tt burn\_state\_out \% e} and 
{\tt burn\_state\_in \% e}. The species change can be computed analogously. 
In normal operation in \castro\, the integration occurs over a time interval 
of $\Delta t/2$, where $\Delta t$ is the hydrodynamics timestep.

If you are interested in using actual nuclear burning networks,
you should download the \href{https://github.com/starkiller-astro/Microphysics}{\tt Microphysics}
repository. This is a collection of microphysics routines that are compatible with the
\boxlib\ codes. We refer you to the documentation in that repository for how to set it up
and for information on the networks provided. That documentation
also goes into more detail about the details of the network code, in case you are interested in
how it works (and in case you want to develop your own network).

\subsection{Required Thermodynamics Quantities}

Three input modes are required of any EOS:
\begin{itemize}
\item {\tt eos\_input\_re}: $\rho$, $e$, and $X_k$ are input
\item {\tt eos\_input\_rt}: $\rho$, $T$, and $X_k$ are input
\item {\tt eos\_input\_rp}: $\rho$, $P$, and $X_k$ are input
\end{itemize}

The {\tt eos\_t} derived type holds a large number of thermodynamics
quantities, but not all of these are needed for basic
\castro\ operation.  The main quantities that any EOS in any mode needs to
supply, if they are not input, are:
\begin{itemize}
  \item {\tt eos\_state \% T}: the temperature
  \item {\tt eos\_state \% P}: total pressure
  \item {\tt eos\_state \% e}: the specific energy
  \item {\tt eos\_state \% gam1}: the first adiabatic index,
       $\Gamma_1 = d\log P / d\log \rho |_s$
\end{itemize}

Additionally the {\tt eos\_input\_re} mode also needs to supply:
\begin{itemize}
  \item {\tt eos\_state \% cs}: the adiabatic sound speed
  \item {\tt eos\_state \% dpdr\_e}: the derivative, $\partial
    p/\partial \rho |_e$---note that the specific internal energy, $e$
    is held constant here.
  \item {\tt eos\_state \% dpde}: the derivative, $\partial p /
    \partial e |_\rho$
\end{itemize}

For radiation hydro, the {\tt eos\_input\_rt} model needs to supply:
\begin{itemize}
  \item {\tt eos\_state \% cv}: the specific heat capacity.
\end{itemize}
Other quantities (e.g., entropy) might be needed for the derived
variables that are optional output into the plotfiles.

For burning, when the temperature equation is evolved, the EOS
needs to supply:
\begin{itemize}
\item {\tt eos\_state \% dhdX(nspec)}: the derivative of the
  specific enthalpy with respect to mass fraction at constant
  $T$ and $p$.  This is commonly computed as:
  \begin{equation}
    \xi_k = e_{X_k} + \frac{1}{p_\rho} \left (\frac{p}{\rho^2} - e_\rho \right ) p_{X_k}\enskip .
  \end{equation}
  with
  \begin{eqnarray}
p_{X_k} &=& \left .\frac{\partial p}{\partial \bar{A}} \right |_{\rho, T, \bar{Z}}
          \frac{\partial \bar{A}}{\partial X_k} +
          \left . \frac{\partial p}{\partial \bar{Z}} \right |_{\rho, T, \bar{A}}
          \frac{\partial \bar{Z}}{\partial X_k} \nonumber \\
        &=& -\frac{\bar{A}^2}{A_k}
          \left .\frac{\partial p}{\partial \bar{A}} \right |_{\rho, T, \bar{Z}} +
          \frac{\bar{A}}{A_k} \left (Z_k - \bar{Z} \right )
          \left . \frac{\partial p}{\partial \bar{Z}} \right |_{\rho, T, \bar{A}}\enskip,
\end{eqnarray}
\begin{eqnarray}
e_{X_k} &=& \left . \frac{\partial e }{\partial \bar{A}} \right |_{\rho, T, \bar{Z}}
        \frac{\partial \bar{A}}{\partial X_k} +
        \left .\frac{\partial e}{\partial \bar{Z}} \right |_{\rho, T, \bar{A}}
        \frac{\partial \bar{Z}}{\partial X_k} \nonumber \\
        &=& -\frac{\bar{A}^2}{A_k}
        \left . \frac{\partial e }{\partial \bar{A}} \right |_{\rho, T, \bar{Z}} +
        \frac{\bar{A}}{A_k} \left (Z_k - \bar{Z}\right )
        \left .\frac{\partial e}{\partial \bar{Z}} \right |_{\rho, T, \bar{A}}\enskip.
\end{eqnarray}
\end{itemize}
(see \cite{maestro:III}, Appendix A).

