\label{chapter:parameters}

\section{Introduction to Runtime Parameters}

Castro has 2 sets of runtime parameters---those controlled by
\cpp\ and those controlled by Fortran.  The \cpp\ parameters are set
in the {\tt inputs} file and managed by the \boxlib\ {\tt ParmParse}
class.  For \castro-specific parameters, we list the runtime
parameters in a file {\tt \_cpp\_parameters} and generate the
\cpp\ code and headers using the {\tt mk\_params.sh} script---note
this script needs to be run every time the {\tt \_cpp\_parameters}
file is updated.

The behavior of the network and EOS are controlled by many runtime
parameters.  These parameters are defined in plain-text files {\tt
  \_parameters} located in the different directories that hold the
microphysics code.  At compile time, a script in the \boxlib\ bulid
system, {\tt findparams.py}, locates all of the {\tt \_parameters}
files that are needed for the given choice of network, integrator, and
EOS, and assembles all of the runtime parameters into a module named
{\tt extern\_probin\_module} (using the {\tt write\_probin.py}
script).  The parameters are set in your {\tt probin} file in the {\tt
  \&extern} namelist.

Parameter definitions take the form of:
\begin{verbatim}
# comment describing the parameter
name              data-type       default-value      priority
\end{verbatim}
Here, the {\tt priority} is simply an integer.  When two directories
define the same parameter, but with different defaults, the version of
the parameter with the highest priority takes precedence.  This allows
specific implementations to override the general parameter defaults.

The documentation below for the \castro\ \cpp\ parameters is
automatically generated, using the comments in the {\tt \_cpp\_parameters}
file.



\section{Removed Runtime Parameters}

The following runtime parameters have been removed for \castro.
\begin{itemize}
\item {\tt castro.ppm\_flatten\_before\_integrals} : this parameter
  controlled whether we applied the flattening of the parabolic
  profiles before we integrated under their profiles or afterwards.
  The default was switched to flattening before the integration,
  which is more consistent with the original PPM methodology.  This
  parameter was removed since the variation enabled by this parameter
  was not that great. 

  (removed in commit: {\tt 9cab697268997714919de16db1ca7e77a95c4f98})


\item {\tt castro.ppm\_reference} and {\tt
  castro.ppm\_reference\_edge\_limit} : these parameters controlled
  whether we use the integral under the parabola for the fastest wave
  moving toward the interface for the reference state and whether in
  the case that the wave is moving away from the interface we use the
  cell-center value or the limit of the parabola on the interface.
  These were removed because there is little reason to not use the
  reference state.

  (removed in commit: {\tt 213f4ffc53463141084551c7be4b37a2720229aa})
\end{itemize}

