\label{chapter:parameters}

\section{Introduction to Runtime Parameters}

Castro has 2 sets of runtime parameters---those controlled by
\cpp\ and those controlled by Fortran.  The \cpp\ parameters are set
in the {\tt inputs} file and managed by the \amrex\ {\tt ParmParse}
class.  For \castro-specific parameters, we list the runtime
parameters in a file \code{\_cpp\_parameters} and generate the
\cpp\ code and headers using the \code{mk\_params.sh} script---note
this script needs to be run every time the {\tt \_cpp\_parameters}
file is updated.

The behavior of the network, EOS, and other microphysics routines are
controlled by a different set of runtime parameters.  These parameters are defined
in plain-text files \code{\_parameters} located in the different
directories that hold the microphysics code.  At compile time, a
script in the \amrex\ bulid system, \code{findparams.py}, locates all
of the {\tt \_parameters} files that are needed for the given choice
of network, integrator, and EOS, and assembles all of the runtime
parameters into a module named \code{extern\_probin\_module} (using the
\code{write\_probin.py} script).  The parameters are set in your {\tt
  probin} file in the {\tt \&extern} namelist.

\subsection{C++ parameter format}

The \cpp\ parameters take the form of:
\begin{verbatim}
# comment describing the parameter
name   type   default   need in Fortran?   ifdef    fortran name    fortran type
\end{verbatim}
Here, {\tt name} is the name of the parameter that will be looked for
in the {\tt inputs} file, {\tt type} is one of {\tt int}, {\tt Real},
or {\tt string}, and {\tt default} is the default value of the
parameter.  The next columns are optional, but you need to fill in all
of the information up to and including any of the optional columns you
need (e.g., if you are going to provide the {\tt fortran name}, you
also need to provide {\tt need in Fortran?} and {\tt ifdef}.  The {\tt
  need in Fortran?} column is {\tt y} if the runtime parameter should
be made available in Fortran (through \code{meth\_params\_module}).
The {\tt ifdef} field provides the name of a preprocessor name that
should wrap this parameter definition---it will only be compiled in if
that name is defined to the preprocessor.  The {\tt fortran name} is
the name that the parameter should use in Fortran---by default it will
be the same as {\tt name}.  The {\tt fortran type} is the data type of
the parameter in Fortran---by default it will be the
Fortran-equivalent to {\tt type}.  Finally, any comment immediately
before the parameter definition will be used to generate the \LaTeX\ documentation
describing the parameters.



\subsection{Microphysics/extern parameter format}

The microphysics/extern parameter definitions take the form of:
\begin{verbatim}
# comment describing the parameter
name              data-type       default-value      priority
\end{verbatim}
Here, the {\tt priority} is simply an integer.  When two directories
define the same parameter, but with different defaults, the version of
the parameter with the highest priority takes precedence.  This allows
specific implementations to override the general parameter defaults.

The documentation below for the \castro\ \cpp\ parameters is
automatically generated, using the comments in the {\tt \_cpp\_parameters}
file.



\section{Removed Runtime Parameters}

The following runtime parameters have been removed for \castro.
\begin{itemize}
\item {\tt castro.ppm\_flatten\_before\_integrals} : this parameter
  controlled whether we applied the flattening of the parabolic
  profiles before we integrated under their profiles or afterwards.
  The default was switched to flattening before the integration,
  which is more consistent with the original PPM methodology.  This
  parameter was removed since the variation enabled by this parameter
  was not that great. 

  (removed in commit: {\tt 9cab697268997714919de16db1ca7e77a95c4f98})


\item {\tt castro.ppm\_reference} and {\tt
  castro.ppm\_reference\_edge\_limit} : these parameters controlled
  whether we use the integral under the parabola for the fastest wave
  moving toward the interface for the reference state and whether in
  the case that the wave is moving away from the interface we use the
  cell-center value or the limit of the parabola on the interface.
  These were removed because there is little reason to not use the
  reference state.

  (removed in commit: {\tt 213f4ffc53463141084551c7be4b37a2720229aa})
\end{itemize}

