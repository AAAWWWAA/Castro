\section{ {\tt castro } Namespace}

\label{ch:parameters}


%%%%%%%%%%%%%%%%
% symbol table
%%%%%%%%%%%%%%%%

\begin{landscape}


{\small

\renewcommand{\arraystretch}{1.5}
%
\begin{center}
\begin{longtable}{|l|p{5.25in}|l|}
\caption[castro :  AMR
 parameters]{castro :  AMR
 parameters} \label{table: castro :  AMR
 parameters runtime} \\
%
\hline \multicolumn{1}{|c|}{\textbf{parameter}} & 
       \multicolumn{1}{ c|}{\textbf{description}} & 
       \multicolumn{1}{ c|}{\textbf{default value}} \\ \hline 
\endfirsthead

\multicolumn{3}{c}%
{{\tablename\ \thetable{}---continued}} \\
\hline \multicolumn{1}{|c|}{\textbf{parameter}} & 
       \multicolumn{1}{ c|}{\textbf{description}} & 
       \multicolumn{1}{ c|}{\textbf{default value}} \\ \hline 
\endhead

\multicolumn{3}{|r|}{{\em continued on next page}} \\ \hline
\endfoot

\hline 
\endlastfoot


\rowcolor{tableShade}
\runparamNS{do\_reflux}{castro} &  do we do the hyperbolic reflux at coarse-fine interfaces? & 1 \\
\runparamNS{lin\_limit\_state\_interp}{castro} &  how to do limiting of the state data when interpolating 0: only prevent new extrema 1: preserve linear combinations of state variables & 0 \\
\rowcolor{tableShade}
\runparamNS{state\_interp\_order}{castro} &  highest order used in interpolation & 1 \\
\runparamNS{state\_nghost}{castro} &  Number of ghost zones for state data to have. Note that if you are using radiation, choosing this to be zero will be overridden since radiation needs at least one ghost zone. & 0 \\
\rowcolor{tableShade}
\runparamNS{update\_sources\_after\_reflux}{castro} &  whether to re-compute new-time source terms after a reflux & 1 \\
\runparamNS{use\_custom\_knapsack\_weights}{castro} &  should we have state data for custom load-balancing weighting? & 0 \\


\end{longtable}
\end{center}

} % ends \small


{\small

\renewcommand{\arraystretch}{1.5}
%
\begin{center}
\begin{longtable}{|l|p{5.25in}|l|}
\caption[castro :  diagnostics, I/O
 parameters]{castro :  diagnostics, I/O
 parameters} \label{table: castro :  diagnostics, I/O
 parameters runtime} \\
%
\hline \multicolumn{1}{|c|}{\textbf{parameter}} & 
       \multicolumn{1}{ c|}{\textbf{description}} & 
       \multicolumn{1}{ c|}{\textbf{default value}} \\ \hline 
\endfirsthead

\multicolumn{3}{c}%
{{\tablename\ \thetable{}---continued}} \\
\hline \multicolumn{1}{|c|}{\textbf{parameter}} & 
       \multicolumn{1}{ c|}{\textbf{description}} & 
       \multicolumn{1}{ c|}{\textbf{default value}} \\ \hline 
\endhead

\multicolumn{3}{|r|}{{\em continued on next page}} \\ \hline
\endfoot

\hline 
\endlastfoot


\rowcolor{tableShade}
\runparamNS{hard\_cfl\_limit}{castro} &  abort if we exceed CFL = 1 over the cource of a timestep & 1 \\
\runparamNS{job\_name}{castro} &  a string describing the simulation that will be copied into the plotfile's {\tt job\_info} file & "" \\
\rowcolor{tableShade}
\runparamNS{output\_at\_completion}{castro} &  write a final plotfile and checkpoint upon completion & 1 \\
\runparamNS{print\_fortran\_warnings}{castro} &  display warnings in Fortran90 routines & (0, 1) \\
\rowcolor{tableShade}
\runparamNS{print\_update\_diagnostics}{castro} &  display information about updates to the state (how much mass, momentum, energy added) & (0, 1) \\
\runparamNS{reset\_checkpoint\_step}{castro} &  Do we want to reset the number of steps in the checkpoint? This ONLY takes effect if amr.regrid\_on\_restart = 1 and amr.checkpoint\_on\_restart = 1, (which require that max\_step and stop\_time be less than the value in the checkpoint) and you set it to value greater than this default value. & -1 \\
\rowcolor{tableShade}
\runparamNS{reset\_checkpoint\_time}{castro} &  Do we want to reset the time in the checkpoint? This ONLY takes effect if amr.regrid\_on\_restart = 1 and amr.checkpoint\_on\_restart = 1, (which require that max\_step and stop\_time be less than the value in the checkpoint) and you set it to value greater than this default value. & -1.e200 \\
\runparamNS{show\_center\_of\_mass}{castro} &  display center of mass diagnostics & 0 \\
\rowcolor{tableShade}
\runparamNS{sum\_interval}{castro} &  how often (number of coarse timesteps) to compute integral sums (for runtime diagnostics) & -1 \\
\runparamNS{sum\_per}{castro} &  how often (simulation time) to compute integral sums (for runtime diagnostics) & -1.0e0 \\
\rowcolor{tableShade}
\runparamNS{track\_grid\_losses}{castro} &  calculate losses of material through physical grid boundaries & 0 \\


\end{longtable}
\end{center}

} % ends \small


{\small

\renewcommand{\arraystretch}{1.5}
%
\begin{center}
\begin{longtable}{|l|p{5.25in}|l|}
\caption[castro :  diffusion
 parameters]{castro :  diffusion
 parameters} \label{table: castro :  diffusion
 parameters runtime} \\
%
\hline \multicolumn{1}{|c|}{\textbf{parameter}} & 
       \multicolumn{1}{ c|}{\textbf{description}} & 
       \multicolumn{1}{ c|}{\textbf{default value}} \\ \hline 
\endfirsthead

\multicolumn{3}{c}%
{{\tablename\ \thetable{}---continued}} \\
\hline \multicolumn{1}{|c|}{\textbf{parameter}} & 
       \multicolumn{1}{ c|}{\textbf{description}} & 
       \multicolumn{1}{ c|}{\textbf{default value}} \\ \hline 
\endhead

\multicolumn{3}{|r|}{{\em continued on next page}} \\ \hline
\endfoot

\hline 
\endlastfoot


\rowcolor{tableShade}
\runparamNS{diffuse\_cond\_scale\_fac}{castro} &  scaling factor for conductivity & 1.0 \\
\runparamNS{diffuse\_cutoff\_density}{castro} &  set a cutoff density for diffusion -- we zero the term out below this density & -1.e200 \\
\rowcolor{tableShade}
\runparamNS{diffuse\_enth}{castro} &  enable enthalpy diffusion & 0 \\
\runparamNS{diffuse\_spec}{castro} &  enable species diffusion & 0 \\
\rowcolor{tableShade}
\runparamNS{diffuse\_temp}{castro} &  enable thermal diffusion & 0 \\
\runparamNS{diffuse\_vel}{castro} &  enable velocity diffusion & 0 \\


\end{longtable}
\end{center}

} % ends \small


{\small

\renewcommand{\arraystretch}{1.5}
%
\begin{center}
\begin{longtable}{|l|p{5.25in}|l|}
\caption[castro :  embiggening
 parameters]{castro :  embiggening
 parameters} \label{table: castro :  embiggening
 parameters runtime} \\
%
\hline \multicolumn{1}{|c|}{\textbf{parameter}} & 
       \multicolumn{1}{ c|}{\textbf{description}} & 
       \multicolumn{1}{ c|}{\textbf{default value}} \\ \hline 
\endfirsthead

\multicolumn{3}{c}%
{{\tablename\ \thetable{}---continued}} \\
\hline \multicolumn{1}{|c|}{\textbf{parameter}} & 
       \multicolumn{1}{ c|}{\textbf{description}} & 
       \multicolumn{1}{ c|}{\textbf{default value}} \\ \hline 
\endhead

\multicolumn{3}{|r|}{{\em continued on next page}} \\ \hline
\endfoot

\hline 
\endlastfoot


\rowcolor{tableShade}
\runparamNS{grown\_factor}{castro} &  the factor by which to extend the domain upon restart for embiggening & 1 \\
\runparamNS{star\_at\_center}{castro} &  used with the embiggening routines to determine how to extend the domain & -1 \\


\end{longtable}
\end{center}

} % ends \small


{\small

\renewcommand{\arraystretch}{1.5}
%
\begin{center}
\begin{longtable}{|l|p{5.25in}|l|}
\caption[castro :  gravity and rotation
 parameters]{castro :  gravity and rotation
 parameters} \label{table: castro :  gravity and rotation
 parameters runtime} \\
%
\hline \multicolumn{1}{|c|}{\textbf{parameter}} & 
       \multicolumn{1}{ c|}{\textbf{description}} & 
       \multicolumn{1}{ c|}{\textbf{default value}} \\ \hline 
\endfirsthead

\multicolumn{3}{c}%
{{\tablename\ \thetable{}---continued}} \\
\hline \multicolumn{1}{|c|}{\textbf{parameter}} & 
       \multicolumn{1}{ c|}{\textbf{description}} & 
       \multicolumn{1}{ c|}{\textbf{default value}} \\ \hline 
\endhead

\multicolumn{3}{|r|}{{\em continued on next page}} \\ \hline
\endfoot

\hline 
\endlastfoot


\rowcolor{tableShade}
\runparamNS{do\_grav}{castro} &  permits gravity calculation to be turned on and off & -1 \\
\runparamNS{do\_rotation}{castro} &  permits rotation calculation to be turned on and off & -1 \\
\rowcolor{tableShade}
\runparamNS{grav\_source\_type}{castro} &  determines how the gravitational source term is added to the momentum and energy state variables. & 4 \\
\runparamNS{implicit\_rotation\_update}{castro} &  we can do a implicit solution of the rotation update to allow for better coupling of the Coriolis terms & 1 \\
\rowcolor{tableShade}
\runparamNS{moving\_center}{castro} &  to we recompute the center used for the multipole gravity solve each step? & 0 \\
\runparamNS{point\_mass}{castro} &  mass of the point mass & 0.0 \\
\rowcolor{tableShade}
\runparamNS{point\_mass\_fix\_solution}{castro} &  if we have a central point mass, we can prevent mass from building up in the zones adjacent to it by keeping their density constant and adding their mass to the point mass object & 0 \\
\runparamNS{rot\_axis}{castro} &  the coordinate axis ($x=1$, $y=2$, $z=3$) for the rotation vector & 3 \\
\rowcolor{tableShade}
\runparamNS{rot\_source\_type}{castro} &  determines how the rotation source terms are added to the momentum and energy equations & 4 \\
\runparamNS{rotation\_include\_centrifugal}{castro} &  permits the centrifugal terms in the rotation to be turned on and off & 1 \\
\rowcolor{tableShade}
\runparamNS{rotation\_include\_coriolis}{castro} &  permits the Coriolis terms in the rotation to be turned on and off & 1 \\
\runparamNS{rotation\_include\_domegadt}{castro} &  permits the d(omega)/dt terms in the rotation to be turned on and off & 1 \\
\rowcolor{tableShade}
\runparamNS{rotational\_dPdt}{castro} &  the rotation periods time evolution---this allows the rotation rate to change durning the simulation time & 0.0 \\
\runparamNS{rotational\_period}{castro} &  the rotation period for the corotating frame & -1.e200 \\
\rowcolor{tableShade}
\runparamNS{state\_in\_rotating\_frame}{castro} &  Which reference frame to measure the state variables with respect to. The standard in the literature when using a rotating reference frame is to measure the state variables with respect to an observer fixed in that rotating frame. If this option is disabled by setting it to 0, the state variables will be measured with respect to an observer fixed in the inertial frame (but the frame will still rotate). & 1 \\
\runparamNS{use\_point\_mass}{castro} &  include a central point mass & 1 \\


\end{longtable}
\end{center}

} % ends \small


{\small

\renewcommand{\arraystretch}{1.5}
%
\begin{center}
\begin{longtable}{|l|p{5.25in}|l|}
\caption[castro :  hydrodynamics
 parameters]{castro :  hydrodynamics
 parameters} \label{table: castro :  hydrodynamics
 parameters runtime} \\
%
\hline \multicolumn{1}{|c|}{\textbf{parameter}} & 
       \multicolumn{1}{ c|}{\textbf{description}} & 
       \multicolumn{1}{ c|}{\textbf{default value}} \\ \hline 
\endfirsthead

\multicolumn{3}{c}%
{{\tablename\ \thetable{}---continued}} \\
\hline \multicolumn{1}{|c|}{\textbf{parameter}} & 
       \multicolumn{1}{ c|}{\textbf{description}} & 
       \multicolumn{1}{ c|}{\textbf{default value}} \\ \hline 
\endhead

\multicolumn{3}{|r|}{{\em continued on next page}} \\ \hline
\endfoot

\hline 
\endlastfoot


\rowcolor{tableShade}
\runparamNS{add\_ext\_src}{castro} &  if true, define an additional source term & 0 \\
\runparamNS{allow\_negative\_energy}{castro} &  Whether or not to allow internal energy to be less than zero & 0 \\
\rowcolor{tableShade}
\runparamNS{allow\_small\_energy}{castro} &  Whether or not to allow the internal energy to be less than the internal energy corresponding to small\_temp & 1 \\
\runparamNS{cg\_blend}{castro} &  for the Colella \& Glaz Riemann solver, what to do if we do not converge to a solution for the star state. 0 = do nothing; print iterations and exit 1 = revert to the original guess for p-star 2 = do a bisection search for another 2 * cg\_maxiter iterations. & 2 \\
\rowcolor{tableShade}
\runparamNS{cg\_maxiter}{castro} &  for the Colella \& Glaz Riemann solver, the maximum number of iterations to take when solving for the star state & 12 \\
\runparamNS{cg\_tol}{castro} &  for the Colella \& Glaz Riemann solver, the tolerance to demand in finding the star state & 1.0e-5 \\
\rowcolor{tableShade}
\runparamNS{density\_reset\_method}{castro} &  Which method to use when resetting a negative/small density 1 = Reset to characteristics of adjacent zone with largest density 2 = Use average of all adjacent zones for all state variables 3 = Reset to the original zone state before the hydro update & 1 \\
\runparamNS{difmag}{castro} &  the coefficient of the artificial viscosity & 0.1 \\
\rowcolor{tableShade}
\runparamNS{do\_hydro}{castro} &  permits hydro to be turned on and off for running pure rad problems & -1 \\
\runparamNS{do\_sponge}{castro} &  permits sponge to be turned on and off & 0 \\
\rowcolor{tableShade}
\runparamNS{dual\_energy\_eta1}{castro} &  Threshold value of (E - K) / E such that above eta1, the hydrodynamic pressure is derived from E - K; otherwise, we use the internal energy variable UEINT. & 1.0e0 \\
\runparamNS{dual\_energy\_eta2}{castro} &  Threshold value of (E - K) / E such that above eta2, we update the internal energy variable UEINT to match E - K. Below this, UEINT remains unchanged. & 1.0e-4 \\
\rowcolor{tableShade}
\runparamNS{dual\_energy\_update\_E\_from\_e}{castro} &  Allow internal energy resets and temperature flooring to change the total energy variable UEDEN in addition to the internal energy variable UEINT. & 1 \\
\runparamNS{first\_order\_hydro}{castro} &  set the flattening parameter to zero to force the reconstructed profiles to be flat, resulting in a first-order method & 0 \\
\rowcolor{tableShade}
\runparamNS{fix\_mass\_flux}{castro} &  & 0 \\
\runparamNS{fourth\_order}{castro} &  do we do fourth-order accurate spatial reconstruction for hydro? (requires SDC or MOL integration) & 0 \\
\rowcolor{tableShade}
\runparamNS{hse\_interp\_temp}{castro} &  if we are doing HSE boundary conditions, should we get the temperature via interpolation (using model\_parser) or hold it constant? & 0 \\
\runparamNS{hse\_reflect\_vels}{castro} &  if we are doing HSE boundary conditions, how do we treat the velocity? reflect? or outflow? & 0 \\
\rowcolor{tableShade}
\runparamNS{hse\_zero\_vels}{castro} &  if we are doing HSE boundary conditions, do we zero the velocity? & 0 \\
\runparamNS{hybrid\_hydro}{castro} &  whether to use the hybrid advection scheme that updates z-angular momentum, cylindrical momentum, and azimuthal momentum (3D only) & 0 \\
\rowcolor{tableShade}
\runparamNS{hybrid\_riemann}{castro} &  do we drop from our regular Riemann solver to HLL when we are in shocks to avoid the odd-even decoupling instability? & 0 \\
\runparamNS{limit\_fluxes\_on\_small\_dens}{castro} &  Should we limit the density fluxes so that we do not create small densities? & 0 \\
\rowcolor{tableShade}
\runparamNS{limit\_fourth\_order}{castro} &  do we use a limiter with the fourth-order accurate reconstruction? & 1 \\
\runparamNS{mol\_order}{castro} &  integration order for MOL integration 1 = first order, 2 = second order TVD, 3 = 3rd order TVD, 4 = 4th order RK & 2 \\
\rowcolor{tableShade}
\runparamNS{plm\_iorder}{castro} &  for piecewise linear, reconstruction order to use & 2 \\
\runparamNS{ppm\_predict\_gammae}{castro} &  do we construct $\gamma_e = p/(\rho e) + 1$ and bring it to the interfaces for additional thermodynamic information (this is the Colella \& Glaz technique) or do we use $(\rho e)$ (the classic \castro\ behavior).  Note this also uses $\tau = 1/\rho$ instead of $\rho$. & 0 \\
\rowcolor{tableShade}
\runparamNS{ppm\_reference\_eigenvectors}{castro} &  do we use the reference state in evaluating the eigenvectors? & 0 \\
\runparamNS{ppm\_temp\_fix}{castro} &  various methods of giving temperature a larger role in the reconstruction---see Zingale \& Katz 2015 & 0 \\
\rowcolor{tableShade}
\runparamNS{ppm\_type}{castro} &  reconstruction type: 0: piecewise linear; 1: classic Colella \& Woodward ppm; 2: extrema-preserving ppm & 1 \\
\runparamNS{riemann\_solver}{castro} &  which Riemann solver do we use: 0: Colella, Glaz, \& Ferguson (a two-shock solver); 1: Colella \& Glaz (a two-shock solver) 2: HLLC & 0 \\
\rowcolor{tableShade}
\runparamNS{sdc\_order}{castro} &  integration order for SDC integration valid options are 2 and 4 & 2 \\
\runparamNS{sdc\_solve\_for\_rhoe}{castro} &  do we solve for (rho e) or (rho E) in the SDC nonlinear solve? & 1 \\
\rowcolor{tableShade}
\runparamNS{sdc\_solver}{castro} &  which SDC nonlinear solver to use?  1 = Newton, 2 = VODE, 3 = VODE for first iter & 1 \\
\runparamNS{sdc\_solver\_tol}{castro} &  which SDC nonlinear solver to use?  1 = Newton, 2 = VODE & 1.e-6 \\
\rowcolor{tableShade}
\runparamNS{sdc\_use\_analytic\_jac}{castro} &  do we use the analytic or numerical Jacobian? & 1 \\
\runparamNS{small\_dens}{castro} &  the small density cutoff.  Densities below this value will be reset & -1.e200 \\
\rowcolor{tableShade}
\runparamNS{small\_ener}{castro} &  the small specific internal energy cutoff.  Internal energies below this value will be reset & -1.e200 \\
\runparamNS{small\_pres}{castro} &  the small pressure cutoff.  Pressures below this value will be reset & -1.e200 \\
\rowcolor{tableShade}
\runparamNS{small\_temp}{castro} &  the small temperature cutoff.  Temperatures below this value will be reset & -1.e200 \\
\runparamNS{source\_term\_predictor}{castro} &  extrapolate the source terms (gravity and rotation) to $n+1/2$ timelevel for use in the interface state prediction & 0 \\
\rowcolor{tableShade}
\runparamNS{sponge\_implicit}{castro} &  if we are using the sponge, whether to use the implicit solve for it & 1 \\
\runparamNS{time\_integration\_method}{castro} &  how do we advance in time? 0 = CTU + Strang, 1 = MOL + Strang, 2 = SDC & 0 \\
\rowcolor{tableShade}
\runparamNS{transverse\_reset\_density}{castro} &  if the transverse interface state correction, if the new density is negative, then replace all of the interface quantities with their values without the transverse correction. & 1 \\
\runparamNS{transverse\_reset\_rhoe}{castro} &  if the interface state for $(\rho e)$ is negative after we add the transverse terms, then replace the interface value of $(\rho e)$ with a value constructed from the $(\rho e)$ evolution equation & 0 \\
\rowcolor{tableShade}
\runparamNS{transverse\_use\_eos}{castro} &  after we add the transverse correction to the interface states, replace the predicted pressure with an EOS call (using $e$ and $\rho$). & 0 \\
\runparamNS{use\_eos\_in\_riemann}{castro} &  should we use the EOS in the Riemann solver to ensure thermodynamic consistency? & 0 \\
\rowcolor{tableShade}
\runparamNS{use\_flattening}{castro} &  flatten the reconstructed profiles around shocks to prevent them from becoming too thin & 1 \\
\runparamNS{use\_pslope}{castro} &  for the piecewise linear reconstruction, do we subtract off $(\rho g)$ from the pressure before limiting? & 1 \\
\rowcolor{tableShade}
\runparamNS{use\_reconstructed\_gamma1}{castro} &  should we use a reconstructed version of Gamma_1 in the Riemann solver? or the default zone average (requires SDC or MOL integration, since we do not trace) & 0 \\
\runparamNS{xl\_ext\_bc\_type}{castro} &  if we are doing an external -x boundary condition, who do we interpret it? & "" \\
\rowcolor{tableShade}
\runparamNS{xr\_ext\_bc\_type}{castro} &  if we are doing an external +x boundary condition, who do we interpret it? & "" \\
\runparamNS{yl\_ext\_bc\_type}{castro} &  if we are doing an external -y boundary condition, who do we interpret it? & "" \\
\rowcolor{tableShade}
\runparamNS{yr\_ext\_bc\_type}{castro} &  if we are doing an external +y boundary condition, who do we interpret it? & "" \\
\runparamNS{zl\_ext\_bc\_type}{castro} &  if we are doing an external -z boundary condition, who do we interpret it? & "" \\
\rowcolor{tableShade}
\runparamNS{zr\_ext\_bc\_type}{castro} &  if we are doing an external +z boundary condition, who do we interpret it? & "" \\


\end{longtable}
\end{center}

} % ends \small


{\small

\renewcommand{\arraystretch}{1.5}
%
\begin{center}
\begin{longtable}{|l|p{5.25in}|l|}
\caption[castro :  parallelization
 parameters]{castro :  parallelization
 parameters} \label{table: castro :  parallelization
 parameters runtime} \\
%
\hline \multicolumn{1}{|c|}{\textbf{parameter}} & 
       \multicolumn{1}{ c|}{\textbf{description}} & 
       \multicolumn{1}{ c|}{\textbf{default value}} \\ \hline 
\endfirsthead

\multicolumn{3}{c}%
{{\tablename\ \thetable{}---continued}} \\
\hline \multicolumn{1}{|c|}{\textbf{parameter}} & 
       \multicolumn{1}{ c|}{\textbf{description}} & 
       \multicolumn{1}{ c|}{\textbf{default value}} \\ \hline 
\endhead

\multicolumn{3}{|r|}{{\em continued on next page}} \\ \hline
\endfoot

\hline 
\endlastfoot


\rowcolor{tableShade}
\runparamNS{bndry\_func\_thread\_safe}{castro} &  & 1 \\
\runparamNS{do\_acc}{castro} &  determines whether we use accelerators for specific loops & -1 \\


\end{longtable}
\end{center}

} % ends \small


{\small

\renewcommand{\arraystretch}{1.5}
%
\begin{center}
\begin{longtable}{|l|p{5.25in}|l|}
\caption[castro :  particles
 parameters]{castro :  particles
 parameters} \label{table: castro :  particles
 parameters runtime} \\
%
\hline \multicolumn{1}{|c|}{\textbf{parameter}} & 
       \multicolumn{1}{ c|}{\textbf{description}} & 
       \multicolumn{1}{ c|}{\textbf{default value}} \\ \hline 
\endfirsthead

\multicolumn{3}{c}%
{{\tablename\ \thetable{}---continued}} \\
\hline \multicolumn{1}{|c|}{\textbf{parameter}} & 
       \multicolumn{1}{ c|}{\textbf{description}} & 
       \multicolumn{1}{ c|}{\textbf{default value}} \\ \hline 
\endhead

\multicolumn{3}{|r|}{{\em continued on next page}} \\ \hline
\endfoot

\hline 
\endlastfoot


\rowcolor{tableShade}
\runparamNS{do\_tracer\_particles}{castro} &  permits tracer particle calculation to be turned on and off & 0 \\


\end{longtable}
\end{center}

} % ends \small


{\small

\renewcommand{\arraystretch}{1.5}
%
\begin{center}
\begin{longtable}{|l|p{5.25in}|l|}
\caption[castro :  reactions
 parameters]{castro :  reactions
 parameters} \label{table: castro :  reactions
 parameters runtime} \\
%
\hline \multicolumn{1}{|c|}{\textbf{parameter}} & 
       \multicolumn{1}{ c|}{\textbf{description}} & 
       \multicolumn{1}{ c|}{\textbf{default value}} \\ \hline 
\endfirsthead

\multicolumn{3}{c}%
{{\tablename\ \thetable{}---continued}} \\
\hline \multicolumn{1}{|c|}{\textbf{parameter}} & 
       \multicolumn{1}{ c|}{\textbf{description}} & 
       \multicolumn{1}{ c|}{\textbf{default value}} \\ \hline 
\endhead

\multicolumn{3}{|r|}{{\em continued on next page}} \\ \hline
\endfoot

\hline 
\endlastfoot


\rowcolor{tableShade}
\runparamNS{disable\_shock\_burning}{castro} &  disable burning inside hydrodynamic shock regions & 0 \\
\runparamNS{do\_react}{castro} &  permits reactions to be turned on and off -- mostly for efficiency's sake & -1 \\
\rowcolor{tableShade}
\runparamNS{dtnuc\_X}{castro} &  Limit the timestep based on how much the burning can change the species mass fractions of a zone. The timestep is equal to {\tt dtnuc}  $\cdot\,(X / \dot{X})$. & 1.e200 \\
\runparamNS{dtnuc\_X\_threshold}{castro} &  If we are using the timestep limiter based on changes in $X$, set a threshold on the species abundance below which the limiter is not applied. This helps prevent the timestep from becoming very small due to changes in trace species. & 1.e-3 \\
\rowcolor{tableShade}
\runparamNS{dtnuc\_e}{castro} &  Limit the timestep based on how much the burning can change the internal energy of a zone. The timestep is equal to {\tt dtnuc}  $\cdot\,(e / \dot{e})$. & 1.e200 \\
\runparamNS{dxnuc}{castro} &  limit the zone size based on how much the burning can change the internal energy of a zone. The zone size on the finest level must be smaller than {\tt dxnuc} $\cdot\, c_s\cdot (e / \dot{e})$, where $c_s$ is the sound speed. This ensures that the sound-crossing time is smaller than the nuclear energy injection timescale. & 1.e200 \\
\rowcolor{tableShade}
\runparamNS{dxnuc\_max}{castro} &  Disable limiting based on dxnuc above this threshold. This allows zones that have already ignited or are about to ignite to be de-refined. & 1.e200 \\
\runparamNS{max\_dxnuc\_lev}{castro} &  Disable limiting based on dxnuc above this AMR level. & 30 \\
\rowcolor{tableShade}
\runparamNS{react\_T\_max}{castro} &  maximum temperature for allowing reactions to occur in a zone & 1.e200 \\
\runparamNS{react\_T\_min}{castro} &  minimum temperature for allowing reactions to occur in a zone & 0.0 \\
\rowcolor{tableShade}
\runparamNS{react\_rho\_max}{castro} &  maximum density for allowing reactions to occur in a zone & 1.e200 \\
\runparamNS{react\_rho\_min}{castro} &  minimum density for allowing reactions to occur in a zone & 0.0 \\


\end{longtable}
\end{center}

} % ends \small


{\small

\renewcommand{\arraystretch}{1.5}
%
\begin{center}
\begin{longtable}{|l|p{5.25in}|l|}
\caption[castro :  refinement
 parameters]{castro :  refinement
 parameters} \label{table: castro :  refinement
 parameters runtime} \\
%
\hline \multicolumn{1}{|c|}{\textbf{parameter}} & 
       \multicolumn{1}{ c|}{\textbf{description}} & 
       \multicolumn{1}{ c|}{\textbf{default value}} \\ \hline 
\endfirsthead

\multicolumn{3}{c}%
{{\tablename\ \thetable{}---continued}} \\
\hline \multicolumn{1}{|c|}{\textbf{parameter}} & 
       \multicolumn{1}{ c|}{\textbf{description}} & 
       \multicolumn{1}{ c|}{\textbf{default value}} \\ \hline 
\endhead

\multicolumn{3}{|r|}{{\em continued on next page}} \\ \hline
\endfoot

\hline 
\endlastfoot


\rowcolor{tableShade}
\runparamNS{do\_special\_tagging}{castro} &  & 0 \\
\runparamNS{spherical\_star}{castro} &  & 0 \\


\end{longtable}
\end{center}

} % ends \small


{\small

\renewcommand{\arraystretch}{1.5}
%
\begin{center}
\begin{longtable}{|l|p{5.25in}|l|}
\caption[castro :  timestep control
 parameters]{castro :  timestep control
 parameters} \label{table: castro :  timestep control
 parameters runtime} \\
%
\hline \multicolumn{1}{|c|}{\textbf{parameter}} & 
       \multicolumn{1}{ c|}{\textbf{description}} & 
       \multicolumn{1}{ c|}{\textbf{default value}} \\ \hline 
\endfirsthead

\multicolumn{3}{c}%
{{\tablename\ \thetable{}---continued}} \\
\hline \multicolumn{1}{|c|}{\textbf{parameter}} & 
       \multicolumn{1}{ c|}{\textbf{description}} & 
       \multicolumn{1}{ c|}{\textbf{default value}} \\ \hline 
\endhead

\multicolumn{3}{|r|}{{\em continued on next page}} \\ \hline
\endfoot

\hline 
\endlastfoot


\rowcolor{tableShade}
\runparamNS{cfl}{castro} &  the effective Courant number to use---we will not allow the hydrodynamic waves to cross more than this fraction of a zone over a single timestep & 0.8 \\
\runparamNS{change\_max}{castro} &  the maximum factor by which the timestep can increase from one step to the next. & 1.1 \\
\rowcolor{tableShade}
\runparamNS{clamp\_subcycles}{castro} &  If we do request more than the maximum number of subcycles, should we fail, or should we clamp to that maximum number and perform that many? & 1 \\
\runparamNS{dt\_cutoff}{castro} &  the smallest valid timestep---if we go below this, we abort & 0.0 \\
\rowcolor{tableShade}
\runparamNS{fixed\_dt}{castro} &  a fixed timestep to use for all steps (negative turns it off) & -1.0 \\
\runparamNS{init\_shrink}{castro} &  a factor by which to reduce the first timestep from that requested by the timestep estimators & 1.0 \\
\rowcolor{tableShade}
\runparamNS{initial\_dt}{castro} &  the initial timestep (negative uses the step returned from the timestep constraints) & -1.0 \\
\runparamNS{max\_dt}{castro} &  the largest valid timestep---limit all timesteps to be no larger than this & 1.e200 \\
\rowcolor{tableShade}
\runparamNS{max\_subcycles}{castro} &  Do not permit more subcycled timesteps than this parameter. Set to a negative value to disable this criterion. & 10 \\
\runparamNS{plot\_per\_is\_exact}{castro} &  enforce that the AMR plot interval must be hit exactly & 0 \\
\rowcolor{tableShade}
\runparamNS{retry\_neg\_dens\_factor}{castro} &  If we're doing retries, set the target threshold for changes in density if a retry is triggered by a negative density. If this is set to a negative number then it will disable retries using this criterion. & 1.e-1 \\
\runparamNS{retry\_subcycle\_factor}{castro} &  When performing a retry, the factor to multiply the current timestep by when trying again. & 0.5 \\
\rowcolor{tableShade}
\runparamNS{retry\_tolerance}{castro} &  Tolerance to use when evaluating whether to do a retry. The timestep suggested by the retry will be multiplied by (1 + this factor) before comparing the actual timestep to it. If set to some number slightly larger than zero, then this prevents retries that are caused by small numerical differences. & 0.02 \\
\runparamNS{sdc\_iters}{castro} &  Number of iterations for the SDC advance. & 2 \\
\rowcolor{tableShade}
\runparamNS{small\_plot\_per\_is\_exact}{castro} &  enforce that the AMR small plot interval must be hit exactly & 0 \\
\runparamNS{use\_post\_step\_regrid}{castro} &  Check for a possible post-timestep regrid if certain stability criteria were violated. & 0 \\
\rowcolor{tableShade}
\runparamNS{use\_retry}{castro} &  Retry a timestep if it violated the timestep-limiting criteria over the course of an advance. The criteria will suggest a new timestep that satisfies the criteria, and we will do subcycled timesteps on the same level until we reach the original target time. & 0 \\


\end{longtable}
\end{center}

} % ends \small


\end{landscape}

%


\section{ {\tt diffusion } Namespace}

\label{ch:parameters}


%%%%%%%%%%%%%%%%
% symbol table
%%%%%%%%%%%%%%%%

\begin{landscape}


{\small

\renewcommand{\arraystretch}{1.5}
%
\begin{center}
\begin{longtable}{|l|p{5.25in}|l|}
\caption[diffusion parameters]{diffusion parameters} \label{table: diffusion parameters runtime} \\
%
\hline \multicolumn{1}{|c|}{\textbf{parameter}} & 
       \multicolumn{1}{ c|}{\textbf{description}} & 
       \multicolumn{1}{ c|}{\textbf{default value}} \\ \hline 
\endfirsthead

\multicolumn{3}{c}%
{{\tablename\ \thetable{}---continued}} \\
\hline \multicolumn{1}{|c|}{\textbf{parameter}} & 
       \multicolumn{1}{ c|}{\textbf{description}} & 
       \multicolumn{1}{ c|}{\textbf{default value}} \\ \hline 
\endhead

\multicolumn{3}{|r|}{{\em continued on next page}} \\ \hline
\endfoot

\hline 
\endlastfoot


\rowcolor{tableShade}
\runparamNS{mlmg\_maxorder}{diffusion} &  Use MLMG as the operator & 4 \\
\runparamNS{v}{diffusion} &  the level of verbosity for the diffusion solve (higher number means more output) & 0 \\


\end{longtable}
\end{center}

} % ends \small


\end{landscape}

%


\section{ {\tt gravity } Namespace}

\label{ch:parameters}


%%%%%%%%%%%%%%%%
% symbol table
%%%%%%%%%%%%%%%%

\begin{landscape}


{\small

\renewcommand{\arraystretch}{1.5}
%
\begin{center}
\begin{longtable}{|l|p{5.25in}|l|}
\caption[gravity parameters]{gravity parameters} \label{table: gravity parameters runtime} \\
%
\hline \multicolumn{1}{|c|}{\textbf{parameter}} & 
       \multicolumn{1}{ c|}{\textbf{description}} & 
       \multicolumn{1}{ c|}{\textbf{default value}} \\ \hline 
\endfirsthead

\multicolumn{3}{c}%
{{\tablename\ \thetable{}---continued}} \\
\hline \multicolumn{1}{|c|}{\textbf{parameter}} & 
       \multicolumn{1}{ c|}{\textbf{description}} & 
       \multicolumn{1}{ c|}{\textbf{default value}} \\ \hline 
\endhead

\multicolumn{3}{|r|}{{\em continued on next page}} \\ \hline
\endfoot

\hline 
\endlastfoot


\rowcolor{tableShade}
\runparamNS{const\_grav}{gravity} &  if doing constant gravity, what is the acceleration & 0.0 \\
\runparamNS{direct\_sum\_bcs}{gravity} &  Check if the user wants to compute the boundary conditions using the brute force method.  Default is false, since this method is slow. & 0 \\
\rowcolor{tableShade}
\runparamNS{do\_composite\_phi\_correction}{gravity} &  should we apply a lagged correction to the potential that gets us closer to the composite solution? This makes the resulting fine grid calculation slightly more accurate, at the cost of an additional Poisson solve per timestep. & 1 \\
\runparamNS{drdxfac}{gravity} &  ratio of dr for monopole gravity binning to grid resolution & 1 \\
\rowcolor{tableShade}
\runparamNS{get\_g\_from\_phi}{gravity} &  For non-Poisson gravity, do we want to construct the gravitational acceleration by taking the gradient of the potential, rather than constructing it directly? & 0 \\
\runparamNS{gravity\_type}{gravity} &  what type & "fillme" \\
\rowcolor{tableShade}
\runparamNS{max\_multipole\_order}{gravity} &  the maximum mulitpole order to use for multipole BCs when doing Poisson gravity & 0 \\
\runparamNS{max\_solve\_level}{gravity} &  For all gravity types, we can choose a maximum level for explicitly calculating the gravity and associated potential. Above that level, we interpolate from coarser levels. & MAX\_LEV-1 \\
\rowcolor{tableShade}
\runparamNS{mlmg\_agglomeration}{gravity} &  Do agglomeration? & 1 \\
\runparamNS{mlmg\_consolidation}{gravity} &  & 1 \\
\rowcolor{tableShade}
\runparamNS{mlmg\_max\_fmg\_iter}{gravity} &  how many FMG cycles? & 0 \\
\runparamNS{mlmg\_nsolve}{gravity} &  Do N-Solve? & 0 \\
\rowcolor{tableShade}
\runparamNS{no\_composite}{gravity} &  do we do a composite solve? & 0 \\
\runparamNS{no\_sync}{gravity} &  do we perform the synchronization at coarse-fine interfaces? & 0 \\
\rowcolor{tableShade}
\runparamNS{v}{gravity} &  the level of verbosity for the gravity solve (higher number means more output on the status of the solve / multigrid & 0 \\


\end{longtable}
\end{center}

} % ends \small


\end{landscape}

%


\section{ {\tt particles } Namespace}

\label{ch:parameters}


%%%%%%%%%%%%%%%%
% symbol table
%%%%%%%%%%%%%%%%

\begin{landscape}


{\small

\renewcommand{\arraystretch}{1.5}
%
\begin{center}
\begin{longtable}{|l|p{5.25in}|l|}
\caption[particles parameters]{particles parameters} \label{table: particles parameters runtime} \\
%
\hline \multicolumn{1}{|c|}{\textbf{parameter}} & 
       \multicolumn{1}{ c|}{\textbf{description}} & 
       \multicolumn{1}{ c|}{\textbf{default value}} \\ \hline 
\endfirsthead

\multicolumn{3}{c}%
{{\tablename\ \thetable{}---continued}} \\
\hline \multicolumn{1}{|c|}{\textbf{parameter}} & 
       \multicolumn{1}{ c|}{\textbf{description}} & 
       \multicolumn{1}{ c|}{\textbf{default value}} \\ \hline 
\endhead

\multicolumn{3}{|r|}{{\em continued on next page}} \\ \hline
\endfoot

\hline 
\endlastfoot


\rowcolor{tableShade}
\runparamNS{particle\_init\_file}{particles} &  the name of an input file containing the total particle number and the initial position of each particle. & "" \\
\runparamNS{particle\_output\_file}{particles} &  the name of timestamp files. & "" \\
\rowcolor{tableShade}
\runparamNS{particle\_restart\_file}{particles} &  the name of a file with new particles at restart & "" \\
\runparamNS{restart\_from\_nonparticle\_chkfile}{particles} &  to restart from a checkpoint that was written with {\tt USE\_PARTICLES}=FALSE & 0 \\
\rowcolor{tableShade}
\runparamNS{timestamp\_density}{particles} &  whether the local densities at given positions of particles are stored in output files & 1 \\
\runparamNS{timestamp\_dir}{particles} &  the name of a directory in which timestamp files are stored. & "" \\
\rowcolor{tableShade}
\runparamNS{timestamp\_temperature}{particles} &  whether the local temperatures at given positions of particles are stored in output files & 0 \\
\runparamNS{v}{particles} &  the level of verbosity for the tracer particle (0 or 1) & 0 \\


\end{longtable}
\end{center}

} % ends \small


\end{landscape}

%


