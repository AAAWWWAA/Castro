\label{ch:gravity}

\section{Introduction}

\subsection{Integration Strategy}

\castro\ uses subcycling to integrate levels at different timesteps.
The gravity algorithm needs to respect this.  Self-gravity is computed
via multigrid.  At coarse-fine interfaces, the stencil used in the
Laplacian understands the coarse-fine interface and is different than
the stencil used in the interior.

There are two types of
solves that we discuss with AMR:
\begin{itemize}
\item {\em composite solve} : This is a multilevel solve, starting at
  a coarse level (usually level 0) and solving for the potential on
  all levels up to the finest level.

\item {\em level solve} : This solves for the potential only on
 a particular level.  Finer levels are ignored.  At coarse-fine
 interfaces, the data from the coarse levels acts as Dirichlet 
 boundary conditions for the current-level-solve.
\end{itemize}

The overall integration strategy is unchanged from the discussion in
\cite{castro_I}.  Briefly:
\begin{itemize}
\item At the beginning of a simulation, we do a multilevel composite
  solve (if \runparam{gravity.no\_composite}{\tt = 0}).  

  We also do a multilevel composite solve after each regrid.

\item The old-time gravity on the coarse level is defined based on
  this composite solve, but we also do a level solve on the coarse
  level, and use it to determine the difference between the composite
  solve and the level solve, and store that in a \multifab.

\item After the hydro advance on the coarse level, we do another level
  solve, and use the (level solve $-$ compositive solve) as a lagged
  predictor of how much we need to add back to that level solve to get
  an effective new-time value for phi on the coarse level, and that's
  what defines the phi used for the new-time gravity

\item Then we do the fine grid timestep(s), each using the same
  strategy

\item At an AMR synchronization step across levels (see Section \ref{sec:amr_synchronization}
  for a description of when these synchronizations occur), if we're choosing
  to synchronize the gravitational field across levels (\runparam{gravity.no\_sync}{\tt = 0})
  we then do a solve starting from
  the coarse grid that adjusts for the mismatch between the fine-grid
  phi and the coarse-grid phi, as well as the mismatch between the
  fine-grid density fluxes and the coarse-grid density fluxes, and add
  the resulting sync solve phi to both the coarse and the fine level

  Thus, to within the gravity error tolerance, you get the same final
  result as if you had done a full composite solve at the end of the
  timestep (assuming \runparam{gravity.no\_sync}{\tt = 0}).
\end{itemize}

If you do \runparam{gravity.no\_composite}{\tt = 1}, then you never do a full
multilevel solve, and the gravity on any level is defined only by the
solve on that level.  The only time this would be appropriate is if
the fine level(s) cover essentially all of the mass on the grid for
all time.




\subsection{Controls}
\castro\ can incorporate gravity as a constant, monopole approximation,
or a full Poisson solve.  To enable gravity in the code, set:
\begin{verbatim}
USE_GRAV = TRUE
\end{verbatim}
in the {\tt GNUmakefile}, and then turn it on in the {\tt inputs} file
via {\tt castro.do\_grav = 1}.  If you want to incorporate a point mass
(through {\tt castro.point\_mass}), you must have
\begin{verbatim}
USE_POINTMASS = TRUE
\end{verbatim}
in the {\tt GNUmakefile}.

There are currently four options for how gravity is calculated,
controlled by setting \runparam{gravity.gravity\_type}.  The options are
{\tt ConstantGrav, PoissonGrav, Monopole Grav} or {\tt
  PrescribedGrav}.  Again, these are only relevant if {\tt USE\_GRAV =
  TRUE} in the {\tt GNUmakefile} and {\tt castro.do\_grav = 1} in the
{\tt inputs} file.  If both of these are set then the user is required
to specify the gravity type in the inputs file or the program will
abort.

Some additional notes:
\begin{itemize}

\item For the full Poisson solver
  (\runparam{gravity.gravity\_type}{\tt = PoissonGrav}), the behavior
  of the full Poisson solve / multigrid solver is controlled by
  \runparam{gravity.no\_sync} and \runparam{gravity.no\_composite}.

\item For isolated boundary conditions, and when
  \runparam{gravity.gravity\_type}{\tt = PoissonGrav}, the parameters
           {\tt gravity.max\_multipole\_order} and
           \runparam{gravity.direct\_sum\_bcs} control the accuracy of
           the Dirichlet boundary conditions.  These are described in
           Section \ref{sec-poisson-3d-bcs}.

\item For {\tt MonopoleGrav}, in 1D we must have {\tt coord\_sys = 2}, and in
  2D we must have {\tt coord\_sys = 1}.
\end{itemize}

The following parameters apply to gravity
solves:
\begin{itemize}
\item \runparam{gravity.gravity\_type} : how should we calculate gravity?
  Can be {\tt ConstantGrav}, {\tt PoissonGrav}, {\tt MonopoleGrav}, or
  {\tt PrescribedGrav}

\item \runparam{gravity.const\_grav} : if {\tt gravity.gravity\_type =
  ConstantGrav}, set the value of constant gravity (default: 0.0)

\item \runparam{gravity.no\_sync} : {\tt gravity.gravity\_type =
  PoissonGrav}, do we perform the ``sync solve"? (0 or 1; default: 0)

\item \runparam{gravity.no\_composite} : if {\tt gravity.gravity\_type
  = PoissonGrav}, whether to perform a composite solve (0 or 1;
  default: 0)

\item \runparam{gravity.max\_solve\_level} : maximum level to solve
  for $\phi$ and $\mathbf{g}$; above this level, interpolate from
  below (default: ${\tt MAX\_LEV} - 1$)

\item \runparam{gravity.abs\_tol} : if {\tt gravity.gravity\_type =
  PoissonGrav}, this is the absolute tolerance for the Poisson
  solve. You can specify a single value for this tolerane (or do
  nothing, and get a reasonable default value), and then the absolute
  tolerance used by the multigrid solve is $\text{abs\_tol} \times
  4\pi G\, \rho_{\text{max}}$ where $\rho_{\text{max}}$ is the maximum
  value of the density on the domain.  On fine levels, this absolute
  tolerance is multiplied by $\text{ref\_ratio}^2$ to account for the
  change in the absolute scale of the Laplacian operator.  You can
  also specify an array of values for \text{abs\_tol}, one for each
  possible level in the simulation, and then the scaling by
  $\text{ref\_ratio}^2$ is not applied.

\item \runparam{\tt gravity.rel\_tol} : if {\tt gravity.gravity\_type
  = PoissonGrav}, this is the relative tolerance for the Poisson
  solve. By default it is zero. You can specify a single value for
  this tolerance and it will apply on every level, or you can specify
  an array of values for \text{rel\_tol}, one for each possible level
  in the simulation. This replaces the old parameter
  \runparam{gravity.ml\_tol}.

\item \runparam{gravity.max\_multipole\_order} : if {\tt
  gravity.gravity\_type = PoissonGrav}, this is the max $\ell$ value
  to use for multipole BCs (must be $\geq 0$; default: 0)

\item \runparam{gravity.direct\_sum\_bcs} : if {\tt
  gravity.gravity\_type = PoissonGrav}, evaluate BCs using exact sum
  (0 or 1; default: 0)

\item \runparam{gravity.drdxfac} : ratio of dr for monopole gravity
  binning to grid resolution
\end{itemize}

The follow parameters affect the coupling of hydro and gravity:
\begin{itemize}
\item \runparam{castro.do\_grav} : turn on/off gravity

\item \runparam{castro.moving\_center} : do we recompute the center
   used for the multipole gravity solver each step?

\item \runparam{castro.point\_mass} : point mass at the center of the star
  (must be $\geq 0$; default: 0.0)
\end{itemize}

Note that in the following, {\tt MAX\_LEV} is a hard-coded parameter
in {\tt Source/Gravity.cpp} which is currently set to {\tt 15}.  It
determines how many levels can be tracked by the {\tt Gravity} object.


\section{Types of Approximations}

\subsection{{\tt ConstantGrav}}

Gravity can be defined as constant in direction and magnitude,
defined in the inputs file by

\noindent {\tt gravity.const\_grav = -9.8}

for example, to set the gravity to have magnitude $9.8$ in the
negative $y$-direction if in 2D, negative $z$-direction if in 3-D.
The actual setting is done in {\tt Gravity.cpp} as:
\begin{lstlisting}
 grav.setVal(const_grav, BL_SPACEDIM-1, 1, ng);
\end{lstlisting}

Note that at present we do not fill the gravitational potential $\phi$ in
this mode; it will be set to zero.

Note: {\tt ConstantGrav} can only be used along a Cartesian direction
(vertical for 2D axisymmetric).


\subsection{{\tt MonopoleGrav}}
\label{sec-monopole-grav}

{\tt MonopoleGrav} integrates the mass distribution on the grid
in spherical shells, defining an {\tt enclosed mass} and uses this
to compute the gravitational potential and acceleration in a 
spherically-symmetric fashion.

\begin{itemize}

\item In 1D spherical coordinates we compute
  \begin{equation}
    g(r) = -\frac{G M_{\rm enclosed}}{ r^2}
  \end{equation}
  where $M_{\rm enclosed}$ is calculated from the density at the time
  of the call.  

  For levels above the coarsest level we define the extent of that
  level's radial arrays as ranging from the center of the star ($r=0$)
  to the cell at that level farthest away from the origin.  If there
  are gaps between fine grids in that range then we interpolate the
  density from a coarser level in order to construct a continuous
  density profile.  We note that the location of values in the density
  profile and in the gravitational field exactly match the location of
  data at that level so there is no need to interpolate between points
  when mapping the 1D radial profile of $g$ back onto the original
  grid.

\item In 2D or 3D we compute a 1D radial average of density and use
  this to compute gravity as a one-dimensional integral, then
  interpolate the gravity vector back onto the Cartesian grid
  cells. At the coarsest level we define the extent of the 1D arrays
  as ranging from the center of the star to the farthest possible
  point in the grid (plus a few extra cells so that we can fill ghost
  cell values of gravity).  At finer levels we first define a single
  box that contains all boxes on that fine level, then we interpolate
  density from coarser levels as needed to fill the value of density
  at every fine cell in that box.  The extent of the radial array is
  from the center of the star to the {\em nearest} cell on one of the
  faces of the single box.  This ensures that all cells at that
  maximum radius of the array are contained in this box.

We then average the density onto a 1D radial array.  We note that
there is a mapping from the Cartesian cells to the radial array and
back; unlike the 1D case this requires interpolation. We use quadratic
interpolation with limiting so that the interpolation does not create
new maxima or minima.

The default resolution of the radial arrays at a level is the grid
cell spacing at that level, i.e., $\Delta r = \Delta x$. O For
increased accuracy, one can define \runparam{gravity.drdxfac} as a number
greater than $1$ ($2$ or $4$ are recommended) and the spacing of the
radial array will then satisfy $\Delta x / \Delta r = $ {\tt drdxfac}.
Individual Cartesian grid cells are subdivided by {\tt drdxfac} in
each coordinate direction for the purposing of averaging the density,
and the integration that creates $g$ is done at the finer resolution
of the new $\Delta r$.

Note that the center of the star is defined in the subroutine {\tt PROBINIT}
and the radius is computed as the distance from that center.

\MarginPar{there is an additional correction at the corners in {\tt
    make\_radial\_grav} that accounts for the volume in a shell that
  is not part of the grid}

\end{itemize}


{\color{red} What about the potential in this case? when does {\tt
    make\_radial\_phi} come into play?}


\subsection{{\tt PoissonGrav}}

The most general case is a self-induced gravitational field,
\begin{equation}
\mathbf{g}(\mathbf{x},t) = \nabla \phi
\end{equation}
where $\phi$ is defined by solving
\begin{equation}
\mathbf{\Delta} \phi = 4 \pi G \rho .\label{eq:Self Gravity}
\end{equation}

We only allow {\tt PoissonGrav} in 2D or 3D because in 1D, computing
the monopole approximation in spherical coordinates is faster and more
accurate than solving the Poisson equation.

\subsubsection{Poisson Boundary Conditions: 2D}

In 2D, if boundary conditions are not periodic in both directions, we
use a monopole approximation at the coarsest level. This involves
computing an effective 1D radial density profile (on {\tt level =
  0} only), integrating it outwards from the center to get the
gravitational acceleration $\mathbf{g}$, and then integrating $g$
outwards from the center to get $\phi$ (using $\phi(0) = 0$ as a
boundary condition, since no mass is enclosed at $r = 0$). For more
details, see Section \ref{sec-monopole-grav}.

\subsubsection{Poisson Boundary Conditions: 3D}\label{sec-poisson-3d-bcs}

The following describes methods for doing isolated boundary
conditions.  The best reference for \castro's implementation of this
is \cite{katz:2016}.

\begin{itemize}
\item \textbf{Multipole Expansion}

In 3D, by default, we use a multipole expansion to estimate the value
of the boundary conditions. According to, for example, Jackson's
\textit{Classical Electrodynamics} (with the corresponding change to
Poisson's equation for electric charges and gravitational
''charges''), an expansion in spherical harmonics for $\phi$ is
\begin{equation}
  \phi(\mathbf{x}) = -G\sum_{l=0}^{\infty}\sum_{m=-l}^{l} \frac{4\pi}{2l + 1} q_{lm} \frac{Y_{lm}(\theta,\phi)}{r^{l+1}}, \label{spherical_harmonic_expansion}
\end{equation}
The origin of the coordinate system is taken to be the \texttt{center}
variable, that must be declared and stored in the \texttt{probdata}
module in your project directory. The validity of the expansion used
here is based on the assumption that a sphere centered on
\texttt{center}, of radius approximately equal to the size of half the
domain, would enclose all of the mass. Furthermore, the lowest order
terms in the expansion capture further and further departures from
spherical symmetry. Therefore, it is crucial that \texttt{center} be
near the center of mass of the system, for this approach to achieve
good results.

The multipole moments $q_{lm}$ can be calculated by expanding the
Green's function for the Poisson equation as a series of spherical
harmonics, which yields
\begin{equation}
  q_{lm} = \int Y^*_{lm}(\theta^\prime, \phi^\prime)\, {r^\prime}^l \rho(\mathbf{x}^\prime)\, d^3x^\prime. \label{multipole_moments_original}
\end{equation}
Some simplification of Equation \ref{spherical_harmonic_expansion} can
be achieved by using the addition theorem for spherical harmonics:
\begin{align}
  &\frac{4\pi}{2l+1} \sum_{m=-l}^{l} Y^*_{lm}(\theta^\prime,\phi^\prime)\, Y_{lm}(\theta, \phi) = P_l(\text{cos}\, \theta) P_l(\text{cos}\, \theta^\prime) \notag \\
  &\ \ + 2 \sum_{m=1}^{l} \frac{(l-m)!}{(l+m)!} P_{l}^{m}(\text{cos}\, \theta)\, P_{l}^{m}(\text{cos}\, \theta^\prime)\, \left[\text{cos}(m\phi)\, \text{cos}(m\phi^\prime) + \text{sin}(m\phi)\, \text{sin}(m\phi^\prime)\right].
\end{align}
Here the $P_{l}^{m}$ are the associated Legendre polynomials and the
$P_l$ are the Legendre polynomials. After some algebraic
simplification, the potential outside of the mass distribution can be
written in the following way:
\begin{equation}
  \phi(\mathbf{x}) \approx -G\sum_{l=0}^{l_{\text{max}}} \left[Q_l^{(0)} \frac{P_l(\text{cos}\, \theta)}{r^{l+1}} + \sum_{m = 1}^{l}\left[ Q_{lm}^{(C)}\, \text{cos}(m\phi) + Q_{lm}^{(S)}\, \text{sin}(m\phi)\right] \frac{P_{l}^{m}(\text{cos}\, \theta)}{r^{l+1}} \right].
\end{equation}
The modified multipole moments are:
\begin{align}
  Q_l^{(0)}   &= \int P_l(\text{cos}\, \theta^\prime)\, {r^{\prime}}^l \rho(\mathbf{x}^\prime)\, d^3 x^\prime \\
  Q_{lm}^{(C)} &= 2\frac{(l-m)!}{(l+m)!} \int P_{l}^{m}(\text{cos}\, \theta^\prime)\, \text{cos}(m\phi^\prime)\, {r^\prime}^l \rho(\mathbf{x}^\prime)\, d^3 x^\prime \\
  Q_{lm}^{(S)} &= 2\frac{(l-m)!}{(l+m)!} \int P_{l}^{m}(\text{cos}\, \theta^\prime)\, \text{sin}(m\phi^\prime)\, {r^\prime}^l \rho(\mathbf{x}^\prime)\, d^3 x^\prime.
\end{align}
Our strategy for the multipole boundary conditions, then, is to pick
some value $l_{\text{max}}$ that is of sufficiently high order to
capture the distribution of mass on the grid, evaluate the discretized
analog of the modified multipole moments for $0 \leq l \leq
l_{\text{max}}$ and $1 \leq m \leq l$, and then directly compute the
value of the potential on all of the boundary zones. This is
ultimately an $\mathcal{O}(N^3)$ operation, the same order as the
monopole approximation, and the wall time required to calculate the
boundary conditions will depend on the chosen value of
$l_{\text{max}}$.

The number of $l$ values calculated is controlled by
\runparam{gravity.max\_multipole\_order} in your inputs file. By
default, it is set to \texttt{0}, which means that a monopole
approximation is used. There is currently a hard-coded limit of
$l_{\text{max}} = 50$. This is because the method used to generate the
Legendre polynomials is not numerically stable for arbitrary $l$
(because the polynomials get very large, for large enough $l$).

\item \textbf{Direct Sum}

Up to truncation error caused by the discretization itself, the
boundary values for the potential can be computed exactly by a direct
sum over all cells in the grid. Suppose I consider some ghost cell
outside of the grid, at location $\mathbf{r}^\prime \equiv (x^\prime,
y^\prime, z^\prime)$. By the principle of linear superposition as
applied to the gravitational potential,
\begin{equation}
  \phi(\mathbf{r}^\prime) = \sum_{\text{ijk}} \frac{-G \rho_{\text{ijk}}\, \Delta V_{\text{ijk}}}{\left[(x - x^\prime)^2 + (y - y^\prime)^2 + (z - z^\prime)^2\right]^{1/2}},
\end{equation}
where $x = x(i)$, $y = y(j)$ and $z = z(k)$ are constructed in the
usual sense from the zone indices. The sum here runs over every cell
in the physical domain (that is, the calculation is $\mathcal{O}(N^3)$
for each boundary cell). There are $6N^2$ ghost cells needed for the
Poisson solve (since there are six physical faces of the domain), so
the total cost of this operation is $\mathcal{O}(N^5)$ (this only
operates on the coarse grid, at present). In practice, we use the
domain decomposition inherent in the code to implement this solve: for
the grids living on any MPI task, we create six $N^2$ arrays
representing each of those faces, and then iterate over every cell on
each of those grids, and compute their respective contributions to all
of the faces. Then, we do a global reduce to add up the contributions
from all cells together. Finally, we place the boundary condition
terms appropriate for each grid onto its respective cells.

This is quite expensive even for reasonable sized domains, so this
option is recommended only for analysis purposes, to check if the
other methods are producing accurate results. It can be enabled by
setting \runparam{gravity.direct\_sum\_bcs}{\tt = 1} in your inputs file.

\end{itemize}


\subsection{{\tt PrescribedGrav}}

With {\tt PrescribedGrav}\footnote{Note: The {\tt PrescribedGrav}
  option and text here were contributed by Jan Frederik Engels of
  University of Gottingen.}, gravity can be defined as a function that
is specified by the user.  The option is allowed in 2D and 3D.  To
define the gravity vector, copy {\tt prescribe\_grav\_nd.f90} from
{\tt Src\_nd} to your run directory.  The makefile system will always
choose this local copy of the file over the one in another directory.
Then define the components of gravity inside a loop over the grid
inside the file.  If your problem uses a radial gravity in the form
$g(r)$, you can simply adapt {\tt
  ca\_prescribe\_grav\_gravityprofile}, otherwise you will have to
adapt {\bf ca\_prescribe\_grav}, both are located in {\tt
  prescribed\_grav\_nd.90}.


\subsection{Point Mass}

Pointmass gravity works with all other forms of gravity, it is not a
separate option.  Since the Poisson equation is linear in potential
(and its derivative, the acceleration, is also linear), the point mass
option works by adding the gravitational acceleration of the point
mass onto the acceleration from whatever other gravity type is under
in the simulation.

Note that point mass can be $< 0$.

A useful option is {\tt point\_mass\_fix\_solution}.  If set to {\tt
  1}, then it takes all zones that are adjacent to the location of the
{\tt center} variable and keeps their density constant.  Any changes
in density that occur after a hydro update in those zones are reset,
and the mass deleted is added to the pointmass.  (If there is
expansion, and the density lowers, then the point mass is reduced and
the mass is added back to the grid).  This calculation is done in {\tt
  pm\_compute\_delta\_mass()} in {\tt
  Source/Src\_nd/pointmass\_nd.f90}.



\section{GR correction}


In the cases of compact objects or very massive stars, the general
relativity (GR) effect starts to play a role\footnote{Note: The GR
  code and text here were contributed by Ken Chen of Univ. of
  Minnesota.}.  First, we consider the hydrostatic equilibrium due to
effects of GR then derive GR-correction term for Newtonian gravity.
The correction term is applied to the monopole approximation only when
{\tt USE\_GR = TRUE} is set in the {\tt GNUmakefile}.

The formulae of GR-correction here are based on \cite{grbk1}. For
detailed physics, please refer to \cite{grbk2}. For describing very
strong gravitational field, we need to use Einstein field equations
\begin{equation}\label{field}
R_{ik}-\frac{1}{2}g_{ik}R=\frac{\kappa}{c^{2}}T_{ik} \quad , \quad
\kappa=\frac{8\pi G}{c^{2}}\quad ,
\end{equation}
where $R_{ik}$ is the Ricci tensor, $g_{ik}$ is the metric tensor, $R$
is the Riemann curvature, $c$ is the speed of light and $G$ is
gravitational constant. $T_{ik}$ is the energy momentum tensor, which
for ideal gas has only the non-vanishing components $T_{00}$ =
$\varrho c^2$ , $T_{11}$ = $T_{22}$ = $T_{33}$ = $P$ ( contains rest
mass and energy density, $P$ is pressure). We are interested in
spherically symmetric mass distribution. Then the line element $ds$
for given spherical coordinate $(r, \vartheta, \varphi)$ has the
general form
\begin{equation}\label{metric}
  ds^{2} = e^{\nu}c^{2}dt^{2}-e^{\lambda}dr^{2}-r^{2}(d\vartheta^{2}+\sin^{2}
  \vartheta d\varphi) \quad ,
\end{equation}
with $\nu = \nu(r)$, $\lambda = \lambda(r)$. Now we can put the
expression of $T_{ik}$ and $ds$ into (\ref{field}), then field
equations can be reduced to 3 ordinary differential equations:
\begin{equation}\label{diff1}
   \frac{\kappa P}{c^{2}} =
   e^{-\lambda}\left (\frac{\nu^{\prime}}{r}+\frac{1}{r^{2}} \right )-\frac{1}{r^{2}}
   \quad ,
\end{equation}
\begin{equation}\label{diff2}
  \frac{\kappa P}{c^{2}} =
  \frac{1}{2}e^{-\lambda}\left (\nu^{\prime\prime}+\frac{1}{2}{\nu^{\prime}}^{2}+\frac{\nu^
    {\prime}-\lambda^{\prime}}{r}
   -\frac{\nu^{\prime}\lambda^{\prime}}{2} \right ) \quad ,
\end{equation}
\begin{equation}\label{diff3}
  \kappa \varrho =
  e^{-\lambda}\left (\frac{\lambda^{\prime}}{r}-\frac{1}{r^{2}}\right )+\frac{1}{r^{2}} \quad ,
\end{equation}
where primes means the derivatives with respect to $r$. After
multiplying with $4\pi r^2$, (\ref{diff3}) can be integrated and
yields
\begin{equation}\label{gmass1}
  \kappa m = 4\pi r (1-e^{-\lambda}) \quad ,
\end{equation}
the $m$ is called ``gravitational mass'' inside r defined as
\begin{equation}\label{gmass2}
  m = \int_{0}^{r}4\pi r^{2}  \varrho dr\quad .
\end{equation}
For the $r = R$, $m$ becomes the mass $M$ of the star. $M$ contains
not only the rest mass but the whole energy (divided by $c^2$), that
includes the internal and gravitational energy. So the $\varrho =
\varrho_0 +U/c^2$ contains the whole energy density $U$ and rest-mass
density $\varrho_0$.  Differentiation of (\ref{diff1}) with respect to
$r$ gives $P = P^{\prime}(\lambda,\lambda^{\prime},
\nu,\nu^{\prime},r)$, where
$\lambda,\lambda^{\prime},\nu,\nu^{\prime}$ can be eliminated by
(\ref{diff1}), (\ref{diff2}), (\ref{diff3}). Finally we reach
\textit{Tolman-Oppenheinmer-Volkoff(TOV)} equation for hydrostatic
equilibrium in general relativity:
\begin{equation}\label{tov}
  \frac{dP}{dr} = -\frac{Gm}{r^{2}}\varrho \left (1+\frac{P}{\varrho
    c^{2}}\right )\left (1+\frac{4\pi r^3 P}{m c^{2}}\right ) \left (1-\frac{2Gm}{r c^{2}} \right)^{-1} \quad .
\end{equation}
For Newtonian case $c^2 \rightarrow  \infty $, it reverts to usual form
\begin{equation}\label{newton}
  \frac{dP}{dr} = -\frac{Gm}{r^{2}}\varrho \quad .
\end{equation}
Now we take effective monopole gravity as
\begin{equation}\label{tov2}
\tilde{g} = -\frac{Gm}{r^{2}} (1+\frac{P}{\varrho
  c^{2}})(1+\frac{4\pi r^3 P}{m c^{2}}) (1-\frac{2Gm}{r c^{2}})^{-1}  \quad .
\end{equation}
For general situations, we neglect the $U/c^2$ and potential energy in
m because they are usually much smaller than $\varrho_0$. Only when
$T$ reaches $10^{13} K$ ($KT \approx m_{p} c^2$, $m_p$ is proton mass)
before it really makes a difference. So (\ref{tov2}) can be expressed
as
\begin{equation}\label{tov3}
  \tilde{g} = -\frac{GM_{\rm enclosed}}{r^{2}} \left (1+\frac{P}{\varrho
    c^{2}} \right )\left (1+\frac{4\pi r^3 P}{M_{\rm enclosed} c^{2}} \right ) \left (1-\frac{2GM_{\rm enclosed}}{r c^{2}} \right )^{-1} \quad ,
\end{equation}
where $M_{enclosed}$ has the same meaning as with the {\tt
  MonopoleGrav} approximation.



\section{Hydrodynamics Source Terms}

There are several options to incorporate the effects of gravity into
the hydrodynamics system.  The main parameter here is {\tt
  castro.grav\_source\_type}.

\begin{itemize}

\item {\tt castro.grav\_source\_type = 1} : we use a
  standard predictor-corrector formalism for updating the momentum and
  energy. Specifically, our first update is equal to $\Delta t \times
  \mathbf{S}^n$ , where $\mathbf{S}^n$ is the value of the source
  terms at the old-time (which is usually called time-level $n$).  At
  the end of the timestep, we do a corrector step where we subtract
  off $\Delta t / 2 \times \mathbf{S}^n$ and add on $\Delta t / 2
  \times \mathbf{S}^{n+1}$, so that at the end of the timestep the
  source term is properly time centered.

\item {\tt castro.grav\_source\_type = 2} : we do something very
  similar to {\tt 1}. The major difference is that when evaluating the
  energy source term at the new time (which is equal to $\mathbf{u}
  \cdot \mathbf{S}^{n+1}_{\rho \mathbf{u}}$, where the latter is the
  momentum source term evaluated at the new time), we first update the
  momentum, rather than using the value of $\mathbf{u}$ before
  entering the gravity source terms. This permits a tighter coupling
  between the momentum and energy update and we have seen that it
  usually results in a more accurate evolution.

\item {\tt castro.grav\_source\_type = 3} : we do the same momentum
  update as the previous two, but for the energy update, we put all of
  the work into updating the kinetic energy alone. In particular, we
  explicitly ensure that $(rho e)$ maintains the same, and update
  $(rho K)$ with the work due to gravity, adding the new kinetic
  energy to the old internal energy to determine the final total gas
  energy.  The physical motivation is that work should be done on the
  velocity, and should not directly update the temperature---only
  indirectly through things like shocks.

\item {\tt castro.grav\_source\_type = 4} : the energy update is done
  in a ``conservative'' fashion.  The previous methods all evaluate
  the value of the source term at the cell center, but this method
  evaluates the change in energy at cell edges, using the
  hydrodynamical mass fluxes, permitting total energy to be conserved
  (excluding possible losses at open domain boundaries).  See
  \cite{katzthesis} for some more details.
\end{itemize}
