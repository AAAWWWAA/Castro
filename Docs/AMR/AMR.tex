Our approach to adaptive refinement in CASTRO uses a nested
hierarchy of logically-rectangular grids with simultaneous refinement
of the grids in both space and time.  The integration algorithm on the grid hierarchy
is a recursive procedure in which coarse grids are advanced in time,
fine grids are advanced multiple steps to reach the same time
as the coarse grids and the data at different levels are then synchronized.

During the regridding step, increasingly finer grids
are recursively embedded in coarse grids until the solution is
sufficiently resolved.  An error estimation procedure based on
user-specified criteria (described in Section \ref{sec:tagging}) 
evaluates where additional refinement is needed
and grid generation procedures dynamically create or
remove rectangular fine grid patches as resolution requirements change.

A good introduction to the style of AMR used here is in Lecture 1
of the Adaptive Mesh Refinement Short Course at
https://ccse.lbl.gov/people/jbb/index.html

\section{Tagging for Refinement}
\label{sec:tagging}

CASTRO determines what zones should be tagged for refinement at the next 
regridding step by using a set of built-in routines that test on 
quantities such as the density and pressure and determining whether the 
quantities themselves or their gradients pass a user-specified threshold. 
This may then be extended if {\tt amr.n\_error\_buf} $> 0$ to a certain number 
of zones beyond these tagged zones. This section describes the process 
by which zones are tagged, and describes how to add customized tagging criteria.

The routines for tagging cells are located in the {\tt Tagging\_nd.f90} file 
in the {\tt Src\_nd} directory. (These are dimension-agnostic routines that loop 
over all three dimensional indices even for 1D or 2D problems.) The main routines are 
{\tt ca\_denerror}, {\tt ca\_temperror}, {\tt ca\_presserror}, {\tt ca\_velerror}, 
and {\tt ca\_raderror}. They refine based on density, temperature, pressure, 
velocity, and radiation (if enabled), respectively. The same approach is used 
for all of them. As an example, we consider the density tagging routine. There 
are four parameters that control tagging. If the density in a zone is greater than 
the user-specified parameter {\tt denerr}, then that zone will be tagged 
for refinement, but only if the current AMR level is less than the 
user-specified parameter {\tt max\_denerr\_lev}. Similarly, 
if the absolute density gradient between a zone and any adjacent zone 
is greater than the user-specified parameter {\tt dengrad}, that zone 
will be tagged for refinement, but only if we are currently on a level 
below {\tt max\_dengrad\_lev}. Note that setting {\tt denerr} alone will 
not do anything; you'll need to set {\tt max\_dengrad\_lev} $>= 1$ for
this to have any effect.

All four of these parameters are set in the {\tt \&tagging} namelist 
in your {\tt probin} file. If left unmodified, they default to a value 
that means we will never tag. The complete set of parameters that can 
be controlled this way is the following:
\begin{multicols}{5}
\begin{itemize}
  \setlength{\itemindent}{-3em}
  \item {\tt denerr}
  \item {\tt max\_denerr\_lev}
  \item {\tt dengrad}
  \item {\tt max\_dengrad\_lev}
  \item {\tt temperr}
  \item {\tt max\_temperr\_lev}
  \item {\tt tempgrad}
  \item {\tt max\_tempgrad\_lev}
  \item {\tt velerr}
  \item {\tt max\_velerr\_lev}
  \item {\tt velgrad}
  \item {\tt max\_velgrad\_lev}
  \item {\tt presserr}
  \item {\tt max\_presserr\_lev}
  \item {\tt pressgrad}
  \item {\tt max\_pressgrad\_lev}
  \item {\tt raderr}
  \item {\tt max\_raderr\_lev}
  \item {\tt radgrad}
  \item {\tt max\_radgrad\_lev}
\end{itemize}
\end{multicols}
Since there are multiple algorithms for determining 
whether a zone is tagged or not, it is worthwhile to specify 
in detail what is happening to a zone in the code during this step. 
We show this in the following pseudocode section. A zone 
is tagged if the variable {\tt itag = SET}, and is not tagged 
if {\tt itag = CLEAR} (these are mapped to 1 and 0, respectively).

\begin{verbatim}
itag = CLEAR

for errfunc[k] from k = 1 ... N
    // Three possibilities for itag: SET or CLEAR or remaining unchanged
    call errfunc[k](itag)  
end for
\end{verbatim}

In particular, notice that there is an order dependence of this operation; if {\tt errfunc[2]} 
{\tt CLEAR}s a zone and then {\tt errfunc[3]} {\tt SET}s that zone, the final operation will 
be to tag that zone (and vice versa). In practice by default this does not matter, because the 
built-in tagging routines never explicitly perform a \texttt{CLEAR}. However, 
it is possible to overwrite the {\tt Tagging\_nd.f90} file if you want to change how 
{\tt ca\_denerror, ca\_temperror}, etc. operate. This is not recommended, and if you do so
be aware that {\tt CLEAR}ing a zone this way may not have the desired effect.

We provide also the ability for the user to define their own tagging criteria.
This is done through the Fortran function {\tt set\_problem\_tags} in the 
{\tt problem\_tagging\_*d.f90} files. This function is provided the entire 
state (including density, temperature, velocity, etc.) and the array 
of tagging status for every zone. As an example of how to use this, suppose we 
have a 3D Cartesian simulation where we want to tag any zone that has a 
density gradient greater than 10, but we don't care about any regions 
outside a radius $r > 75$ from the problem origin; we leave them always unrefined. 
We also want to ensure that the region $r \leq 10$ is always refined.
In our {\tt probin} file we would set {\tt denerr = 10} and {\tt max\_denerr\_lev = 1}
in the {\tt \&tagging} namelist. We would also make a copy of 
{\tt problem\_tagging\_3d.f90} to our work directory and set it up as follows:
\clearpage
\begin{verbatim}
subroutine set_problem_tags(tag,tagl1,tagl2,tagl3,tagh1,tagh2,tagh3, &
                            state,state_l1,state_l2,state_l3,state_h1,state_h2,state_h3,&
                            set,clear,&
                            lo,hi,&
                            dx,problo,time,level)

  use bl_constants_module, only: ZERO, HALF
  use prob_params_module, only: center
  use meth_params_module, only: URHO, UMX, UMY, UMZ, UEDEN, NVAR
 
  implicit none
  
  integer         ,intent(in   ) :: lo(3),hi(3)
  integer         ,intent(in   ) :: state_l1,state_l2,state_l3, &
                                    state_h1,state_h2,state_h3
  integer         ,intent(in   ) :: tagl1,tagl2,tagl3,tagh1,tagh2,tagh3
  double precision,intent(in   ) :: state(state_l1:state_h1, &
                                          state_l2:state_h2, &
                                          state_l3:state_h3,NVAR)
  integer         ,intent(inout) :: tag(tagl1:tagh1,tagl2:tagh2,tagl3:tagh3)
  double precision,intent(in   ) :: problo(3),dx(3),time
  integer         ,intent(in   ) :: level,set,clear

  double precision :: x, y, z, r

  do k = lo(3), hi(3)
     z = problo(3) + (dble(k) + HALF) * dx(3) - center(3)
     do j = lo(2), hi(2)
        y = problo(2) + (dble(j) + HALF) * dx(2) - center(2)
        do i = lo(1), hi(1)
           x = problo(1) + (dble(i) + HALF) * dx(1) - center(2)

           r = (x**2 + y**2 + z**2)**(HALF)

           if (r > 75.0) then
             tag(i,j,k) = clear
           elseif (r <= 10.0) then
             tag(i,j,k) = set
           endif
        enddo
     enddo
  enddo
  
end subroutine set_problem_tags
\end{verbatim}

\section{Synchronization Algorithm}

Here is the AMR algorithm for the compressible equations with
self-gravity.  The gravity component of the algorithm 
is closely related to (but not identical to) that in Miniati and Colella,
JCP, 2007 (in press).

Over a coarse grid time step 
we collect flux register information for the hyperbolic part of the synchronization:
\begin{equation}
\delta\Fb = -\Delta t_c A^c F^c + \sum \Delta t_f A^f F^f
\end{equation}
Analogously, at the end of a coarse grid time step 
we store the mismatch in normal gradients of $\phi$ at the coarse-fine interface:
\begin{equation}
\delta F_\phi =  - A^c \frac{\partial \phi^c}{\partial n}
+ \sum A^f \frac{\partial \phi^f}{\partial n}
\end{equation}
Because we want the composite $\phi^{c-f}$ to satisfy the multilevel version of  (\ref{eq:Self Gravity}) at each time $t^n$, we do not accumulate $\frac{\partial \phi^f}{\partial n}$ over time, rather we add coarse and fine fluxes only at integer coarse times.

At the end of a coarse grid time step we can define
${\overline{\Ub}}^{c-f}$ and $\overline{\phi}^{c-f}$
as the composite of the data from coarse and fine grids as a provisional solution at
time $n+1$. (Assume $\overline{\Ub}$ has been averaged down so that the data on coarse cells
underlying fine cells is the average of the fine cell data above it.)

The synchronization consists of two parts: 

\begin{itemize}
\item Step 1:  Hyperbolic reflux

In the hyperbolic reflux step, we update the conserved variables with the flux synchronization
and adjust the gravitational terms to reflect the changes in $\rho$ and $\ub$.
\begin{equation}
{\Ub}^{c, \star} = \overline{\Ub}^{c} + \frac{\delta\Fb}{V},
\end{equation}
where $V$ is the volume of the cell and the correction from $\delta\Fb$ is supported only on coarse cells adjacent to fine grids.   

\item Step 2:  Gravitational synchronization

In this step we correct for the mismatch in normal derivative in $\phi^{c-f}$ at the coarse-fine
interface, as well as accounting for the changes in source terms for $(\rho \ub)$
and $(\rho E)$ due to the change in $\rho.$

On the coarse grid only, we define
\begin{equation}
(\delta \rho)^{c} =  \rho^{c, \star} - {\overline{\rho}}^{c}  .
\end{equation}

We then form the composite residual, which is composed of two contributions.
The first is the degree to which the current $ \overline{\phi}^{c-f}$
does not satisfy the original equation on a composite grid (since we have solved for 
$\overline{\phi}^{c-f}$ separately on the coarse and fine levels).  The second
is the response of $\phi$ to the change in $\rho.$
We define 
\begin{equation} R \equiv  - 4 \pi G \rho^{\star,c-f} - \Delta^{c-f} \; \overline{\phi}^{c-f} 
= - 4 \pi G (\delta \rho)^c - (\nabla \cdot \delta F_\phi ) |_c   .
\end{equation}

Then we solve
\begin{equation}
 \Delta^{c-f} \; \delta \phi^{c-f} = R
\label{eq:gravsync}
\end{equation}
as a two level solve at the coarse and fine levels.  
We define the update to gravity,
\begin{equation}
\delta {\bf g}^{c-f} = \nabla (\delta \phi^{c-f})  .
\end{equation}

Define the \sync sources for momentum on the coarse and fine, $S^{\sync,c}_{\rho \ub}$, and
$S^{\sync,f}_{\rho \ub}$, respectively as follows:
\begin{eqnarray}
S^{\sync,c}_{\rho \ub} &=& \left(
        \overline{\rho}^c + (\delta \rho)^c \right) ({\bf g}^{c,n+1} + \delta {\bf g}^{c-f}) -
        \overline{\rho}^c                 \; {\bf g}^{c,n+1} \nonumber \\
           &=& \left[( \delta \rho)^c {\bf g}^{c,n+1} + 
                \rho^{c,\star} \delta {\bf g}^c)  \right ] \\
S^{\sync,f}_{\rho \ub} &=& \overline{\rho}^{f} \; \delta {\bf g}^f  .
\end{eqnarray}
These momentum sources lead to the following energy sources:
\begin{eqnarray}
S^{\sync,c}_{\rho E} &=& S^{\sync,c}_{\rho \ub} \cdot
         \left(   \overline{\ub}^{c,n+1} + S^{\sync,c}_{\rho \ub} \Delta t_c / (2 \overline{\rho}^{c,n+1}) \right) \\
S^{\sync,f}_{\rho E} &=& S^{\sync,f}_{\rho \ub} \cdot
         \left(
    \overline{\ub}^{f} + S^{\sync,f}_{\rho \ub} \Delta t_f / (2 \; \overline{\rho}^{f} )
         \right)
\end{eqnarray}

The coarse and fine level state is updated using:
\begin{eqnarray}
(\rho \ub)^{c,n+1} =  (\rho \ub)^{c, \star } + \myhalf \Delta t_c S^{\sync,c}_{\rho \ub} ,&&
(\rho \ub)^{f,n+1} =  (\overline{\rho \ub})^{f} + \myhalf \Delta t_f S^{\sync,f}_{\rho \ub},  \\
(\rho E)^{c,n+1} =  (\rho E)^{c, \star } + \myhalf \Delta t_c S^{\sync,c}_{\rho E} ,&&
(\rho E)^{f,n+1} =  (\overline{\rho E})^{f} + \myhalf \Delta t_f S^{\sync,f}_{\rho E}  .
\end{eqnarray}
As the final component of this step we need to 
\begin{itemize}
\item add $\delta \phi^{c-f}$ directly to 
to $\phi^{c}$ and $\phi^{f}$ and interpolate $\delta \phi^{c-f}$ to any finer
levels and add to the current $\phi$ at those levels.

\item if level $c$ is not the coarsest level in the calculation, then we must transmit the
effect of this change in $\phi$ to the coarser levels by updating the flux register between
level $c$ and the next coarser level, $cc.$ In particular, we set
\begin{equation}
\delta {F_\phi}^{cc-c} = \delta F_\phi^{cc-c} 
+ \sum A^c \frac{\partial (\delta \phi)^{c-f}}{\partial n}  .
\end{equation}
\end{itemize}

\end{itemize}
