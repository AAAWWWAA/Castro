
\section{Introduction}

There are several different time-evolution methods currently
implemented in \castro.  As best as possible, they share the same
driver routines and use preprocessor or runtime variables to separate
the different code paths.

\begin{itemize}
\item Strang-splitting: the Strang evolution does the burning on the
  state for $\Delta t/2$, then updates the hydrodynamics using the
  burned state, and then does the final $\Delta t/2$ burning.  No
  explicit coupling of the burning and hydro is done.  Within the
  Strang code path, there are two methods for doing the hydrodynamics,
  controlled by \runparam{castro.do\_ctu}.
  \begin{itemize}
    \item Corner-transport upwind (CTU): this implements the unsplit, 
      characteristic tracing method of \cite{colella:1990}.

    \item Method of lines (MOL): this discretizes the space part of 
      our system without any characteristic tracing and uses an
      ODE integrator to advance the state.  Multiple stages can be done,
      each requiring reconstruction, Riemann solve, etc., and the final
      solution is pieced together from the intermediate stages.
  \end{itemize}

\item SDC: the SDC path is enabled by the \ifdef{SDC} preprocessor
  variable.  This iteratively couples the reactions and hydrodynamics together.

\end{itemize}

Several helper functions are used throughout:
\begin{itemize}
  \item \code{clean\_state}:
\end{itemize}

\section{Overview of a single step (no SDC)}
\label{flow:sec:nosdc}

The main evolution for a single step is contained in
\code{Castro\_advance.cpp}, as \code{Castro::advance()}.  This does
the following advancement.  Note, some parts of this are only done
depending on which preprocessor directives are defined at
compile-time---the relevant directive is noted in the [\ ] at the start
of each step.

\begin{enumerate}
\item {\em Initialization} (\code{initialize\_advance()})

  This sets up the current level for advancement.  The following
  actions are performend (note, we omit the actions taken for a retry,
  which we will describe later):
  \begin{itemize}
  \item Sync up the level information to the Fortran-side of \castro

  \item Do any radiation initialization

  \item Initialize all of the intermediate storage arrays (like those
    that hold source terms, etc.).

  \item Swap the \statedata\ from the new to old (e.g., ensures that
    the next evolution starts with the result from the previous step).

  \item Do a \code{clean\_state}

  \item Create the \multifab s that hold the primitive variable information
   for the hydro solve.

  \item For method of lines integration: allocate the storage for the 
    intermediate stage updates, \code{k\_mol}, and the \code{Sburn} \multifab\
    that holds the post burn state.

  \item Zero out all of the fluxes

  \end{itemize}

      

\item {\em Advancement} 

  The update strategy differs for CTU vs MOL:
  \begin{itemize}
  \item CTU: Calls \code{do\_advance} to take a single step,
    incorporating hydrodynamics, reactions, and source terms.

  \item MOL: Call \code{do\_advance\_mol} \variable{MOL\_STAGES} times
    (i.e., once for each of the intermediate stages in the ODE
    integration).  Within {\tt do\_advance} we will use the stage
    number, \variable{mol\_iteration}, to do an pre- or post-hydro
    sources (e.g., burning).

  \end{itemize}

  In either case, for radiation-hydrodynamics, this step does the
  advective (hyperbolic) portion of the radiation update only.
  Source terms, including gravity, rotation, and diffusion are
  included in this step, and are time-centered to achieve second-order
  accuracy.

  If \runparam{castro.use\_retry} is set, then we subcycle the current
  step if we violated any stability criteria to reach the desired
  $\Delta t$.  The idea is the following: if the timestep that you
  took had a timestep that was not sufficient to enforce the stability
  criteria that you would like to achieve, such as the CFL criterion
  for hydrodynamics or the burning stability criterion for reactions,
  you can retry the timestep by setting {\tt castro.use\_retry = 1} in
  your inputs file. This will save the current state data at the
  beginning of the level advance, and then if the criteria are not
  satisfied, will reject that advance and start over from the old
  data, with a series of subcycled timesteps that should be small
  enough to satisfy the criteria.  Note that this will effectively
  double the memory footprint on each level if you choose to use it.


\item {[\ifdef{AUX\_UPDATE}]} {\em Auxiliary quantitiy evolution} 

  Auxiliary variables in Castro are those that obey a continuity
  equation (with optional sources) that are passed into the EOS, but
  not subjected to the constraint on mass fractions (summing to one).

  The advection and source terms are already dealt with in the 
  main hydrodynamics advance (above step).  A user-supplied routine
  \code{ca\_auxupdate} can be provided here to further update these
  quantities.
  
\item {\em Radial data and {\rm[\ifdef{POINTMASS}]} point mass} 

  If \runparam{castro.spherical\_star} is set, then we average the state data
  over angles here to create a radial profile.  This is then used in the 
  boundary filling routines to properly set Dirichlet BCs when our domain
  is smaller than the star, so the profile on the boundaries will not 
  be uniform.

  If \runparam{castro.point\_mass\_fix\_solution} is set, then we
  change the mass of the point mass that optionally contributes to the
  gravitational potential by taking mass from the surrounding zones
  (keeping the density in those zones constant).

\item {[\ifdef{RADIATION}]} {\em Radiation implicit update} 

  The {\tt do\_advance()} routine only handled the hyperbolic
  portion of the radiation update.  This step does the implicit solve
  (either gray or multigroup) to advance the radiation energies to the 
  new time level.  Note that at the moment, this is backward-difference
  implicit (first-order in time) for stability.

  This is handled by \code{final\_radiation\_call()}.

\item {[\ifdef{PARTICLES}]} {\em Particles} 

  If we are including passively-advected particles, they are
  advanced in this step.

\item {\em Finalize}

  This cleans up at the end of a step:
  \begin{itemize}
    \item Update the flux registers to account for mismatches at
      coarse-fine interfaces.  This cleans up the memory used during
      the step.

    \item If \runparam{castro.track\_grid\_losses} is set, then we
      also add up the mass that left through the boundary over this
      step.\footnote{Note: this functionality assumes that only the
        coarse grid touches the physical boundary.  It does not use
        any use masks to prevent double counting if multiple levels
        touch the boundary.}

    \item Free any memory allocated for the level advance.
  \end{itemize}

\end{enumerate}

\subsection{Main Hydro, Reaction, and Gravity Advancement (CTU w/ Strang-splitting)}

The explicit portion of the system advancement (reactions,
hydrodynamics, and gravity) is done by \code{do\_advance()}.  Consider
our system of equations as:
\begin{equation}
\frac{\partial\Ub}{\partial t} = -{\bf A}(\Ub) + \Rb(\Ub) + \Sb,
\end{equation}
where ${\bf A}(\Ub) = \nabla \cdot \Fb(\Ub)$, with $\Fb$ the flux vector, $\Rb$ are the reaction
source terms, and $\Sb$ are the non-reaction source terms, which
includes any user-defined external sources, $\Sb_{\rm ext}$.  We use
Strang splitting to discretize the advection-reaction equations.  In
summary, for each time step, we update the conservative variables,
$\Ub$, by reacting for half a time step, advecting for a full time
step (ignoring the reaction terms), and reacting for half a time step.
The treatment of source terms complicates this a little.  The actual
update, in sequence, looks like:
\begin{align}
\Ub^\star &= \Ub^n + \frac{\dt}{2}\Rb(\Ub^n) \\
\Ub^{n+1,(a)} &= \Ub^\star + \dt\, \Sb(\Ub^\star) \\
\Ub^{n+1,(b)} &= \Ub^{n+1,(a)} - \dt\, {\bf A}(\Ub^\star) \\
\Ub^{n+1,(c)} &= \Ub^{n+1,(b)} + \frac{\dt}{2}\, [\Sb(\Ub^{n+1,(b)}) - \Sb(\Ub^\star)] \label{eq:source_correct}\\
\Ub^{n+1}     &= \Ub^{n+1,(c)} + \frac{\dt}{2} \Rb(\Ub^{n+1,(c)})
\end{align}
Note that in the first step, we add a full $\Delta t$ of the old-time
source to the state.  This prediction ensures consistency when it
comes time to predicting the new-time source at the end of the update.
The construction of the advective terms, ${\bf A(\Ub)}$ is purely
explicit, and based on an unsplit second-order Godunov method.  We
predict the standard primitive variables, as well as $\rho e$, at
time-centered edges and use an approximate Riemann solver construct
fluxes.

At the beginning of the time step, we assume that $\Ub$ and $\phi$ are
defined consistently, i.e., $\rho^n$ and $\phi^n$ satisfy equation
(\ref{eq:Self Gravity}).  Note that in
Eq.~\ref{eq:source_correct}, we actually can actually do some 
sources implicitly by updating density first, and then momentum, 
and then energy.  This is done for rotating and gravity, and can
make the update more akin to:
\begin{equation}
\Ub^{n+1,(c)} = \Ub^{n+1,(b)} + \frac{\dt}{2} [\Sb(\Ub^{n+1,(c)}) - \Sb(\Ub^n)]
\end{equation}

\castro\ also supports radiation.  This part of the update algorithm
only deals with the advective / hyperbolic terms in the radiation update.

Here is the single-level algorithm.  The goal here is to update the
\variable{State\_Type} \statedata\ from the old to new time (see
\S~\ref{soft:sec:statedata}).  We will use the following notation
here, consistent with the names used in the code:
\begin{itemize}
\item \variable{S\_old} is a multifab reference to the old-time-level
  {\tt State\_Type} data.

\item \variable{Sborder} is a multifab that has ghost cells and is
  initialized from {\tt S\_old}.  This is what the hydrodynamic
  reconstruction will work from.

\item \variable{S\_new} is a multifab reference to the new-time-level
  {\tt State\_Type} data.
\end{itemize}

In the code, the objective is to evolve the state from the old time,
{\tt S\_old}, to the new time, {\tt S\_new}.

\begin{enumerate}
\item \label{strang:init} {\em Initialize}

  \begin{enumerate}
  \item In \code{initialize\_do\_advance()} :
    \begin{enumerate}
    \item Create \variable{Sborder}, initialized from {\tt S\_old}
    \end{enumerate}

  \item Check for NaNs in the initial state, {\tt S\_old}.
  \end{enumerate}

\item {\em React $\Delta t/2$.} [\code{strang\_react\_first\_half()}]

  Update the solution due to the effect of reactions over half a time
  step.  The integration method and system of equations used here is
  determined by a host of runtime parameters that are part of the
  \microphysics\ package.  But the basic idea is to evolve the energy
  release from the reactions, the species mass fractions, and
  temperature through $\Delta t/2$.

  Using the notation above, we begin with the time-level $n$ state,
  $\Ub^n$, and produce a state that has evolved only due to reactions,
  $\Ub^\star$.

  \begin{align}
    (\rho e)^\star &= (\rho e)^\star - \frac{\dt}{2} \rho H_\mathrm{nuc} \\
    (\rho E)^\star &= (\rho E)^\star - \frac{\dt}{2} \rho H_\mathrm{nuc} \\
    (\rho X_k)^\star &= (\rho X_k)^\star + \frac{\dt}{2}(\rho\omegadot_k)^n.
  \end{align}
  Here, $H_\mathrm{nuc}$ is the energy release (erg/g/s) over the
  burn, and $\omegadot_k$ is the creation rate for species $k$.

  After exiting the burner, we call the EOS with $\rho^\star$,
  $e^\star$, and $X_k^\star$ to get the new temperature, $T^\star$.

  Note that the density, $\rho$, does not change via reactions in the
  Strang-split formulation.

  The reaction data needs to be valid in the ghost cells.  The logic
  in this routine (accomplished throuh the use of a mask) will burn
  only in the valid interior cells or in any ghost cells that are on a
  coarse-fine interface or physical boundary.  This allows us to just
  use a level {\tt FillBoundary()} call to fill all of the ghost cells
  on the same level with valid data.

  An experimental option (enabled via
  \runparam{use\_custom\_knapsack\_weights}) will create a custom
  distribution map based on the work needed in burning a zone and
  redistribute the boxes across processors before burning, to better
  load balance..  \MarginPar{need to explain the logic here,
    especially what the parallel copies are doing}

  After reactions, \code{clean\_state} is called.

  At the end of this step, \variable{Sborder} sees the effects of the
  reactions.


\item \label{strang:oldsource} {\em Construct time-level $n$ sources and apply} 
  [\code{construct\_old\_gravity()}, \code{do\_old\_sources()}]

  The time level $n$ sources are computed, and added to the
  \statedata\ \variable{Source\_Type}.  The sources are then applied
  to the state after the burn, $\Ub^\star$ with a full $\Delta t$
  weighting (this will be corrected later).  This produces the
  intermediate state, $\Ub^{n+1,(a)}$.

  The sources that we deal with here are:
  \begin{enumerate}
  \item sponge : the sponge\index{sponge} is a damping term added to
    the momentum equation that is designed to drive the velocities to
    zero over some timescale.  Our implementation of the sponge
    follows that of \maestro~\cite{maestro:III}

  \item external sources : users can define problem-specific sources
    in the \code{ext\_src\_?d.f90} file.  Sources for the different
    equations in the conservative state vector, $\Ub$, are indexed
    using the integer keys defined in {\tt meth\_params\_module}
    (e.g., {\tt URHO}).

    This is most commonly used for external heat sources (see the
    \problem{toy\_convect} problem setup) for an example.  But most
    problems will not use this.

  \item {[\ifdef{DIFFUSION}]} diffusion : thermal diffusion can be
    added in an explicit formulation.  Second-order accuracy is
    achieved by averaging the time-level $n$ and $n+1$ terms, using
    the same predictor-corrector strategy described here.

    Note: thermal diffusion is distinct from radiation hydrodynamics.

    Also note that incorporating diffusion brings in an additional
    timestep constraint, since the treatment is explicit.  See
    Chapter~\ref{ch:diffusion} for more details. 

  \item {[\ifdef{HYBRID\_MOMENTUM}]} angular momentum 

    \MarginPar{need to write this up}

  \item {[\ifdef{GRAVITY}]} gravity:

    For full Poisson gravity, we solve for for gravity using:
    \begin{equation}
      \gb^n = -\nabla\phi^n, \qquad
      \Delta\phi^n = 4\pi G\rho^n,
    \end{equation}

    The construction of the form of the gravity source for the
    momentum and energy equation is dependent on the parameter
    \runparam{castro.grav\_source\_type}.  Full details of the gravity
    solver are given in Chapter~\ref{ch:gravity}.

    \MarginPar{we should add a description of whether we do a level solve or a composite solve}

    \MarginPar{what do we store? phi and g? source?}

  \item {[\ifdef{ROTATION}]} rotation

    We compute the rotational potential (for use in the energy update)
    and the rotational acceleration (for use in the momentum
    equation).  This includes the Coriolis and centrifugal terms in a
    constant-angular-velocity co-rotating frame.  The form of the
    rotational source that is constructed then depends on the
    parameter \runparam{castro.rot\_source\_type}.  More details are
    given in Chapter~\ref{ch:rotation}.
    
  \end{enumerate}

  The source terms here are evaluated using the post-burn state,
  $\Ub^\star$ (\variable{Sborder}), and later corrected by using the
  new state just before the burn, $\Ub^{n+1,(b)}$.  This is compatible
  with Strang-splitting, since the hydro and sources takes place
  completely inside of the surrounding burn operations.

  Note that the source terms are already applied to \variable{S\_new}
  in this step, with a full $\Delta t$---this will be corrected later.

  For MOL integration, we need to preserve the state with the result
  of the burn and old-time sources, so we save this in a
  \multifab\ \variable{S\_burn}.  After all the stages of the MOL
  integration are computed, we use this as the starting point for 
  constructing the final update. \MarginPar{need to revisit sources with MOL}.

\item \label{strang:hydro} {\em Construct the hydro update} [\code{construct\_hydro\_source()}]

  The goal is to advance our system considering only the advective
  terms (which in Cartesian coordinates can be written as the
  divergence of a flux).

  We do the hydro update in two parts---first we construct the
  advective update and store it in the \variable{hydro\_source}
  \multifab, then we do the conservative update in a separate step.  This
  separation allows us to use the advective update separately in more
  complex time-integration schemes.

  In the Strang-split formulation, we start the reconstruction using
  the state after burning, $\Ub^\star$ (\variable{Sborder}).  There
  are two approaches we use, the corner transport upwind (CTU) method
  that uses characteristic tracing as described in
  \cite{colella:1990}, and a method-of-lines approach.  The choice is 
  determined by the parameter \runparam{castro.do\_ctu}.

  \begin{enumerate}
  \item CTU method:

    For the CTU method, we predict to the half-time ($n+1/2$) to get a
    second-order accurate method.  Note: \variable{Sborder} does not
    know of any sources except for reactions.  The advection step is
    complicated, and more detail is given in Section
    \ref{Sec:Advection Step}.  Here is the summarized version:
  \begin{enumerate}
  \item Compute primitive variables.
  \item Convert the source terms to those acting on primitive variables
  \item Predict primitive variables to time-centered edges.
  \item Solve the Riemann problem.
  \item Compute fluxes and update.
  \end{enumerate}

  To start the hydrodynamics, we need to know the hydrodynamics source
  terms at time-level $n$, since this enters into the prediction to
  the interface states.  This is essentially the same vector that was
  computed in the previous step, with a few modifications.  The most
  important is that if we set
  \runparam{castro.source\_term\_predictor}, then we extrapolate the
  source terms from $n$ to $n+1/2$, using the change from the previous
  step.

  Note: we neglect the reaction source terms, since those are already
  accounted for in the state directly, due to the Strang-splitting
  nature of this method.

  The update computed here is then immediately applied to
  \variable{S\_new}.


  \item method of lines
    \MarginPar{this needs to be filled}

  \end{enumerate}

\item \label{strang:clean} {\em Clean State} [\code{clean\_state()}]

  \MarginPar{we only seem to do this for the MOL integration}

  There are many ways that the hydrodynamics state may become
  unphysical in the evolution.  The \code{clean\_state()} routine
  enforces some checks on the state.  In particular, it
  \begin{enumerate}
  \item enforces that the density is above \runparam{castro.small\_dens}
  \item normalizes the species so that the mass fractions sum to 1
  \item resets the internal energy if necessary (too small or negative)
    and computes the temperature for all zones to be thermodynamically 
    consistent with the state.
  \end{enumerate}
  This is done on \variable{S\_new}.

  After these checks, we check the state for NaNs.

\item \label{strang:radial} {\em Update radial data and center of mass for monopole gravity}
 \MarginPar{is that right?}

 These quantities are computed using \variable{S\_new}.



\item \label{strang:newsource} {\em Correct the source terms with the $n+1$ contribution}
  [\code{construct\_new\_gravity()}, \code{do\_new\_sources}]

  Previously we added $\Delta t\, \Sb(\Ub^\star)$ to the state, when
  we really want a time-centered approach, $(\Delta t/2)[\Sb(\Ub^\star
    + \Sb(\Ub^{n+1,(b)})]$.  We fix that here.

  We start by computing the source term vector $\Sb(\Ub^{n+1,(b)})$
  using the updated state, $\Ub^{n+1,(b)}$.  We then compute the
  correction, $(\Delta t/2)[\Sb(\Ub^{n+1,(b)}) - \Sb(\Ub^\star)]$ to
  add to $\Ub^{n+1,(b)}$ to give us the properly time-centered source,
  and the fully updated state, $\Ub^{n+1,(c)}$.  This correction is stored
  in the \variable{new\_sources} \multifab\footnote{The correction for gravity is slightly different since we directly compute the time-centered gravitational source term using the hydrodynamic fluxes.}.


  In the process of updating the sources, we update the temperature to
  make it consistent with the new state.

%% We need to correct the solution by effectively time-centering the
%% source terms.  These corrections are to be performed sequentially
%% since new source term evaluations may depend on previous corrections.

%% First, we correct the solution with the updated gravity:
%% \begin{eqnarray}
%% (\rho\ub)^{n+1} &=& (\rho\ub)^{n+1} + \frac{\dt}{2}\left[(\rho\gb)^{n+1} - (\rho\gb)^n\right], \\
%% (\rho E)^{n+1} &=& (\rho E)^{n+1} + \frac{\dt}{2}\left[\left(\rho\ub\cdot\gb\right)^{n+1} - \left(\rho\ub\cdot\gb\right)^n\right].
%% \end{eqnarray}

%% Next, we correct $\Ub$ with updated external sources.  For example,
%% for the momentum, we correct using
%% \begin{equation}
%% (\rho\ub)^{n+1} = (\rho\ub)^{n+1} + \frac{\dt}{2}\left(\Sb_{{\rm ext},\rho\ub}^{n+1} - \Sb_{{\rm ext},\rho\ub}^n\right).
%% \end{equation}
%% We correct $\rho E, \rho A_k, \rho X_k$, and $\rho Y_k$ in an
%% analogous manner.

%% Finally, we correct the solution with updated thermal diffusion using
%% \begin{equation}
%% (\rho E)^{n+1} = (\rho E)^{n+1} + \frac{\dt}{2}\left(\nabla\cdot\kappa\nabla T^{n+1} - \nabla\cdot\kappa\nabla T^n\right).
%% \end{equation}


\item {\em React $\Delta t/2$.} [\code{strang\_react\_second\_half()}]

  We do the final $\dt/2$ reacting on the state, begining with $\Ub^{n+1,(c)}$ to
  give us the final state on this level, $\Ub^{n+1}$.

  This is largely the same as \code{strang\_react\_first\_half()}, but
  it does not currently fill the reactions in the ghost cells. \MarginPar{confirm this}

\item \label{strang:finalize} {\em Finalize} [\code{finalize\_do\_advance()}]

  Finalize does the following:
  \begin{enumerate}
  \item for the momentum sources, we compute $d\Sb/dt$, to use in the
    source term prediction/extrapolation for the hydrodynamic
    interface states during the next step.

  \item If we are doing the hybrid momentum algorithm, then we sync up
    the hybrid and linear momenta
  \end{enumerate}

\end{enumerate}

A summary of which state is the input and which is updated for each of
these processes is presented below:
\begin{center}
\renewcommand{\arraystretch}{1.5}
\begin{tabular}{l|lll}
{\em step} & {\tt S\_old} & {\tt Sborder} & {\tt S\_new} \\
\hline
1. init    &   input      &   updated \\
2. react   &              &    input / updated \\
3. old sources &          &   input    & updated \\
4. hydro   &              & input      & updated \\
5. clean   &              &            & input / updated \\
6. radial / center  &     &            & input \\
7. correct sources &      &            & input / updated \\
8. react   &              &            & input / updated
\end{tabular}
\end{center}




\subsection{Main Hydro, Reaction, and Gravity Advancement (MOL w/ Strang-splitting)}

The handling of sources differs in the MOL integration, as compared to CTU.
Again, consider our system as:
\begin{equation}
\frac{\partial\Ub}{\partial t} = -{\bf A}(\Ub) + \Rb(\Ub) + \Sb \, .
\end{equation}
We will again use Strang splitting to discretize the
advection-reaction equations, but the hydro update will consist of $s$
stages.  The update first does the reactions, as with CTU:
\begin{equation}
\Ub^\star = \Ub^n + \frac{\dt}{2}\Rb(\Ub^n)
\end{equation}
We then consider the hydro update discretized in space, but not time, written
as:
\begin{equation}
\frac{\partial \Ub}{\partial t} = -{\bf A}(\Ub) + \Sb(\Ub)
\end{equation}
Using a Runge-Kutta (or similar) integrator, we write the update as:
\begin{equation}
\Ub^{n+1,\star} = \Ub^\star + \dt \sum_{l=1}^s b_i {\bf k}_l
\end{equation}
where $b_i$ is the weight for stage $i$ and $k_i$ is the stage update:
\begin{equation}
{\bf k}_l = -{\bf A}(\Ub_l) + \Sb(\Ub_l)
\end{equation}
with 
\begin{equation}
\Ub_l = \Ub^\star  + \dt \sum_{m=1}^{l-1} a_{lm} {\bf k}_m
\end{equation}
Finally, there is the last part of the reactions:
\begin{equation}
\Ub^{n+1} = \Ub^{n+1,\star} + \frac{\dt}{2} \Rb(\Ub^{n+1,\star})
\end{equation}
In contrast to the CTU method, the sources are treated together
with the advection here.

The integration coefficients are stored in the vectors
\variable{a\_mol} and \variable{b\_mol}, and the stage updates are
stored in the \multifab\ \variable{k\_mol}.

Here is the single-level algorithm.  We use the same notation
as in the CTU flowchart.

In the code, the objective is to evolve the state from the old time,
{\tt S\_old}, to the new time, {\tt S\_new}.

\begin{enumerate}
\item \label{strang:init} {\em Initialize}

  \begin{enumerate}
  \item In \code{initialize\_do\_advance()} :
    \begin{enumerate}
    \item Create \variable{Sborder}, initialized from {\tt S\_old}
    \end{enumerate}

  \item Check for NaNs in the initial state, {\tt S\_old}.
  \end{enumerate}

\item {\em React $\Delta t/2$.} [\code{strang\_react\_first\_half()}]

  This step is unchanged from the CTU version.  At the end of this
  step, \variable{Sborder} sees the effects of the reactions.


\item \label{strang:oldsource} {\em Construct time-level $n$ sources and apply} 
  [\code{construct\_old\_gravity()}, \code{do\_old\_sources()}]

  The time level $n$ sources are computed, and added to the
  \statedata\ \variable{Source\_Type}.  The sources are then applied
  to the state after the burn, $\Ub^\star$ with a full $\Delta t$
  weighting (this will be corrected later).  This produces the
  intermediate state, $\Ub^{n+1,(a)}$.

  For full Poisson gravity, we solve for for gravity using:
  \begin{equation}
    \gb^n = -\nabla\phi^n, \qquad
    \Delta\phi^n = 4\pi G\rho^n,
  \end{equation}

  How are they evaluated?

  For MOL integration, we need to preserve the state with the result
  of the burn and old-time sources, so we save this in a
  \multifab\ \variable{S\_burn}.  After all the stages of the MOL
  integration are computed, we use this as the starting point for 
  constructing the final update. \MarginPar{need to revisit sources with MOL}.

\item \label{strang:hydro} {\em Construct the hydro update} [\code{construct\_hydro\_source()}]

  The goal is to advance our system considering only the advective
  terms (which in Cartesian coordinates can be written as the
  divergence of a flux).

  We do the hydro update in two parts---first we construct the
  advective update and store it in the \variable{hydro\_source}
  \multifab, then we do the conservative update in a separate step.  This
  separation allows us to use the advective update separately in more
  complex time-integration schemes.

  In the Strang-split formulation, we start the reconstruction using
  the state after burning, $\Ub^\star$ (\variable{Sborder}).  There
  are two approaches we use, the corner transport upwind (CTU) method
  that uses characteristic tracing as described in
  \cite{colella:1990}, and a method-of-lines approach.  The choice is 
  determined by the parameter \runparam{castro.do\_ctu}.

  \begin{enumerate}
  \item CTU method:

    For the CTU method, we predict to the half-time ($n+1/2$) to get a
    second-order accurate method.  Note: \variable{Sborder} does not
    know of any sources except for reactions.  The advection step is
    complicated, and more detail is given in Section
    \ref{Sec:Advection Step}.  Here is the summarized version:
  \begin{enumerate}
  \item Compute primitive variables.
  \item Convert the source terms to those acting on primitive variables
  \item Predict primitive variables to time-centered edges.
  \item Solve the Riemann problem.
  \item Compute fluxes and update.
  \end{enumerate}

  To start the hydrodynamics, we need to know the hydrodynamics source
  terms at time-level $n$, since this enters into the prediction to
  the interface states.  This is essentially the same vector that was
  computed in the previous step, with a few modifications.  The most
  important is that if we set
  \runparam{castro.source\_term\_predictor}, then we extrapolate the
  source terms from $n$ to $n+1/2$, using the change from the previous
  step.

  Note: we neglect the reaction source terms, since those are already
  accounted for in the state directly, due to the Strang-splitting
  nature of this method.

  The update computed here is then immediately applied to
  \variable{S\_new}.


  \item method of lines
    \MarginPar{this needs to be filled}

  \end{enumerate}

\item \label{strang:clean} {\em Clean State} [\code{clean\_state()}]

  \MarginPar{we only seem to do this for the MOL integration}

  There are many ways that the hydrodynamics state may become
  unphysical in the evolution.  The \code{clean\_state()} routine
  enforces some checks on the state.  In particular, it
  \begin{enumerate}
  \item enforces that the density is above \runparam{castro.small\_dens}
  \item normalizes the species so that the mass fractions sum to 1
  \item resets the internal energy if necessary (too small or negative)
    and computes the temperature for all zones to be thermodynamically 
    consistent with the state.
  \end{enumerate}
  This is done on \variable{S\_new}.

  After these checks, we check the state for NaNs.

\item \label{strang:radial} {\em Update radial data and center of mass for monopole gravity}
 \MarginPar{is that right?}

 These quantities are computed using \variable{S\_new}.



\item \label{strang:newsource} {\em Correct the source terms with the $n+1$ contribution}
  [\code{construct\_new\_gravity()}, \code{do\_new\_sources}]

  Previously we added $\Delta t\, \Sb(\Ub^\star)$ to the state, when
  we really want a time-centered approach, $(\Delta t/2)[\Sb(\Ub^\star
    + \Sb(\Ub^{n+1,(b)})]$.  We fix that here.

  We start by computing the source term vector $\Sb(\Ub^{n+1,(b)})$
  using the updated state, $\Ub^{n+1,(b)}$.  We then compute the
  correction, $(\Delta t/2)[\Sb(\Ub^{n+1,(b)}) - \Sb(\Ub^\star)]$ to
  add to $\Ub^{n+1,(b)}$ to give us the properly time-centered source,
  and the fully updated state, $\Ub^{n+1,(c)}$.  This correction is stored
  in the \variable{new\_sources} \multifab\footnote{The correction for gravity is slightly different since we directly compute the time-centered gravitational source term using the hydrodynamic fluxes.}.


  In the process of updating the sources, we update the temperature to
  make it consistent with the new state.

%% We need to correct the solution by effectively time-centering the
%% source terms.  These corrections are to be performed sequentially
%% since new source term evaluations may depend on previous corrections.

%% First, we correct the solution with the updated gravity:
%% \begin{eqnarray}
%% (\rho\ub)^{n+1} &=& (\rho\ub)^{n+1} + \frac{\dt}{2}\left[(\rho\gb)^{n+1} - (\rho\gb)^n\right], \\
%% (\rho E)^{n+1} &=& (\rho E)^{n+1} + \frac{\dt}{2}\left[\left(\rho\ub\cdot\gb\right)^{n+1} - \left(\rho\ub\cdot\gb\right)^n\right].
%% \end{eqnarray}

%% Next, we correct $\Ub$ with updated external sources.  For example,
%% for the momentum, we correct using
%% \begin{equation}
%% (\rho\ub)^{n+1} = (\rho\ub)^{n+1} + \frac{\dt}{2}\left(\Sb_{{\rm ext},\rho\ub}^{n+1} - \Sb_{{\rm ext},\rho\ub}^n\right).
%% \end{equation}
%% We correct $\rho E, \rho A_k, \rho X_k$, and $\rho Y_k$ in an
%% analogous manner.

%% Finally, we correct the solution with updated thermal diffusion using
%% \begin{equation}
%% (\rho E)^{n+1} = (\rho E)^{n+1} + \frac{\dt}{2}\left(\nabla\cdot\kappa\nabla T^{n+1} - \nabla\cdot\kappa\nabla T^n\right).
%% \end{equation}


\item {\em React $\Delta t/2$.} [\code{strang\_react\_second\_half()}]

  We do the final $\dt/2$ reacting on the state, begining with $\Ub^{n+1,(c)}$ to
  give us the final state on this level, $\Ub^{n+1}$.

  This is largely the same as \code{strang\_react\_first\_half()}, but
  it does not currently fill the reactions in the ghost cells. \MarginPar{confirm this}

\item \label{strang:finalize} {\em Finalize} [\code{finalize\_do\_advance()}]

  Finalize does the following:
  \begin{enumerate}
  \item for the momentum sources, we compute $d\Sb/dt$, to use in the
    source term prediction/extrapolation for the hydrodynamic
    interface states during the next step.

  \item If we are doing the hybrid momentum algorithm, then we sync up
    the hybrid and linear momenta
  \end{enumerate}

\end{enumerate}

A summary of which state is the input and which is updated for each of
these processes is presented below:
\begin{center}
\renewcommand{\arraystretch}{1.5}
\begin{tabular}{l|lll}
{\em step} & {\tt S\_old} & {\tt Sborder} & {\tt S\_new} \\
\hline
1. init    &   input      &   updated \\
2. react   &              &    input / updated \\
3. old sources &          &   input    & updated \\
4. hydro   &              & input      & updated \\
5. clean   &              &            & input / updated \\
6. radial / center  &     &            & input \\
7. correct sources &      &            & input / updated \\
8. react   &              &            & input / updated
\end{tabular}
\end{center}



\section{Overview of a single step (with SDC)}

We express our system as:
\begin{equation}
\Ub_t = \mathcal{A}(\Ub) + \Rb(\Ub)
\end{equation}
here $\mathcal{A}$ is the advective source, which includes both the 
flux divergence and the hydrodynamic source terms (e.g.\ gravity):
\begin{equation}
\mathcal{A}(\Ub) = -\nabla \cdot \Fb(\Ub) + \Sb
\end{equation}

The SDC version of the main advance loop looks similar to the no-SDC
version, but includes an iteration loop over the hydro, gravity, and
reaction update.  So the only difference happens in step 2 of the
flowchart outlined in \S~\ref{flow:sec:nosdc}.  In particular this
step now proceeds as:

\begin{enumerate}
\setcounter{enumi}{1}

\item {\em Advancement}

  Loop $k$ from 0 to {\tt sdc\_iters}, doing:

  \begin{enumerate}
    \item {\em Hydrodynamics advance}: This is done through {\tt
      do\_advance}---in SDC mode, this only updates the hydrodynamics,
      including the non-reacting sources.  However, in predicting the
      interface states, we use an iteratively-lagged approximation to the 
      reaction source on the primitive variables, $\mathcal{I}_q^{k-1}$.  \MarginPar{first time through is this source 0?}

      The result of this is an approximation to $\mathcal{A}(\Ub)$,
      stored in \variable{hydro\_sources} (the flux divergence)
      and \variable{old\_sources} and \variable{new\_sources}.

    \item {\em React}: Reactions are integrated with the advective
      update as a source---this way the reactions see the
      time-evolution due to advection as we integrate:
      \begin{equation}
        \frac{d\Ub}{dt} = \left [ \mathcal{A}(\Ub) \right ]^{n+1/2} + \Rb(\Ub)
      \end{equation}
     The advective source includes both the divergence of the fluxes
      as well as the time-centered source terms.  This is computed by
      \code{sum\_of\_sources()} by summing over all source components
      \variable{hydro\_source}, \variable{old\_sources}, and
      \variable{new\_sources}.  

    \item {\em Clean state}: This ensures that the thermodynamic state is
      valid and consistent.

    \item {\em Construct reaction source terms}: Construct the change
      in the primitive variables due only to reactions over the
      timestep, $\mathcal{I}_q^{k}$.  This will be used in the next
      iteration.
  \end{enumerate}


\end{enumerate}


Note that is it likely that some of the other updates (like any
non-advective auxiliary quantity updates) should be inside the SDC
loop, but presently they are only done at the end.  Also note that the
radiation implicit update is not done as part of the SDC iterations.


\subsection{Main Hydro and Gravity Advancement (SDC)}

The evolution in {\tt do\_advance} is substantially different than the
Strang case.  In particular, reactions are not evolved.  Here we
summarize those differences.

\begin{enumerate}
\item {\em Initialize} [\code{initialize\_do\_advance()}]

This is unchanged from step \ref{strang:init} in the Strang algorithm.


\item {\em Construct time-level $n$ sources and apply} 
  [\code{construct\_old\_gravity()}, \code{do\_old\_sources()}]

This corresponds to step \ref{strang:oldsource} in the Strang
algorithm.  There are not differences compared to the Strang
algorithm, although we note, this only needs to be done for the first
SDC iteration in the advancement, since the old state does not change.\MarginPar{we need to fix the code to only do this once}

\item {\em Construct the hydro update} [\code{construct\_hydro\_source()}]

This corresponds to step~\ref{strang:hydro} in the Strang
algorithm.  There are a few major differences with the Strang case:
\begin{itemize}
\item There is no need to extrapolate source terms to the half-time
  for the prediction (the \runparam{castro.source\_term\_predictor}
  parameter), since SDC provides a natural way to approximate the
  time-centered source---we simply use the iteratively-lagged new-time
  source.

\item The primitive variable source terms that are used for the
  prediction include the contribution due to reactions (from the last
  SDC iteration).  This addition is done in
  \code{construct\_hydro\_source()} after the source terms are
  converted to primitive variables.
\end{itemize}

\item {\em Update radial data and center of mass for monopole gravity}
  
  This is the same as the Strang step~\ref{strang:radial}


\item {\em Clean State} [\code{clean\_state()}]

  This is the same as the Strang step~\ref{strang:clean}

\item \label{strang:newsource} {\em Correct the source terms with the $n+1$ contribution}
  [\code{construct\_new\_gravity()}, \code{do\_new\_sources}]

  This is the same as the Strang step~\ref{strang:newsource}

\item {\em Finalize} [\code{finalize\_do\_advance()}]

  This differs from Strang step~\ref{strang:finalize} in that we do not
  construct $d\Sb/dt$, but instead store the total hydrodynamical source
  term at the new time.  As discussed above, this will be used in the 
  next iteration to approximate the time-centered source term.

\end{enumerate}




