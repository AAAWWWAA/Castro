\section{Primary Data-structures}

\castro\ is built on the \cpp\ \boxlib\ framework, and relies on the
{\tt Amr} class defined there to drive the simulation.

The following other classes / concepts also play a major role laying
out \castro's algorithm:
\begin{itemize}
\item \code{AmrLevel} : this is a class that has the whole fluid state \MarginPar{more}

\item \code{Castro} : this is an {\tt AmrLevel}-derived class for a
  single level---it contains all the data needed to advance the state
  for a single level

\item \code{MultiFab} : this is the collection of boxes (w/ data) at the 
  same level of refinement.  A \tt {\tt MultiFab} can have multiple
  components, ghost cells, etc.  \MarginPar{what else? does it know about boundaries?}

\item \code{\statedata}

\end{itemize}


\section{Overview of a single step (no SDC)}

The main evolution for a single step is contained in
\code{Castro\_advance.cpp}, as \code{Castro::advance()}.  This does
the following advancement:
\begin{enumerate}
\item {\em Initialization} 

  This sets up the current level for advancement.  If we are at the
  start of a coarse level timestep, or we are in the middle of
  subcycling on a finer level (\code{amr\_iteration} {\tt > 1}), then
  we swap the \statedata\ from the new to old (e.g., this ensures that
  the next evolution starts with the result from the previous step).

  This also syncs up the level information to the Fortran-side of
  \castro, does any radiation initialization, and initializes all of
  the intermediate storage arrays (like those that hold source terms,
  etc.).

\item {\em Advancement} \\

\item {\em Retries} \\

\item {\em Auxillary quantitiy evolution} \\

\item {\em Levelsets} \\

\item {\em Radial data and point mass} \\

\item {\em Radiation implicit update} \\

  The {\tt do\_advance()} routine only handled the hyperbolic
  portion of the radiation update.  This step does the implicit solve
  (either gray or multigroup) to advance the radiation energies to the 
  new time level.  Note that at the moment, this is backward-difference
  implicit (first-order in time) for stability.

\item {\em Particles} \\

\item {\em Finalize}

\end{enumerate}


Consider our system of  Our equations look like:
\begin{equation}
\frac{\partial\Ub}{\partial t} = \nabla\cdot\Fb + \Sb_{\rm react} + \Sb,
\end{equation}
where $\Fb$ is the flux vector, $\Sb_{\rm react}$ are the reaction
source terms, and $\Sb$ are the non-reaction source terms, which
includes any user-defined external sources, $\Sb_{\rm ext}$.  We use
Strang splitting to discretize the advection-reaction equations.  In
summary, for each time step, we update the conservative variables,
$\Ub$, by reacting for half a time step, advecting for a full time
step (ignoring the reaction terms), and reacting for half a time step.
In summary,
\begin{equation}
\Ub^n = \Ub^n + \frac{\dt}{2}\Sb_{\rm react}^n,
\end{equation}
\begin{equation}
\Ub^{n+1} = \Ub^n - \Delta t \nabla \cdot\Fb^\nph + \dt\frac{\Sb^n + \Sb^{n+1}}{2},
\end{equation}
\begin{equation}
\Ub^{n+1} = \Ub^{n+1} + \frac{\dt}{2}\Sb_{\rm react}^{n+1},
\end{equation}
The construction of $F$ is purely explicit, and based on an unsplit
second-order Godunov method.  We predict the standard primitive
variables, as well as $\rho e$, at time-centered edges and use an
approximate Riemann solver construct fluxes.  At the beginning of the
time step, we assume that $\Ub$ and $\phi$ are defined consistently,
i.e., $\rho^n$ and $\phi^n$ satisfy equation (\ref{eq:Self
  Gravity}).\\

\castro\ also supports radiation (Chapter \ref{Chap:Radiation}) and
level sets (Chapter \ref{Chap:Level Sets}).  We omit the details in
this section.  Here is the single-level algorithm:
\begin{description}
\item[Step 1:] {\em React $\Delta t/2$.}

Update the solution due to the effect of reactions over half a time step.
\begin{eqnarray}
(\rho E)^n &=& (\rho E)^n - \frac{\dt}{2}\sum_k(\rho q_k\omegadot_k)^n,\\
(\rho X_k)^n &=& (\rho X_k)^n + \frac{\dt}{2}(\rho\omegadot_k)^n.
\end{eqnarray}
\item[Step 2:] {\em Solve for gravity.}

Solve for gravity using:
\begin{equation}
\gb^n = \nabla\phi^n, \qquad 
\Delta\phi^n = 4\pi G\rho^n,
\end{equation}
or use one of the simpler gravity types.

\item[Step 3:] {\em Compute explicit source terms.}

We now compute explicit source terms for each variable in $\Qb$ and
$\Ub$.  The primitive variable source terms will be used to construct
time-centered fluxes.  The conserved variable source will be used to
advance the solution.  We neglect reaction source terms since they are
accounted for in {\bf Steps 1} and {\bf 6}.  The source terms are:
\begin{equation}
\Sb_{\Qb}^n =
\left(\begin{array}{c}
S_\rho \\
\Sb_{\ub} \\
S_p \\
S_{\rho e} \\
S_{A_k} \\
S_{X_k} \\
S_{Y_k}
\end{array}\right)^n
=
\left(\begin{array}{c}
S_{{\rm ext},\rho} \\
\gb + \frac{1}{\rho}\Sb_{{\rm ext},\rho\ub} \\
\frac{1}{\rho}\frac{\partial p}{\partial e}S_{{\rm ext},\rho E} + \frac{\partial p}{\partial\rho}S_{{\rm ext}\rho} \\
\nabla\cdot\kappa\nabla T + S_{{\rm ext},\rho E} \\
\frac{1}{\rho}S_{{\rm ext},\rho A_k} \\
\frac{1}{\rho}S_{{\rm ext},\rho X_k} \\
\frac{1}{\rho}S_{{\rm ext},\rho Y_k}
\end{array}\right)^n,
\end{equation}
\begin{equation}
\Sb_{\Ub}^n =
\left(\begin{array}{c}
\Sb_{\rho\ub} \\
S_{\rho E} \\
S_{\rho A_k} \\
S_{\rho X_k} \\
S_{\rho Y_k}
\end{array}\right)^n
=
\left(\begin{array}{c}
\rho \gb + \Sb_{{\rm ext},\rho\ub} \\
\rho \ub \cdot \gb + \nabla\cdot\kappa\nabla T + S_{{\rm ext},\rho E} \\
S_{{\rm ext},\rho A_k} \\
S_{{\rm ext},\rho X_k} \\
S_{{\rm ext},\rho Y_k}
\end{array}\right)^n.
\end{equation}

\item[Step 4:] {\em Advect $\Delta t$.}

The goal is to advance
\begin{equation}
\Ub^{n+1} = \Ub^n - \dt\nabla\cdot\Fb^\nph + \dt\Sb^n.
\end{equation}
neglecting reaction terms.  Note that since the source term is not
time centered, this is not a second-order method.  After the advective
update, we correct the solution, effectively time-centering the source
term. The advection step is complicated, and more detail is given in
Section \ref{Sec:Advection Step}.  Here is the summarized version:
\begin{enumerate}
\item Compute primitive variables.
\item Predict primitive variables to time-centered edges.
\item Solve the Riemann problem.
\item Compute fluxes and update.
\end{enumerate}
\item[Step 4:] {\em Solve for updated gravity.}

Solve for gravity using:
\begin{equation}
\gb^{n+1} = \nabla\phi^{n+1}; \qquad \Delta\phi^{n+1} = 4\pi G\rho^{n+1},
\end{equation}
or use one of the simpler gravity types.
\item[Step 6:] {\em Correct solution with time-centered source terms.}

We need to correct the solution by effectively time-centering the
source terms.  These corrections are to be performed sequentially
since new source term evaluations may depend on previous corrections.

First, we correct the solution with the updated gravity:
\begin{eqnarray}
(\rho\ub)^{n+1} &=& (\rho\ub)^{n+1} + \frac{\dt}{2}\left[(\rho\gb)^{n+1} - (\rho\gb)^n\right], \\
(\rho E)^{n+1} &=& (\rho E)^{n+1} + \frac{\dt}{2}\left[\left(\rho\ub\cdot\gb\right)^{n+1} - \left(\rho\ub\cdot\gb\right)^n\right].
\end{eqnarray}

Next, we correct $\Ub$ with updated external sources.  For example,
for the momentum, we correct using
\begin{equation}
(\rho\ub)^{n+1} = (\rho\ub)^{n+1} + \frac{\dt}{2}\left(\Sb_{{\rm ext},\rho\ub}^{n+1} - \Sb_{{\rm ext},\rho\ub}^n\right).
\end{equation}
We correct $\rho E, \rho A_k, \rho X_k$, and $\rho Y_k$ in an
analogous manner.

Finally, we correct the solution with updated thermal diffusion using
\begin{equation}
(\rho E)^{n+1} = (\rho E)^{n+1} + \frac{\dt}{2}\left(\nabla\cdot\kappa\nabla T^{n+1} - \nabla\cdot\kappa\nabla T^n\right).
\end{equation}
\item[Step 7:] {\em React $\Delta t/2$.}

Update the solution due to the effect of reactions over half a time step.
\begin{eqnarray}
(\rho E)^{n+1} &=& (\rho E)^{n+1} - \frac{\dt}{2}\sum_k(\rho q_k\omegadot_k)^{n+1},\\
(\rho X_k)^{n+1} &=& (\rho X_k)^{n+1} + \frac{\dt}{2}(\rho\omegadot_k)^{n+1}.
\end{eqnarray}
\item[Step 8:] {\em Modify auxiliary variables.}

This is problem-dependent.  By default we treat the auxiliary
variables as advected quantities, so no additional steps are required.

\item[Step 9:] {\em Optionally, retry the timestep from Step 1.}

If the timestep that you took had a timestep that was not sufficient to
enforce the stability criteria that you would like to achieve, such as
the CFL criterion for hydrodynamics or the burning stability criterion
for reactions, you can retry the timestep by setting {\tt castro.use\_retry = 1}
in your inputs file. This will save the current state data at the beginning
of the level advance, and then if the criteria are not satisfied, will
reject that advance and start over from the old data, with a series of
subcycled timesteps that should be small enough to satisfy the criteria.
Note that this will effectively double the memory footprint on each level
if you choose to use it.

\end{description}
Thus concludes the single-level algorithm description.


There are a number of parameters that affect which physics is turned on:
\begin{itemize}
\item {\tt castro.do\_hydro}
\item {\tt castro.do\_react}
\end{itemize}


\section{Outline of {\tt Castro::advance()}}

\MarginPar{this seems out of date}

\noindent

if (doReact)

\hspace{.1in}  strangChem()

end if

if (doGrav)

\hspace{.1in}  define oldGravityVector

end if

if (Diffusion)

\hspace{.1in}  getOldDiffusionTerm()

end if

if (addExtSource)

\hspace{.1in}  getSource() at old time

end if

AdvanceSolution()

if (doGrav)

\hspace{.1in}  define newGravityVector

\hspace{.1in}  correct solution due to new gravity

end if

if (addExtSource)

\hspace{.1in}  getSource() at new time

\hspace{.1in}  correct solution due to new source

end if

if (Diffusion)

\hspace{.1in}  getNewDiffusionTerm()

\hspace{.1in}  correct solution due to new diffusion term

\hspace{.1in}  computeTemp()

end if

if (doReact)

\hspace{.1in}  strangChem()

end if

if (advanceAux)

\hspace{.1in}  advanceAux()

end if

if (LevelSet)

\hspace{.1in}  advanceLevelSet()

end if

