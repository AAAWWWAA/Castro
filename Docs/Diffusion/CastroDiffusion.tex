\label{ch:diffusion}

\section{Introduction}

Castro incorporates explicit thermal diffusion into the energy equation.  
In terms of the specific internal energy, $e$, this appears as:
\begin{equation}
\rho \frac{De}{Dt} + p \nabla \cdot U = \nabla \cdot k_\mathrm{th} \nabla T
\end{equation}
where $k_\mathrm{th}$ is the thermal conductivity, with units
$\mathrm{erg~cm^{-1}~s^{-1}~K^{-1}}$.

To see the similarity to the thermal diffusion equation, consider the special
case of constant conductivity, $k_\mathrm{th}$, and density, and assume an
ideal gas, so $e = c_v T$, where $c_v$ is the specific heat at constant volume.
Finally, ignore hydrodynamics, so $U = 0$.  This gives:
\begin{equation}
\frac{\partial T}{\partial t} = D \nabla^2 T
\end{equation}
where $D \equiv k_\mathrm{th}/(\rho c_v)$.  Solving this equation
explicitly requires a timestep limiter of
\begin{equation}
\Delta t_\mathrm{diff} \le \frac{1}{2} \frac{\Delta x^2}{D}
\end{equation}
(this is implemented in {\tt Castro/Source/Src\_nd/timestep.F90}).

Support for diffusion must be compiled into the code by setting
{\tt USE\_DIFFUSION = TRUE} in your {\tt GNUmakefile}.


The following parameter affects diffusion:
\begin{itemize}
\item {\tt castro.diffuse\_temp}:  enable thermal diffusion (0 or 1; default 0)
\end{itemize}

A pure diffusion problem (with no hydrodynamics) can be run by setting
\begin{verbatim}
castro.diffuse_temp = 1
castro.do_hydro = 0
\end{verbatim}

To complete the setup, a thermal conductivity must be specified.
A simple constant conductivity is provided in {\tt Conductivity/constant/}
and can be selected by setting
\begin{verbatim}
Conductivity_dir := constant
\end{verbatim}
in your GNUmakefile.  To set the value of the conductivity (e.g., to
$100$), you add to your {\tt probin} file's {\tt \&extern} namelist:
\begin{verbatim}
const_conductivity = 100.0
\end{verbatim}


The diffusion approximation breaks down at the surface of stars,
where the density rapidly drops and the mean free path becomes 
large.  In those instances, you should use the flux limited diffusion
module in Castro to evolve a radiation field.  However, if your
interest is only on the diffusion in the interior, you can use
the {\tt castro.diff\_cutoff\_density} parameter to specify a density,
below which, diffusion is not modeled.  This is implemented in the
code by zeroing out the conductivity and skipping the estimation
of the timestep limit in these zones.


The interface for the conductivity routine takes
the thermodynamic state through an {\tt eos\_t} type (you can assume
that $\rho$, $T$, and $X_k$ are set).  More complicated conductivities
can be created by modifying this simple version.

A simple test problem that sets up a Gaussian temperature profile 
and does pure diffusion is provided as {\tt diffusion\_test}.

